\documentclass[tikz,10pt]{standalone}

\usepackage[T2A]{fontenc}
\usepackage[utf8x]{inputenc}
\usepackage[russian]{babel}
% \usepackage[utf8x]{inputenc}
\usepackage{amsmath}
\usepackage{amssymb}

\usepackage{cmap,pgfplots,pgfplotstable}
\usetikzlibrary{arrows,calc}
\pgfplotsset{compat=newest}
\usepackage[outline]{contour}
\begin{document}

% \pgfplotstableread{data.tsv}\mytable
	\begin{tikzpicture}
		\begin{axis}[
			%%%%%%%%%%%%%%%%%%%%%%%%%%%%% НАСТРОЙКИ ГРАФИКА %%%%%%%%%%%%%%%%%%%%%%%%%%%%%
			height=8cm,
			% width=10.5cm,
			scale=0.7,
			% grid=both, 				% вКлючаем отоБражение сетКи на графиКе
			%
			xlabel={$U$}, 			% ПодПись оси Y
			ylabel={$y$}, 			% ПодПись оси X
			%
			major grid style={
				line width=0.5pt, 	% толщина основных линий сетКи
				draw=black!100, 		% цвет основных линий сетКи: 50% черного (80% Белого) 
				draw=none,
			},
			%
			minor grid style={
				line width=0.5pt, 	% толщина Промежуточных линий сетКи
				% draw=black!20,		% цвет Промежуточных линий сетКи
				draw=none,
			},
			%
			minor x tick num=0,		% Количество Промежуточных линий между основными
			minor y tick num=0,		% Количество Промежуточных линий между основными
			%
 			ticklabel style={
 				scale=0.80			% уменьшим размер ПодПисей метоК на осях
 			},    
 			%
 			%ticks=none,
	    	axis lines=middle, 		% выравнивание оси y:  middle (в нуле)|left|right
	    	%
			% ymin = 7,				% маКсимумы и минимумы осей на графиКе
			% ymax = 1.2,
			% ymax = 0.28,
			ymin = -0.7,
			ymax= 1.2,
    		x label style={
    		at={(current axis.right of origin)}, 
            xshift=1ex, anchor=center
    		},
			% enlargelimits=false,
			% ymin = 7,	
			% xmax = 8,
			% xmin=-2,
			% xmax=7.5,
			% ymin=-2.1,
			ymax=2.5,
			ymin=-1,
			%
			% xtick distance=1,		% расстояние между метКами По оси X
			% ytick distance=1/50,		% расстояние между метКами По оси Y
			xticklabels={},
			yticklabels={},
			xtick=\empty,
			ytick=\empty,
			disabledatascaling,
			% ymin={-11/16},
			% extra y ticks={-9/16,-6/16},	% доПолнительные метКи на осях
			extra x ticks={0,2,3},	% доПолнительные метКи на осях
			extra x tick labels={	% можно уКазать сПециальные ПодПиси К ним
			 		{$0$,$2V$,$3V$}
			 	}, 	
			axis x line=center,				
			%
			unit vector ratio = 1 1,% масштаБ 1:1 осей X и Y
			% x={(1cm/1.3,0cm)}, y={(0cm,50cm/1.3)},
			%
			% x axis line style ={draw=none},
			% x axis line style ={d},
			x axis line style ={black!40},
			y axis line style ={black!40},
	      % y tick/.style={
	      %   draw=none,
	      %   % semithick,
	      % },
			% x label style={
			% 	at={(axis description cs:1.05,0)},
			% 	anchor=center,		% расПоложение метКи ровно в точКе (1.1,0)
			% 	rotate=360,			% вооБще метКу еще можно Повернуть)
			% 	black				% цвет метКи
			% },
    		y label style={
    			% at={(axis description cs:0.01,1.07)},
    			yshift=0.5em,
    			anchor=center,		% расПоложение метКи ровно в точКе (0,1.1)
    			black				% цвет метКи
    		},			
			%							
			%%%%%%%%%%%%%%%%%%%%%%%%%%%%%%%%%%%%%%%%%%%%%%%%%%%%%%%%%s%%%%%%%%%%%%%%%%%%%%
			% stressstrainset
		]
		
		\addplot[domain={-3:5},samples=700] {x^3/6-x^2/2};

     \end{axis}
	\end{tikzpicture}	
\end{document}