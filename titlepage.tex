%!TEX root = lections.tex
\begin{titlepage}
\thispagestyle{empty}

\begin{center}
	{\small\textsc{Нижегородский государственный университет имени Н.\,И. Лобачевского}}
	\vskip 3pt \hrule \vskip 5pt
	{\small\textsc{Радиофизический факультет}}

	\vfill

	\begin{spacing}{2}
	% {\huge \bf  Курин\, В.В.}\\[1.5em]
	{\Huge \bf  Лекции по основам \\ теории колебаний}\\%\vspace{1em}
	\end{spacing}
	{ Набор и вёрстка:}\\[.5em]
	{ 	\href{https://github.com/AnnaKarusevich}{\color{black}{Карусевич А.А.}}
		\href{https://github.com/KirillPonur}{\color{black}{Понур К.А.}},
		\href{https://github.com/fedorsarafanov}{\color{black}{Сарафанов Ф.Г.}}, 
		\href{https://github.com/BigBigGamer}{\color{black}{Шиков А.П.}},
	\\ Платонова М.В.}\\[2em]
	% {\large }\\
	\vspace{1em}
\end{center}

\textbf{Disclaimer.} В данном документе нами набраны лекции по теории колебаний (нелинейной динамике), прочитанные на 3 курсе радиофизического факультета ННГУ \textbf{Владимиром Исааковичем Некоркиным}, но не вошедшие в существующие методические пособия. Документ призван облегчить подготовку к зачётам и экзаменам и восполнить пробелы в знаниях читателя по теории колебаний. Разрешено копирование и распространение данного документа с обязательным указанием первоисточника. 

% При обнаружении ошибок, опечаток и прочих вещей, требующих исправления, можно либо создать issues в \href{https://github.com/FedorSarafanov/NonlinearDynamics}{репозитории на github.com}, либо написать на электронную почту  \href{mailto:sfg180@yandex.ru}{\color{black}{sfg180@yandex.ru}}.

\begin{center}
	\vfill
	4 апреля -- \today\\Нижний Новгород
\end{center}

\end{titlepage}
