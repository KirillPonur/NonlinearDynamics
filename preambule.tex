\documentclass[a4paper,14pt]{extarticle}

% Шрифты, кодировки, символьные таблицы, переносы
\usepackage{cmap}
\usepackage[T2A]{fontenc}
\usepackage[utf8x]{inputenc}
\usepackage[english, russian]{babel}

% Пакеты американского математического сообщества
\usepackage{amssymb,amsfonts,amsmath,amsthm}  
% Сокращения
\usepackage{cancel}

\theoremstyle{definition}
\newtheorem{definition}{Определение}

% Красная строка
\usepackage{indentfirst}

% Ссылки в pdf
\usepackage[unicode, colorlinks, urlcolor=magenta, linkcolor=black]{hyperref}

% Таблицы
\usepackage{makecell,multirow} 

% Графика
\usepackage{graphicx}
\usepackage[usenames,dvipsnames]{color} 
\usepackage{float}
% \usepackage{subcaption}

% Геометрия страницы
\usepackage{geometry}
\geometry{left=2cm,right=2cm,top=2.5cm,bottom=2.5cm,bindingoffset=0cm,headheight=18pt}

% Колонтитулы
\usepackage{fancyhdr} 
% применим колонтитул к стилю страницы
\pagestyle{fancy} 
%очистим "шапку" страницы
\fancyhead{} 
%слева сверху на четных и справа на нечетных
\fancyhead[R]{Лекции В.И. Некоркина 2018-2019} 
% \fancyhead[R]{Сарафанов Ф.Г., Понур К.А. и др.} 
%справа сверху на четных и слева на нечетных
\fancyhead[L]{Теория колебаний} 
%очистим "подвал" страницы
\fancyfoot{} 
% номер страницы в нижнем колинтуле в центре
\fancyfoot[C]{\thepage} 

% Межстрочный отступ
\usepackage{setspace}
\linespread{1.15} % капельку увеличенный
\frenchspacing % <<французские>> пробелы

% Нумерация
\renewcommand{\labelenumii}{\theenumii)}
% В заголовках появляется точка, но при ссылке на них ее нет
\usepackage{misccorr}

% Содержание
\usepackage{tocloft}
\usepackage{secdot}
\sectiondot{subsection}

% Физика
\usepackage{physics}

% Новые команды
\newcommand{\Mean}[1]{\langle#1\rangle}
\newcommand{\Defi}{\underset{def}{=}}
\newcommand{\Inte}{\int\limits_{-\infty}^{\infty}} 

\addto\captionsrussian{%
	\renewcommand{\contentsname}{Оглавление}
	\renewcommand{\partname}{Раздел}%
}
\def\thepart{\arabic{part}}
\usepackage{tocloft}
\renewcommand{\cftpartleader}{\cftdotfill{\cftdotsep}} % for parts
% \renewcommand{\cftchapleader}{\cftdotfill{\cftdotsep}} % for chapters
\renewcommand{\cftsecleader}{\cftdotfill{\cftdotsep}} % for chapters
% \newlength\mylen
\renewcommand\thepart{\arabic{part}.}
% \renewcommand\cftpartpresnum{Лекция~}
% \renewcommand\cftsecpresnum{Лекция~}

% \setlength{\cftsecnumwidth}{6em}
% \renewcommand{\cftsecpresnum}{Лекция\ }
\renewcommand{\cftsecaftersnum}{.}

% \renewcommand{\cftsecaftersnumb}{\newline}
\renewcommand{\cftsecdotsep}{\cftdotsep}
\renewcommand{\kappa}{\varkappa}
\renewcommand{\phi}{\varphi}
\renewcommand{\epsilon}{\varepsilon}

% #1: math symbol
% #2: legend
\def\alegend#1#2{\overset{\underset{\scriptstyle\downarrow}{\scriptstyle\text{#2}}}{#1}}
\def\blegend#1#2{\underset{\underset{\scriptstyle\text{#2}}{\scriptstyle\uparrow}}{#1}}
\def\hp{\hat{p}}
\def\hx{\hat{x}}
\def\hH{\hat{H}}

\usepackage[explicit]{titlesec}
% \titleformat{\section}{\normalfont\Large\bfseries}{}{0em}{Лекция\ \thesection.\ #1}
\usepackage{epigraph}


\newcommand\praktika[1]{
\stepcounter{section}
\vspace{1.5em}
\noindent\textbf{\Large{Занятие \arabic{section}.\hspace{.2em} #1}}
% \newline 
\vspace{-0.5em}
\addcontentsline{toc}{section}{Занятие \arabic{section}.\hspace{.5em} #1}
}

% \usepackage{mathtools}
% \mathtoolsset{showonlyrefs=true}


% https://tex.stackexchange.com/questions/8720/overbrace-underbrace-but-with-an-arrow-instead

\usepackage{xparse}% http://ctan.org/pkg/xparse

\NewDocumentCommand{\overarrow}{O{=} O{\uparrow} m}{%
  \overset{\makebox[0pt]{\begin{tabular}{@{}c@{}}$#3$\\[0pt]\ensuremath{#2}\end{tabular}}}{#1}
}
\NewDocumentCommand{\underarrow}{O{=} O{\downarrow} m}{%
  \underset{\makebox[0pt]{\begin{tabular}{@{}c@{}}\ensuremath{#2}\\[0pt]$#3$\end{tabular}}}{#1}
}

\newcommand\undernoteqty[2]{
	%
	\underarrow[
		\qty(\underbrace{#1})
	][\uparrow]{\substack{#2}}
	%
}

\newcommand{\pvec}[1]{\vec{#1}\mkern2mu\vphantom{#1}}
% Нормальный вектор для штрихов
\newcommand{\phat}[1]{\hat{#1}\mkern2mu\vphantom{#1}}

\newcommand\undernote[2]{
	%
	\underarrow[
		#1
	][\uparrow]{\substack{#2}}
	%
}



% ##############################################################################
% ##############################################################################
\newcommand*\dotvec[1][1,1]{\crossproducttemp#1\relax}
\def\crossproducttemp#1,#2\relax{{\qty[\vec{#1}\times\vec{#2}\,]}}

\newcommand*\prodvec[1][1,1]{\crossproducttempa#1\relax}
\def\crossproducttempa#1,#2\relax{{\qty[{#1}\times{#2}\,]}}
% ##############################################################################
% ##############################################################################

