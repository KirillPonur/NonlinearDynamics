\documentclass[tikz,10pt]{standalone}

\usepackage[T2A]{fontenc}
\usepackage[utf8x]{inputenc}
\usepackage[russian]{babel}
% \usepackage[utf8x]{inputenc}
\usepackage{amsmath}
\usepackage{amssymb}

\usepackage{cmap,pgfplots,pgfplotstable}
\usetikzlibrary{arrows,calc}
\usetikzlibrary{intersections}
\pgfplotsset{compat=newest}
\usepackage[outline]{contour}
\newcommand{\hevi}[2]{(x<#1?0:1)*(x>#2?0:1)}
\begin{document}

% \pgfplotstableread{data.tsv}\mytable

	\begin{tikzpicture}
		\begin{axis}[
			%%%%%%%%%%%%%%%%%%%%%%%%%%%%% НАСТРОЙКИ ГРАФИКА %%%%%%%%%%%%%%%%%%%%%%%%%%%%%
			height=5cm,
			% width=10.5cm,
			scale=1,
			% grid=both, 				% вКлючаем отоБражение сетКи на графиКе
			%
			xlabel={$k$}, 			% ПодПись оси Y
			ylabel={$\omega$}, 			% ПодПись оси X
			%
			major grid style={
				line width=0.5pt, 	% толщина основных линий сетКи
				% draw=black!50, 		% цвет основных линий сетКи: 50% черного (80% Белого) 
				draw=none,
			},
			%
			minor grid style={
				line width=0.5pt, 	% толщина Промежуточных линий сетКи
				% draw=black!20,		% цвет Промежуточных линий сетКи
				draw=none,
			},
			%
			% minor x tick num=0,		% Количество Промежуточных линий между основными
			% minor y tick num=0,		% Количество Промежуточных линий между основными
			%
 			ticklabel style={
 				scale=0.95			% уменьшим размер ПодПисей метоК на осях
 			},    
 			%
 			% ticks=none,
	    	axis lines=middle, 		% выравнивание оси y:  middle (в нуле)|left|right
	    	%
			% ymin = 7,% маКсимумы и минимумы осей на графиКе
			% ymax = 10,
			% ymax = 0.28,
			% ymin = -2,
			% ymax= 1.5,
    		x label style={
    		at={(current axis.right of origin)}, 
            xshift=1ex, anchor=center
    		},
			enlargelimits=true,
			ymin = 0,	
			% xmax = 8,
			% xmin=-3*pi/2*0.8,
			% xmax=2*pi,
			% xmin=-2*pi,
			% xmax=7.5,
			% ymin=-2.1,
			% ymax=0.5,
			% ymin=-2.25,
			 % clip=false,
		  restrict y to domain=-20:20,
		  unbounded coords=jump,
			%
			% xtick distance=1,		% расстояние между метКами По оси X
			% ytick distance=1/50,		% расстояние между метКами По оси Y
				xticklabels={},
				yticklabels={},
				xtick=\empty,
				ytick=\empty,
			% disabledatascaling,
			% ymin={-11/16},
			% extra y ticks={-9/16,-6/16},	% доПолнительные метКи на осях
			% extra x ticks={-3.14,3.14},	% доПолнительные метКи на осях
			% extra x tick labels={	% можно уКазать сПециальные ПодПиси К ним
			% 		{$-\frac{\pi}{a}$},
			% 		{$\frac{\pi}{a}$}
			% 	}, 		
			% extra y ticks={1},	% доПолнительные метКи на осях
			% extra y tick labels={	% можно уКазать сПециальные ПодПиси К ним
			% 		{$\omega_0\,\,\,\,$},
			% 		% {$\omega_{\max}$}
			% 	}, 					
			%
			% unit vector ratio = 2.5 1,% масштаБ 1:1 осей X и Y
			% x={(1cm/1.3,0cm)}, y={(0cm,50cm/1.3)},
			%
			% x axis line style ={draw=none},
			% x axis line style ={d},
			x axis line style ={black!40},
			y axis line style ={black!40},
	      % y tick/.style={
	      %   draw=none,
	      %   % semithick,
	      % },
			% x label style={
			% 	at={(axis description cs:1.05,0)},
			% 	anchor=center,		% расПоложение метКи ровно в точКе (1.1,0)
			% 	rotate=360,			% вооБще метКу еще можно Повернуть)
			% 	black				% цвет метКи
			% },
    		y label style={
    			% at={(axis description cs:0.01,1.07)},
    			yshift=1em,
    			anchor=center,		% расПоложение метКи ровно в точКе (0,1.1)
    			black				% цвет метКи
    		},			
			%							
			%%%%%%%%%%%%%%%%%%%%%%%%%%%%%%%%%%%%%%%%%%%%%%%%%%%%%%%%%s%%%%%%%%%%%%%%%%%%%%
			% stressstrainset
		]
		
		% \addplot[dashed, blue!50,domain={-3*pi/2:3*pi/2},samples=1000] {tan(deg(x))};

		% \addplot[black, domain={-3*pi/2:3*pi/2},samples=1000] {x*tan(deg(x))};

		% % \pgfplotsinvokeforeach{-1,1,3,-3}{%
		% % 	\draw[dashed, black!20] ({axis cs:pi/2*#1,0}|-{rel axis cs:0,0}) -- ({axis cs:pi/2*#1,0}|-{rel axis cs:0,1});
		% % }

		% \begin{scope}[opacity=0.5]
		% \xdef\w{(pi/2+pi/4)}
		% \addplot[red, domain={-\w:\w},samples=1000] {2.5*sqrt(\w^2-x^2)};

		% \draw[fill=black] (1.315,4.9) circle (0.5pt);
		% \draw[fill=black] (-1.315,4.9) circle (0.5pt);
		% \draw[dashed,black] (-1.315,4.9) -- (1.315,4.9);			
		% \end{scope}



		% \xdef\w{(pi+pi/4)}
		% \addplot[red, domain={-1.5*pi:1.5*pi},samples=1000] {sqrt(x^2+1)};

		\addplot[domain={-1.5*pi:1.5*pi},samples=1000,red] {sqrt(x^2)};% node [below,pos=0.75,sloped] {$\omega=\pm vk$};

		% \draw[dashed, black!35] (pi,0) -- (pi,{sqrt(pi^2+1)});
		% \draw[dashed, black!35] (-pi,0) -- (-pi,{sqrt(pi^2+1)});
		% \def\H{3.77}\def\X{2.69}
		% \draw[fill=black,dashed] (\H,\X) circle (0.5pt) -- (-\H,\X) circle (0.5pt);

		% \def\HH{1.43}\def\XX{9.19}
		% \draw[fill=black,dashed] (\HH,\XX) circle (0.5pt) -- (-\HH,\XX) circle (0.5pt);

		% \draw (0,4) node[left] {$E_1$};


		% % \draw (-3,1) -- node[above] {$U_1$} (-2,1) -- (-2,0) -- node[below] {$U_2$} (-1,0) -- (-1,1.5) -- node[above] {$U_3$} (0.5,1.5) -- (0.5,1) -- node[above] {$U_4$} (2,1);

		% \draw (-3,0) -- node[above] {$0$} (-1,0) -- (-1,-1.5) -- node[below] {$U_0$} (1,-1.5) -- (1,0) -- node[above] {$0$} (3,0); 

		% \draw[fill=magenta!20] (-2,-0.75) circle (0.5em) node {1};
		% \draw[fill=magenta!20] (2,-0.75) circle (0.5em) node {1};
		% \draw[fill=magenta!20] (0,-0.75) circle (0.5em) node {2};
		% % \draw[fill=magenta!20] (-1.5,0.5) circle (0.5em) node {2};
		% % \draw[fill=magenta!20] (-.25,0.5) circle (0.5em) node {3};
		% % \draw[fill=magenta!20] (1.25,0.5) circle (0.5em) node {4};
		% % \draw[fill=magenta!20] (-2.5,0.5) circle (0.65em) node {1};

     \end{axis}
	\end{tikzpicture}	
\end{document}