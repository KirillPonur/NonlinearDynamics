\documentclass[tikz,10pt]{standalone}

\usepackage[T2A]{fontenc}
\usepackage[utf8x]{inputenc}
\usepackage[russian]{babel}
% \usepackage[utf8x]{inputenc}
\usepackage{amsmath}
\usepackage{amssymb}

\usepackage{cmap,pgfplots,pgfplotstable}
\usetikzlibrary{arrows,calc}
\usetikzlibrary{intersections,patterns}
\pgfplotsset{compat=newest}
\usepackage[outline]{contour}
% \newcommand{\hevi}[2]{(x<#1?0:1)*(x>#2?0:1)}

% Draw line annotation
% Input:
%   #1 Line offset (optional)
%   #2 Line angle
%   #3 Line length
%   #5 Line label
% Example:
%   \lineann[1]{30}{2}{$L_1$}
\newcommand{\lineann}[4][0.5]{%
    \begin{scope}[rotate=#2, blue,inner sep=2pt]
        \draw[dashed, blue!40] (0,0) -- +(0,#1)
            node [coordinate, near end] (a) {};
        \draw[dashed, blue!40] (#3,0) -- +(0,#1)
            node [coordinate, near end] (b) {};
        \draw[|<->|] (a) -- node[fill=white] {#4} (b);
    \end{scope}
}

\usepackage{pgfplots}
\pgfplotsset{compat=1.11}
\usepgfplotslibrary{fillbetween}
\usetikzlibrary{intersections}

\pgfdeclarelayer{bg}
\pgfsetlayers{bg,main}
\begin{document}
\begin{tikzpicture}[scale=1.5]
% \draw[red,name path = A] (0,0) rectangle (3,3);

% \draw (2,5)--(2,-1);
\draw[fill = blue!15, draw = none,scale=1,name path =B, domain=0:3,smooth,variable=\x] plot ({\x},{sin(deg(2*pi*\x/2))/5+1.5}) -- (3,0) -- (0,0) -- cycle;
\draw[scale=1,blue!50!magenta,name path =B, domain=0:3,smooth,variable=\x] plot ({\x},{sin(deg(2*pi*\x/2))/5+1.5});

\draw[<->] (1.5,0) -- node[fill=blue!15] {$h(z)$} (1.5,{sin(deg(2*1.5*pi/2))/5+1.5});
%\fillbetween[of=A and B];
%\draw[fill] (2,2) circle [radius=0.1];
% \begin{pgfonlayer}{bg}
% \fill [orange!50,
%           intersection segments={
%             of=A and B,
%             % sequence={L1--R1}
%           }];
% \end{pgfonlayer}

% \foreach \x in {0,0.01,...,3}{
%     \draw[fill=blue!10,draw=none] (\x,0) rectangle (\x+0.02,{sin(deg(2*pi*\x/2))/5+1.5});
% }

\draw[->] (-0.1,0) -- (3.2,0) node [right] {$x$};
\draw[->] (0,-0.1) -- (0,2.2) node [above] {$z$};

\draw[-] (-0.1,1.5) node [left] {$h_0$} -- ++(0.2,0);


% \foreach \y in {30,60}{
% 	\draw[white] (\y:0) -- (\y:1) node[pos=1.35, scale=0.7,black] {$n=\pgfmathparse{int(\y/30)}\pgfmathresult$};
% }

% \foreach \y in {0}{
% 	\draw[white] (\y:0) -- (\y:1) node[pos=1,right=0.5em, scale=0.7,black] {$n=\pgfmathparse{int(\y/30)}\pgfmathresult, n=N$};
% }

% \draw (0,0) circle (1);
% \foreach \y in {0,30,...,330}{
% 	\draw (\y:0.4) coordinate (u) -- (\y:1);% node[pos=1.5, scale=0.7] {$n=\pgfmathparse{int(\y/30)}\pgfmathresult$};
% 	\draw[fill, magenta] (u) circle (1pt);
% 	\draw[<->,rotate=\y,black!60] (0.7,-0.1) -- ++ (0,0.2);
% }

%     \xdef\r{1}
%     \xdef\dr{0}
%     \xdef\step{5}
%     \foreach \i [evaluate=\i as \To using \r+\dr*rand,
%                    remember=\To as \From (initially \r)] in {0,\step,...,355} {
%         \draw[blue] (\i:\From) -- (\i-3:\From+0.08); 
%         % \ifnum \i<355
%            % \draw (\i:\From) to[out=90+\i,in=-90+\i+\step] (\i+\step:\To); 
%         % \else 
%            % \draw (\i:\From) to[out=90+\i,in=-90+\i+\step] (\i+\step:\r);
%         % \fi  
        
%         % \pgfmathparse{int(mod(\i,10))}
%         % \let\ress\pgfmathresult
%         % \ifnum \ress=0
%            % \draw[->,black!60] (\i:0) -- (\i:\From); 
%         % \else 
%            % \draw (\i:\From) to[out=90+\i,in=-90+\i+\step] (\i+\step:\r);
%         % \fi  
%     }

\begin{scope}[yshift=1.5cm,xshift=.5cm]
\lineann[0.5]{0}{2}{$\lambda$}
\end{scope}
% 	\draw[scale=1, thick,domain=0.01:1.99,smooth,variable=\x,red] plot ({\x},{sin(deg(2*pi*\x/2))/2});
% 	\draw[line width=1.5pt] (0,-0.7) -- (0,0.7);
% 	\draw[draw=none,pattern=north east lines,pattern color=blue, preaction={fill=white}] (0,-0.7) rectangle (-0.1,0.7); 

% 	\draw[line width=1.5pt] (2,-0.7) -- (2,0.7);
% 	\draw[draw=none,pattern=north east lines,pattern color=blue, preaction={fill=white}] (2,-0.7) rectangle (2.1,0.7); 


% \end{scope}
% \draw (4,-1.5) node {б)};
% \draw (0,-1.5) node {a)};

\end{tikzpicture}	
\end{document}