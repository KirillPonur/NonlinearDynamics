%!TEX root = ..\lections.tex

\section{Двукратный предельный цикл}%
\label{sec:9.1}

Рассмотрим систему на фазовой плоскости, правые части которой зависят от управляющего параметра $\mu$ (см. \ref{lect8}, система \eqref{eq:8.1}). Предположим, что система \eqref{eq:8.1} имеет предельный цикл $L_0$. В малой окрестности $L_0$ траектории системы \eqref{eq:8.1} порождает отображение Пуанкаре
(см. рис.\ref{fig:9.1}), которое можно представить в следующем виде
\begin{equation}
        \label{eq:9.1}
        \bar \xi = g(\xi,\mu).
\end{equation}
Без ограничения общности будем считать, что начало координат на секущей
Пуанкаре выбрано в неподвижной точке, то есть $g(0,\mu)=0$. Пусть при $\mu=0$
мультипликатор предельного цикла $L_0$ удовлетворяет условию
\begin{equation}
        \label{eq:9.2}
        s(0) = \pdv{g}{\xi} \eval_{(0,0)} = 1.
\end{equation}
\begin{figure}[h]
        \centering
        \includegraphics[width=0.6\linewidth]{example-image-a}
        \caption{Отображение Пуанкаре в окрестности предельного цикла $L_0$.}
        \label{fig:9.1}
\end{figure}
Раскладывая $g(\xi,\mu)$ в ряд Тейлора в окрестности точки $(0,0)$ получим
\begin{align}
        \label{eq:9.3}
        g(\xi,\mu) = g(0,0) + g'_{\xi}(0,0) \xi + g'_{\mu}(0,0) \mu 
        + &\frac{1}{2}g''_{\xi\xi}(0,0) \xi^2+ \\
        + &g''_{\xi\mu}(0,0) \xi \mu + \frac{1}{2} g''_{\mu\mu}(0,0)\mu^2+ \dots
\end{align}
Из \eqref{eq:9.1} - \eqref{eq:9.3} следует следующее представление отображение Пуанкаре в окрестности $L_0$ 
\begin{equation}
        \label{eq:9.4}
        \bar \xi = \alpha(\mu) + \xi( 1+ \beta(\mu)) + \gamma \xi^2 + \dots,
\end{equation}
где
\begin{align}
        \label{eq:}
        \alpha(\mu) = g'_{\mu}(0,0)\mu+\dots &, &\beta(\mu) = g''_{\xi\mu}(0,0) \mu + \dots \\
        \alpha(0) = \beta(0) = 0 &, &\gamma= \frac{1}{2} g''_{\xi\xi}(0,0)+ \dots
\end{align}
Предположим, что
\begin{equation}
        \label{eq:9.5}
        \gamma \neq 0.
\end{equation}
Таким образом, мы имеем одно бифуркационное условие \eqref{eq:9.2} и одно условие невырожденности
\eqref{eq:9.5}, а нормальная форма для бифуркации двукратный предельный цикл задается уравнением \eqref{eq:9.4}.

Проведем исследование отображения \eqref{eq:9.4}. Рассмотрим два случая.

\paragraph{Случай $\gamma>0.$}%
Прежде всего изучим свойства функции последования $g(\xi,\mu)$ отображения
\eqref{eq:9.4}. Непосредственно из \eqref{eq:9.4} имеем
\begin{equation}
        \label{eq:9.6}
        \begin{gathered}
                g'_{\xi} = 1 + \beta(\mu) + 2 \gamma \xi+ \dots \\
                g''_{\xi\xi} = 2 \gamma + \dots
        \end{gathered}
\end{equation}
Поскольку мы рассматриваем \eqref{eq:9.4} в окрестности точки $(\xi,\mu)=(0,0)$ и $\beta(0)=0,$ 
в силу \eqref{eq:9.6} получаем что
\begin{equation}
        \label{eq:}
        g'_{\xi}>0,\quad g''_{\xi\xi}>0.
\end{equation}
Следовательно, $g(\xi,\mu)$ -- монотонно возрастающая функция, выпуклая вниз.
Координаты неподвижных точек отображения \eqref{eq:9.4} определяются уравнением
\begin{equation}
        \label{eq:9.7}
        0 = \alpha(\mu) + \beta(\mu) \xi + \gamma(\mu) \xi^2 + \dots
\end{equation}
Пусть для определенности функция $\alpha(\mu)$ удовлетворяет условию $\alpha(\mu)\cdot \mu>0, \mu \neq 0.$ Тогда из \eqref{eq:9.7} получаем, что при $\mu=0$ отображение \eqref{eq:9.4} имеет единственную неподвижную точку $O_0 (\xi=0+\dots)$, а при $\mu<0$ -- две неподвижные точки:
$\displaystyle O_1 \qty( \xi = - \sqrt{ -\frac{\alpha(\mu)}{\gamma}} + \dots )$ и
$\displaystyle O_2 \qty( \xi = \sqrt{ -\frac{\alpha(\mu)}{\gamma}}+\dots)$. 
При $m>0$ отображение \eqref{eq:9.4} неподвижных точек не имеет. Легко видеть, что $O_1$ является устойчивой, $O_2$ -- неустойчивой, $O_0$ -- полуустойчивой неподвижными точками.
На рис.\ref{fig:9.2}а представлен вид отображения \eqref{eq:9.4} для различных значений параметра $\mu$,
вытекающий из установленных выше свойств этого отображения, а на рис.\ref{fig:9.2}b, соответствующие
фазовые портреты системы \eqref{eq:8.1}. При $\mu=0$ существует двухкратный (полуустойчивый) предельный цикл $L_0$, который при $\mu<0$ распадается на два грубых -- устойчивый $L_1$ и неустойчивый $L_2$.
При увеличении $\mu$ от нуля в сторону $\mu>0$ цикл $L_0$ исчезает.

