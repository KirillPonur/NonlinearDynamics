%!TEX root = ..\lections.tex

\section{Двукратный предельный цикл}%
\label{sec:9.1}

Рассмотрим систему на фазовой плоскости, правые части которой зависят от управляющего параметра $\mu$ (см. \ref{lect8}, система \eqref{eq:8.1}). Предположим, что система \eqref{eq:8.1} имеет предельный цикл $L_0$. В малой окрестности $L_0$ траектории системы \eqref{eq:8.1} порождает отображение Пуанкаре
(см. рис.\ref{fig:9.1}), которое можно представить в следующем виде
\begin{equation}
        \label{eq:9.1}
        \bar \xi = g(\xi,\mu).
\end{equation}
Без ограничения общности будем считать, что начало координат на секущей
Пуанкаре выбрано в неподвижной точке, то есть $g(0,\mu)=0$. Пусть при $\mu=0$
мультипликатор предельного цикла $L_0$ удовлетворяет условию
\begin{equation}
        \label{eq:9.2}
        s(0) = \pdv{g}{\xi} \eval_{(0,0)} = 1.
\end{equation}
\begin{figure}[h]
        \centering
        \includegraphics[width=0.6\linewidth]{example-image-a}
        \caption{Отображение Пуанкаре в окрестности предельного цикла $L_0$.}
        \label{fig:9.1}
\end{figure}
Раскладывая $g(\xi,\mu)$ в ряд Тейлора в окрестности точки $(0,0)$ получим
\begin{align}
        \label{eq:9.3}
        g(\xi,\mu) = g(0,0) + g'_{\xi}(0,0) \xi + g'_{\mu}(0,0) \mu 
        + &\frac{1}{2}g''_{\xi\xi}(0,0) \xi^2+ \\
        + &g''_{\xi\mu}(0,0) \xi \mu + \frac{1}{2} g''_{\mu\mu}(0,0)\mu^2+ \dots
\end{align}
Из \eqref{eq:9.1} - \eqref{eq:9.3} следует следующее представление отображение Пуанкаре в окрестности $L_0$ 
\begin{equation}
        \label{eq:9.4}
        \bar \xi = \alpha(\mu) + \xi( 1+ \beta(\mu)) + \gamma \xi^2 + \dots,
\end{equation}
где
\begin{align}
        \label{eq:}
        \alpha(\mu) = g'_{\mu}(0,0)\mu+\dots &, &\beta(\mu) = g''_{\xi\mu}(0,0) \mu + \dots \\
        \alpha(0) = \beta(0) = 0 &, &\gamma= \frac{1}{2} g''_{\xi\xi}(0,0)+ \dots
\end{align}
Предположим, что
\begin{equation}
        \label{eq:9.5}
        \gamma \neq 0.
\end{equation}
Таким образом, мы имеем одно бифуркационное условие \eqref{eq:9.2} и одно условие невырожденности
\eqref{eq:9.5}, а нормальная форма для бифуркации двукратный предельный цикл задается уравнением \eqref{eq:9.4}.

Проведем исследование отображения \eqref{eq:9.4}. Рассмотрим два случая.

\paragraph{Случай $\gamma>0.$}%
Прежде всего изучим свойства функции последования $g(\xi,\mu)$ отображения
\eqref{eq:9.4}. Непосредственно из \eqref{eq:9.4} имеем
\begin{equation}
        \label{eq:9.6}
        \begin{gathered}
                g'_{\xi} = 1 + \beta(\mu) + 2 \gamma \xi+ \dots \\
                g''_{\xi\xi} = 2 \gamma + \dots
        \end{gathered}
\end{equation}
Поскольку мы рассматриваем \eqref{eq:9.4} в окрестности точки $(\xi,\mu)=(0,0)$ и $\beta(0)=0,$ 
в силу \eqref{eq:9.6} получаем что
\begin{equation}
        \label{eq:}
        g'_{\xi}>0,\quad g''_{\xi\xi}>0.
\end{equation}
Следовательно, $g(\xi,\mu)$ -- монотонно возрастающая функция, выпуклая вниз.
Координаты неподвижных точек отображения \eqref{eq:9.4} определяются уравнением
\begin{equation}
        \label{eq:9.7}
        0 = \alpha(\mu) + \beta(\mu) \xi + \gamma(\mu) \xi^2 + \dots
\end{equation}
Пусть для определенности функция $\alpha(\mu)$ удовлетворяет условию $\alpha(\mu)\cdot \mu>0, \mu \neq 0.$ Тогда из \eqref{eq:9.7} получаем, что при $\mu=0$ отображение \eqref{eq:9.4} имеет единственную неподвижную точку $O_0 (\xi=0+\dots)$, а при $\mu<0$ -- две неподвижные точки:
$\displaystyle O_1 \qty( \xi = - \sqrt{ -\frac{\alpha(\mu)}{\gamma}} + \dots )$ и
$\displaystyle O_2 \qty( \xi = \sqrt{ -\frac{\alpha(\mu)}{\gamma}}+\dots)$. 
При $m>0$ отображение \eqref{eq:9.4} неподвижных точек не имеет. Легко видеть, что $O_1$ является устойчивой, $O_2$ -- неустойчивой, $O_0$ -- полуустойчивой неподвижными точками.
На рис.\ref{fig:9.2}а представлен вид отображения \eqref{eq:9.4} для различных значений параметра $\mu$,
вытекающий из установленных выше свойств этого отображения, а на рис.\ref{fig:9.2}b, соответствующие
фазовые портреты системы \eqref{eq:8.1}. При $\mu=0$ существует двухкратный (полуустойчивый) предельный цикл $L_0$, который при $\mu<0$ распадается на два грубых -- устойчивый $L_1$ и неустойчивый $L_2$.
При увеличении $\mu$ от нуля в сторону $\mu>0$ цикл $L_0$ исчезает.
\begin{figure}[h]
        \centering
        \includegraphics[width=0.6\linewidth]{example-image-a}
        \caption{Отображение Пуанкаре в случае $\gamma>0$ (a); фазовые портреты системы  \eqref{eq:8.1}, отвечающие этому отображению (b).}
        \label{fig:9.2}
\end{figure}

\paragraph{Случай $\gamma>0$}%

Прежде всего изучим свойства функции последования $g(\xi,\mu)$ отображения \eqref{eq:9.4}. Непосредственно из \eqref{eq:9.4} имеем
\begin{equation}
        \label{eq:9.6}
        \begin{gathered}
                g'_\xi = 1 + \beta(\mu) + 2\gamma\xi+\dots \\
                g''_{\xi\xi}=2\gamma + \dots.
        \end{gathered}
\end{equation}
Поскольку мы рассматриваем \eqref{eq:9.4} в окрестности точки $(\xi,\mu)=(0,0)$ и $\beta(0) = 0$,
в силу \eqref{eq:9.6} получаем, что
\begin{equation}
        \label{eq:}
        g'_\xi>0, g''_{\xi\xi}>0
\end{equation}
Следовательно, $g(\xi,\mu)$ - монотонно возрастающая функция, выпуклая вниз.
Координаты неподвижных точек отображения \eqref{eq:9.4} определяются уравнением
\begin{equation}
        \label{eq:9.7}
        0 = \alpha(\mu) + \beta(\mu)\xi + \gamma(\mu) \xi^2 + \dots
\end{equation}
Пусть для определенности функция $\alpha(\mu)$ удовлетворяет условию $\alpha(\mu)\cdot \mu>0$,
$\mu\neq 0$. Тогда из \eqref{eq:9.7} получаем, что при $\mu=0$ отображение \eqref{eq:9.4}
имеет единственную неподвижную точку $O_0(\xi=0+\dots)$, а при $\mu<0$ -- две неподвижные точки:
$\displaystyle O_1\qty(\xi= - \sqrt{ - \frac{\alpha(\mu)}{\gamma}} + \dots)$
и 
$\displaystyle O_2\qty(\xi=  \sqrt{ - \frac{\alpha(\mu)}{\gamma}} + \dots)$.
При $\mu>0$ отображение \eqref{eq:9.4} неподвижных точек не имеет. Легко видеть, что $O_1$ является устойчивой, $O_2$ -- неустойчивой, $O_0$-- полуустойчивой неподвижными точками. На рис.\ref{fig:9.2}а представлен вид отображения \eqref{eq:9.4} для различных значений параметра $\mu$, 
вытекающий из установленных выше свойств этого отображения, а на рис.\ref{fig:9.2}b, соответствующие
фазовые портреты системы \eqref{eq:8.1}. При $\mu=0$ существует двукратный (полуустойчивый)
предельный цикл $L_0$, который при $\mu<0$ распадается на два грубых -- устойчивый 
$L_1$ и неустойчивый $L_2$. При увеличении $\mu$ от нуля в сторону $\mu>0$ цикл $L_0$ исчезает.
\begin{figure}[h]
        \centering
        \begin{minipage}{\linewidth}
            \centering
            \includegraphics[width=0.6\linewidth]{example-image-a}                
        \end{minipage}
        \hfill
        \begin{minipage}{\linewidth}
            \centering
            \includegraphics[width=0.6\linewidth]{example-image-a} 
        \end{minipage}
        \caption{Отображение в случае $\gamma>0$ (a); фазовые портреты системы \eqref{eq:8.1},
        отвечающие этому отображению (b).}
        \label{fig:9.2}
\end{figure}
\paragraph{Случай $\gamma<0$.}%
В этом случае исследование \eqref{eq:9.4} можно провести полностью аналогично 
предыдущему. На рис.\ref{fig:9.3} представлен вид
отображения \eqref{eq:9.4} в этом случае и соответствующие фазовые портреты. 
Здесь при $\mu=0$ также существует двухкратный предельный цикл, однако грубые 
предельные циклы $L_1$ и $L_2$ появляются на фазовой плоскости в области $\mu>0$.
При этом цикл $L_2$ является устойчивым, а $L_1$ -- неустойчивым. В области $\mu<0$ 
предельные циклы не существуют.

\begin{figure}[h]
        \centering
        \begin{minipage}{\linewidth}
            \centering
            \includegraphics[width=0.6\linewidth]{example-image-a}                
        \end{minipage}
        \hfill
        \begin{minipage}{\linewidth}
            \centering
            \includegraphics[width=0.6\linewidth]{example-image-a} 
        \end{minipage}
        \caption{Отображение в случае $\gamma<0$ (a); фазовые портреты системы \eqref{eq:8.1},
        отвечающие этому отображению (b).}
        \label{fig:9.2}
\end{figure}

Заметим, что бифуркацию образования двукратного цикла также часто называют седло-узловой бифуркацией
циклов.

\paragraph{Пример}%
Рассмотрим систему в полярных координатах следующего вида
\begin{equation}
        \label{eq:9.8}
        \begin{cases}
                \dot \rho = \rho\qty[ - \mu - (\rho-1)^2],
                \dot \phi = \omega,
        \end{cases}
\end{equation}
где параметр $\omega>0$, а $\mu$ - контрольный параметр. Переменные в системе \eqref{eq:9.8}
разделены и их динамику можно анализировать отдельно друг от друга.
Из второго уравнения следует, что переменная $\phi$ совершает вращательные движения с частотой $\omega$. Эволюция переменной $\rho$ зависит от параметра $\mu$. При $\mu>0$ выполняется неравенство
$\dot \rho<0$ и, следовательно, любая траектория системы \eqref{eq:9.8} на фазовой плоскости $(x_1,x_2)$ $(x_1=\rho\cos \phi, x_2=\rho\sin \phi)$ имеет форму спирали, скручивающейся к состоянию
равновесия в начале координат.
Другими словами, на фазовой плоскости существует устойчивый фокус, притягивающий все 
траектории системы \eqref{eq:9.8} (см. рис.\ref{fig:9.4}c). При $\mu=0$ уравнение для $\rho$, кроме 
устойчивого равновесия $\rho=0$, имеет также полуустойчивое состояние равновесия $\rho=1$, которому на фазовой плоскости $(x_1,x_2)$ соответствует полуустойчивый предельный цикл системы \eqref{eq:9.8}
(см. рис.\ref{fig:9.4}b).
\begin{figure}[h]
        \centering
        \begin{minipage}{0.33\linewidth}
                \centering
                \includegraphics[width=0.6\linewidth]{example-image-a}

                (a)
        \end{minipage}
        \vfill
        \begin{minipage}{0.33\linewidth}
                \centering
                \includegraphics[width=0.6\linewidth]{example-image-a}
  
                (b)
        \end{minipage}
        \vfill
        \begin{minipage}{0.33\linewidth}
                \centering
                \includegraphics[width=0.6\linewidth]{example-image-a}

                (c)
        \end{minipage}
        \vfill
        \caption{Бифуркация двухкратный предельный цикл в системе \eqref{eq:9.8}.}
        \label{fig:9.4}
\end{figure}
При $\mu<0$ этот полуустойчивый цикл разваливается на два цикла -- устойчивый и неустойчивый,
амплитуды которых равны соответственно $\rho=1 + \sqrt{-\mu}$ и $\rho = 1 - \sqrt{-\mu}$ 
(см. рис.\ref{fig:9.4}a).

\section{Петля сепаратрис седла}%

Предположим, что система \eqref{eq:8.1} является диссипативной и имеет в начале
координат состояние равновесия седло $O_0$. Обозначим через $\lambda_1(\mu)<0$ 
и $\lambda_2(\mu) > 0$
характеристические показатели седла $O_0$. Пусть при $\mu=0$ одна из выходящих
сепаратрис седла возвращается при $t \to + \infty$ в точку $O_0$, тем самым образуя
траекторию $\Gamma_0$ двоякоасимптотическую к седлу (рис. \ref{fig:9.5}а) – так называемую
гомоклиническую траекторию. Поскольку к седлу асимптотически
приближаются только две траектории – устойчивые сепаратрисы, траектория $\Gamma_0$
может существовать только в том случае, если неустойчивая и устойчивая
сепаратрисы совпадают. Поэтому траекторию $\Gamma_0$ часто называют петлей
сепаратрис. Траектория $\Gamma_0$ является негрубой и при изменении параметра $\mu=0$ она
разрушается. Для характеристики взаимного расположения сепаратрис седла
введем так называемую \textbf{функцию расщепления сепаратрис}. Обозначим через
$M^u(x_1^u(\mu),0)$ - точку, в которой неустойчивая сепаратриса $W^u$ 
 первый раз
 пересекает ось абсцисс, а через $M^s(x_1^s(\mu),0)$ - точку первого пересечения
 этой осью устойчивой сепаратрисы $W^s$ (см. рис.\ref{fig:9.5}b).
\begin{figure}[h]
        \centering
        \begin{minipage}{0.33\linewidth}
                \centering
                \includegraphics[width=0.6\linewidth]{example-image-a}

                (a)
        \end{minipage}
        \vfill
        \begin{minipage}{0.33\linewidth}
                \centering
                \includegraphics[width=0.6\linewidth]{example-image-a}
  
                (b)
        \end{minipage}
        \vfill
        \begin{minipage}{0.33\linewidth}
                \centering
                \includegraphics[width=0.6\linewidth]{example-image-a}

                (c)
        \end{minipage}
        \vfill
        \caption{Петля сепаратрис (гомоклиническая траектория $\Gamma_0$ ) (a);
                два различных взаимных расположения сепаратрис седла (b),(c).}
        \label{fig:9.5}
\end{figure}
Введем функцию расщепления следующим образом
\begin{equation}
        \label{eq:9.9}
        \rho(\mu) = x_1^u(\mu) - x_2^s(\mu)
\end{equation}
Очевидно, что взаимному расположению сепаратрис, представленному на рис.\ref{fig:9.5}b, соответствует 
$\rho(\mu) > 0 $, а на рис.\ref{fig:9.5}c - $\rho(\mu) < 0.$ Функция $\rho(\mu)$ 
яыляется непрерывной функцией параметра $\mu$, а её нулям отвечают гомоклинические траектории системы
\eqref{eq:8.1}.

Другой важной характеристикой петли сепаратрис седла является величина
\begin{equation}
        \label{eq:9.10}
        \sigma(\mu) = \lambda_1(\mu) + \lambda_2(\mu)
\end{equation}
называемая \textbf{седловой}. Седловая величина также является функцией параметра,
а её смысл будет ясен из дальнейшего изложения.

\subsection{Точечное отображение в окрестности петли сепаратрис седла}%
\label{sec:9.2.1}
С помощью неособого линейного преобразования координат (см. лекцию \ref{lect3})
система \eqref{eq:8.1} может быть приведена к следующему виду
\begin{equation}
        \label{eq:9.11}
        \begin{cases}
                \dot u_1 = \lambda_1(\mu) u_1 + g_1(u_1,u_2,\mu),
                \dot u_2 = \lambda_2(\mu) u_2 + g_2(u_1,u_2,\mu),
        \end{cases}
\end{equation}
где нелинейные функции $g_i(0,0,\mu)=0,~ i =1,2.$. В этой системе координат 
касательные к сепаратрисам седла $O_0(0,0)$ совпадают с осями координат.
Предположим, что система \eqref{eq:9.11} при $\mu=0$ имеет гомоклиническую траекторию
$\Gamma_0$ (см. рис.\ref{fig:9.6}b). Введем в рассмотрение два, трансверсальных к траекториям системы \eqref{eq:9.11}, отрезка

\begin{gather}
        \Sigma_0 = \qty{u_1,u_2 | u_1 = d_1, |u_2| \leq \epsilon}, \\
        \Sigma_1 = \qty{y_1,u_2 | u_2 = d_2, |u_1| \leq \epsilon},
\end{gather}
где $d_1, d_2,\epsilon$ -- достаточно малые положительные величины. 
Введем также функцию расщепления $\rho(\mu)$, используя в качестве секущей отрезок $\Sigma_0$, 
следующим образом
\begin{equation}
        \label{eq:}
        \rho(\mu) = u_2^u(\mu) - u_2^s(\mu),
\end{equation}
где $u_2^u$ и $u_2^s$ -- ординаты точек первого пересечения неустойчивой и устойчивой сепаратрис седла
с секущей $\Sigma_0.$ Тогда бифуркационное условие существования траектории $\Gamma_0$ можно записать
в следующем виде
\begin{equation}
        \label{eq:9.12}
        \rho(0) =0
\end{equation}
Пусть будут выполнены также следующие условия невырожденности
\begin{align}
        \label{eq:9.13}
        & \sigma(0) \neq 0, \\
        \label{eq:9.14}
        & \rho'(0) \neq 0
\end{align}
Смысл условия \eqref{eq:9.13} будет ясен из дальнейшего, а условия \eqref{eq:9.14} означает, что
взаимное расположение сепаратрис будет различным при $\mu<0$ и $\mu>0$.

Построим точечное отображение в окрестности траектории $\Gamma_0$ в виде суперпозиции двух отображений $T=T_l \cdot T_g$, где отображение $T_l$ действует в окрестности седла $O_0$, а $T_g$-- в окрестности глобальной части $\Gamma_0$.

\paragraph{Отображение $T_l.$}%

Для достаточно малых значений $d_1,d_2$ и $\epsilon$ траектории системы \eqref{eq:9.11} в окрестности $O_0$ определяется в основном линейной частью системы \eqref{eq:9.11}, то есть уравнениями
\begin{equation}
        \label{eq:9.15}
        \dot u_1 = \lambda_1(\mu) u_1,\quad \dot u_2 = \lambda_2(\mu) u_2
\end{equation}
и, очевидно, порождают отображение
\begin{equation}
        \label{eq:}
        T_l:\quad \Sigma_0 \to \Sigma_1
\end{equation}
Найдем вид $T_l$. В окрестности $O_0$ систему \eqref{eq:9.11} будем аппроксимировать системой \eqref{eq:9.15}. Пусть при $t=0$ выполнены условия
\begin{equation}
        \label{eq:9.16}
        u_1(0) = d_1, \quad u_2(0) = u_2^0>0,\quad \qty( u_1(0),u_2(0) ) \in \Sigma_0
\end{equation}
Запишем уравнение траектории, удовлетворяющей условию \eqref{eq:9.16}
\begin{equation}
        \label{eq:9.17}
        \begin{cases}
                u_1(t) = d_1 e^{\lambda_1t} \\
                u_2(t) = u_2^0 e^{\lambda_2t}
        \end{cases}
\end{equation}
Обозначим через $\tau$ - время движения между $\Sigma_0$ и $\Sigma_1 $ вдоль траектории 
\eqref{eq:9.17}, т.е.
\begin{equation}
        \label{eq:9.18}
        u_2(\tau) = d_2
\end{equation}
Из \eqref{eq:9.18} находим
\begin{equation}
        \label{eq:9.18}
        \tau = \frac{1}{\lambda_2} \ln \frac{d_2}{u_2^0}
\end{equation}
Пусть $u_1(\tau) = u_2^0.$ Тогда из \eqref{eq:9.17}, \eqref{eq:9.19} имеем
\begin{equation}
        \label{eq:}
        u_1^0 = d_1 e^{\frac{\lambda_1}{\lambda_2} \ln \frac{d_2}{u_2^0}}=
        d_1(d_2)^{\frac{\lambda_1}{\lambda_2}} \qty(u_2^0)^{-\frac{\lambda_1}{\lambda_2}}
\end{equation}
Следовательно, отображение $T_l$ задается следующим образом
\begin{equation}
        \label{eq:9.20}
        u_1^0 = C\qty(u_2^0)^q,
\end{equation}
где $C= d_1(d_2)^{\lambda_1 / \lambda_2} = \const, q = - \frac{\lambda_1}{\lambda_2}$. Заметим, что учет нелинейных слагаемых системы \eqref{eq:9.11} вносит лишь незначительную поправку, которой можно
пренебречь при построении полного отображения.

\paragraph{Отображение $T_g$.}%
Покажем теперь, что траектории системы \eqref{eq:9.1} порождают отображение
\begin{equation}
        \label{eq:}
        T_g: \quad \Sigma_1 \to \Sigma_0
\end{equation}
Действительно, гомоклиническая траектория $\Gamma_0$ при $\mu=0$ соединяет
$\Sigma_1$ и $\Sigma_0$ (см. рис.\ref{fig:9.6}). При этом время движения вдоль $\Gamma_0$ 
от $\Sigma_1$ до $\Sigma_0$ будем конечным.
Отсюда и непрерывной зависимости траекторий системы \eqref{eq:9.11} от начальных
условий вытекает существование отображения $T_g$, которое можно представить в
следующем виде
\begin{equation}
        \label{eq:9.21}
        \bar{ u_2^0} = p(u_1^0, \mu),
\end{equation}
где $\bar {u_2^0}$ -- значение $u_2$ в момент пересечения  $\Sigma_0$ траекторией, выходящей из 
точки $\qty( u_1^0,d_2) \in \Sigma_1$ (см. рис.\ref{fig:9.6}). Отображение $T_g$ является
диффиоморфизмом. Разложим функцию $p(u_1^0, \mu)$ в ряд Тейлора в окрестности точки $(0,0)$ 
\begin{equation}
        \label{eq:9.22}
        \bar u_2^0 = p(0,0) + \pdv{p}{u_1^0} \eval_{(0,0)} u_1^0 + \pdv{p}{\mu} \eval_{(0,0)}\mu
        + \mu + \dots = a u_1^0 + b \mu + dots,
\end{equation}
где всегда $a>0$ (неравенство $a<0$ приводит к пересечению фазовой траекторий),
а слагаемое $b \mu$ описывает функцию расщепления в окрестности $\mu=0.$

Таким образом, траектории системы \eqref{eq:9.11} в окрестности $\Gamma_0$ 
порождают точечное отображение
\begin{equation}
        \label{eq:}
        T: \quad \Sigma_0 \to \Sigma_0
\end{equation}
Используя \eqref{eq:9.20} и \eqref{eq:9.22}, устанавливаем, что $T$ задается    следующим образом

\begin{equation}
        \label{eq:9.23}
        \bar u_2^0 = b \mu + a C\qty(u_2^0)^q + \dots
\end{equation}

\begin{figure}[h]
        \centering
        \begin{minipage}{\linewidth}
                \centering 
                \includegraphics[width=0.6\linewidth]{example-image-a}

                (a)
        \end{minipage}
        \begin{minipage}{\linewidth}
                \centering 
                \includegraphics[width=0.6\linewidth]{example-image-a}

                (b)
        \end{minipage}
        \caption{Отображение Пуанкаре (a)  фазовые портреты (b) системы  \eqref{eq:9.11} в случае
        $\sigma(0)<0.$}
        \label{fig:9.7}
\end{figure}
Функция последования отображения \eqref{eq:9.23} имеет степенной характер, и, как известно, её вид различен
для $q>1$ и $q<1$. Исследуем отображение \eqref{eq:9.23} в этих двух случаях, считая для определенности,
что $b>0.$
 
\paragraph{$\sigma(0) = \lambda_1(0) + \lambda_2(0) <0.$}%

В этом случае $q(0) >1$  вид отображения \eqref{eq:9.23} для различных значений $\mu$ представлен на рис.\ref{fig:9.7}а. Анализируя свойства отображения $T$, устанавливаем соответствующие фазовые портреты, которые представлены на рис.\ref{fig:9.7}b. В области $\mu>0$ на фазовой плоскости $(u_1,u_2)$ существует устойчивый предельный цикл, который появился из траектории $\Gamma_0.$ 
При этом $\rho(\mu) >0 .$ Для $\mu<0$ предельный цикл не существует и $\rho(\mu) <0.$
\begin{figure}[h]
        \centering
        \begin{minipage}{\linewidth}
                \centering 
                \includegraphics[width=0.6\linewidth]{example-image-a}

                (a)
        \end{minipage}
        \begin{minipage}{\linewidth}
                \centering 
                \includegraphics[width=0.6\linewidth]{example-image-a}

                (b)
        \end{minipage}
        \caption{Отображение Пуанкаре (a)  фазовые портреты (b) системы  \eqref{eq:9.11} в случае
        $\sigma(0)>0.$}
        \label{fig:9.8}
\end{figure}

\paragraph{Случай $\sigma(0) = \lambda_1(0) + \lambda_2(0) >0.$}%

Этот случай отвечает значению $q(0) <1$. 
Исследование точечного отображения \eqref{eq:9.23} и соответствующих фазовых 
портретов системы \eqref{eq:9.11} проводится полностью аналогично предыдущему.
Результаты такого анализа представлены на рис.\ref{fig:9.8}. Из траектории $\Gamma_0$
рождается неустойчивый предельный цикл, который существует при $\mu<0$. При
$\mu>0$ система \eqref{eq:9.11} предельных циклов не имеет.

\subsection{Колебательные и вращательные петли сепаратрис}%
\label{sub:9.2.2}

Петли сепаратрис (гомоклинические траектории) могут образовываться не только
сепаратрисами седел на плоскости, но и сепаратрисами седел систем с 
цилиндрическим фазовым пространством. В этом случае, вообще говоря, возможно образование двух типов
петель сепаратрис -- колебательных, которых угловая переменная $\phi$ набегает $2\pi$ и они охватывают цилиндр (см. рис.\ref{fig:9.9}b).
\begin{figure}[h]
        \centering
        \begin{minipage}{0.49\linewidth}
                \centering 
                \includegraphics[width=0.6\linewidth]{example-image-a}

                (a)
        \end{minipage}
        \begin{minipage}{0.49\linewidth}
                \centering 
                \includegraphics[width=0.6\linewidth]{example-image-a}

                (b)
        \end{minipage}
        \caption{Гомоклиническая траектория колебательного типа (a); гомоклиническая траектория 
        вращательного типа (b).}
        \label{fig:9.9}
\end{figure}

Все утверждения раздела \ref{sub:9.2.1} остаются справедливыми для этих типов гомоклинических траекторий. Так, при $\sigma(0)<0$ из петель сепаратрис седла рождаются 
устойчивые предельные циклы, а при $\sigma(0)>0$--неустойчивые.
При этом циклы имею тот же характер (колебательный или вращательный), что и соответствующие петли 
сепаратрис.

