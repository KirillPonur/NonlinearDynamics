%!TEX root = ..\lections.tex

Сверхпроводимость -- это свойство некоторых материалов обладать 
нулевым электрическим сопротивлением, когда их температура достигает 
значания ниже некоторого критического. Такое свойство демонстрируют несколько десятков чистых элементов, керамик и сплавов.

Рассмотрим систему, состоящую из двух сверхпроводников, разделенных 
тонким (толщина порядка $10^{-7}$ см) изолирующим слоем, образованным, например, нормальным металлом. В 1962 г. 22-х летний студент-дипломник 
Б.Джозефсон (Brian Josephson) опубликовал статью, в которой, опираясь на
экспериментальные результаты Гиавера, пришел к выводу о том, что через
такой контакт возможно протекание сверхпроводящих токов. В частности,
протекание постоянного тока даже в случае отсутствия разности потенциалов
между сверхпроводниками. С точки зрения классических представлений такой
эффект невозможен. Джозефсон показал, что протекание сверхпроводящих
токов является следствием туннелирования так называемых куперовских пар
через контакт. Куперовская пара представляет собой квазичастицу,
образованную за счет взаимодействия и связывания двух электронов в
сверхпроводнике. Поток куперовских пар и формирует ток при
сверхпроводимости. Состояние куперовской пары можно описать с помощью
волновой функции. Оказывается, что в сверхпроводнике куперовские пары не
могут двигаться независимо друг от друга. Взаимодействие таких пар приводит
к их взаимной упорядоченности, в следствие которой состояние куперовских
пар в сверхпроводнике характеризуется \textbf{единой волновой функцией}. Другими
словами, куперовские пары скапливаются в одном и том же квантовом
состоянии и поэтому описываются одной и той же волновой функцией.
\section{Стационарный и нестационарный эффекты}%
\label{sec:11.1}

Предположим, что волновые функции куперовских пар в
сверхпроводниках не зависят от пространственных координат и изменяются
только во времени. В таком приближении волновые функции задаются
следующим образом
\begin{equation}
        \label{eq:11.1}
        \Psi_i = \sqrt{\rho_{i}} e^{i\phi_i},~ i =1,2,
\end{equation}
где $\rho_i$ -- плотность зарядов электронов в сверхпроводниках, а $\phi_i$ -- общая для всех частиц фаза в $i$-ом сверхпроводнике. При достаточном сближении сверхпроводников их волновые функции начинают перекрываться в области
изолирующего зазора (барьера), формируя туннельный контакт. Возникает процесс переноса куперовских пар
через барьер, который описывается следующей системой для волновых функций
\begin{equation}
        \label{eq:11.2}
        \begin{cases}
                i \hbar \pdv{\Psi_1}{t} = E_1 \Psi_1 + K \Psi_2,\\
                i \hbar \pdv{\Psi_2}{t} = E_2 \Psi_2+ K \Psi_1,
        \end{cases}
\end{equation}
где $\hbar$-- постоянная Планка, $E_1$, $E_2$ -- энергии основных состояний каждого из проводников, $K$ - амплитуда взаимодействия двух состояний контакта, зависящая от его специфики (геометрии электродов, параметров барьера и др.). Приложим к контакту постоянную разность потенциалов $V$, которая приведет к сдвигу
\begin{equation}
        \label{eq:11.3}
        E_1-E_2=2eV.
\end{equation}
Без ограничения общности будем считать, что энергия отсчитывается от среднего между величинами $E_1$ и $E_2$. Тогда \eqref{eq:11.2} и \eqref{eq:11.3} имеем
\begin{equation}
        \label{eq:11.4}
        \begin{cases}
                i \hbar \pdv{\Psi_1}{t} = eV\Psi_1+ K\Psi_2,\\
                i \hbar \pdv{\Psi_2}{t} = - eV \Psi_2 + K \Psi_1.
        \end{cases}
\end{equation}
Подставляя в \eqref{eq:11.4} выражения $\Psi_1$ и $\Psi_2$ из \eqref{eq:11.1} и разделяя в полученных
уравнениях действительные и мнимые части, находим
\begin{equation}
        \label{eq:11.5}
        \pdv{\rho_1}{t} = \frac{2K \sqrt{\rho_1\rho_2}}{\hbar} \sin \phi, \quad
        \pdv{\rho_2}{t} - \frac{2K \sqrt{\rho_1 \rho_2}}{\hbar} \sin \phi,
\end{equation}
