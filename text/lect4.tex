%!TEX root = ../lections.tex
Предыдущая лекция была посвящена исследованию состояний равновесия двумерных линейных систем. Было показано, что их характер может быть определен из анализа расположения корней характеристического уравнения на комплексной плоскости. Перейдем теперь к исследованию состояний равновесия нелинейных систем.

\section{Метод линеаризации}%
\label{sec:metod_linearizatsii}

Рассмотрим записанную в векторной форме нелинейную систему

\begin{equation}
        \label{eq:4.1}
        \dot{\vec x} = \vec F(\vec x), ~ \vec x \in \R^n, \vec F: \R^n \to \R^n,    
\end{equation}
где $\vec F(\vec x)$ -- гладкая вектор-функция. Предположим, что система \eqref{eq:4.1}  имеет состояние равновесия $\vec x = \vec x^*$. Введем малое возмущение $\vec \xi(t) = \vec x(t) - \vec x^*    $,
для которого из системы \eqref{eq:4.1} имеем
\begin{equation}
        \label{eq:4.2}
        \dot{\vec \xi} = \vec F( \vec x^* + \vec \xi).  
\end{equation}
Раскладывая правую часть системы \eqref{eq:4.2} в ряд Тейлора, получим
\begin{equation}
        \label{eq:4.3}
        \dot{\vec \xi} = \vec A \vec \xi + \dots,
\end{equation}
где $\vec A$ -- $n \times n$ -- матрица Якобы с элементами
\begin{equation}
        \label{eq:}
        a_{ik} = \pdv{F_i}{x_k} \eval_{x=x^*}.
\end{equation}
Отбросим в правой части \eqref{eq:4.3} все нелинейные по $\vec \xi$ слагаемые и рассмотрим систему
\begin{equation}
        \label{eq:4.4}
        \dot{\vec \xi} = \vec A \vec \xi.
\end{equation}
Переход от нелинейной системы \eqref{eq:4.3}  к линейной системе \eqref{eq:4.4} называется её линеаризацией. Мы не будем пока обсуждать соотношение между траекториями систем \eqref{eq:4.3} и \eqref{eq:4.4} , а рассмотрим возможные типы состояний равновесия линейной системы \eqref{eq:4.4} .

На предыдущих лекциях мы уже рассмотрели свойства системы \eqref{eq:4.4}  в одномерном и двумерном случаях. Как было показано, в этих случаях поведение траекторий зависит от корней характеристического уравнения. Аналогичным свойством обладает и система \eqref{eq:4.4} , в случае когда размерность её выше двух . Будем искать решение системы \eqref{eq:4.4} в виде
\begin{equation}
        \label{eq:4.5}
\vec \xi = \vec C e^{\lambda t},
\end{equation}
где $\vec C$ -- постоянная матрица столбец. Подстановка \eqref{eq:4.5} в \eqref{eq:4.4} приводит к системе линейных однородных уравнений, которая имеет нетривиальное решение, если 
\begin{equation}
        \label{eq:4.6}
        \det(\vec A - \lambda \vec E) =0,
\end{equation}
где $\vec  E$ -- единичная матрица. Уравнение \eqref{eq:4.6} эквивалентно алгебраическому уравнению
\begin{equation}
        \label{eq:4.7}
        a_0 \lambda^n + a` \lambda^{n-1} + \dots + a_n =0.
\end{equation}
Уравнение \eqref{eq:4.7} называется характеристическим, а его корни характеристическими показателями состояния равновесия $\vec x =\vec x^*$. 
Справедливы следующие, установленные А.М. Ляпуновым, утверждения:
\begin{itemize}
        \item Если корни уравнения \eqref{eq:4.7} имеют отрицательные вещественные части, т.е. $\Re \lambda_i < 0 ~ (i= 1,2,\dots,n)$, то состояние равновесия системы \eqref{eq:4.4} асимптотически устойчиво.
 \item Если среди корней уравнения \eqref{eq:4.7} есть хотя бы один с положительной вещественной частью, то состояние равновесия системы \eqref{eq:4.4} неустойчиво по Ляпунову.
 \item  Если уравнение \eqref{eq:4.7} не имеет корней с положительной вещественной частью, но имеет некоторое число корней с нулевой вещественной частью, то состояние равновесия системы \eqref{eq:4.4} может быть как устойчивым (но не асимптотически), так и неустойчивым.
\end{itemize}
Таким образом, вопрос об устойчивости состояний равновесия многомерных линейных систем сводится к исследованию характера корней алгебраического уравнения.

Вернемся теперь к исходной нелинейной системы \eqref{eq:4.1},
устойчивость состояний равновесия которой может быть установлена теоремами Ляпунова. Согласно теореме Ляпунова об устойчивости по первому приближению (так называемый первый метод Ляпунова), если корни уравнения \eqref{eq:4.7} удовлетворяют условию $\Re \lambda_i \neq 0~ (i=1,2,\dots,n)$, то характер устойчивости состояний равновесия нелинейной системы \eqref{eq:4.1} и соответствующей линеаризованной системы \eqref{eq:4.7} совпадают. Таким образом, состояние равновесия системы \eqref{eq:4.1} является асимптотически устойчивым, если $\Re \lambda_i<0~ (i=1,2,\dots,n) $ и неустойчивым если среди корней уравнений \eqref{eq:4.7} имеется хотя бы один с положительной вещественной частью.

\section{Критерий Рауса-Гурвица}%
\label{sec:4.2}

Из сказанного выше следует, что решение задачи об устойчивости
состояний равновесия нелинейных систем сводится к анализу расположения
корней характеристического уравнения на комплексной плоскости, т.е. к чисто
алгебраической задаче. Однако, в случае многомерных (размерности три и
выше) систем, как правило, найти характеристические показатели
$\lambda_i$ в явном
виде не удается. Поэтому были развиты критерии и методы, позволяющие
судить об устойчивости состояний равновесия без непосредственного решения
характеристического уравнения. Одним из наиболее известных таких критериев
является критерий Рауса-Гурвица.

Критерии устойчивости Рауса (Routg E.Y.) и Гурвица (Hurwitz A.), вошедших в виде единого критерия, были разработаны в конце 18-го века в связи с проблемами, возникающими в тот момент, в теории автоматического управления. Сформулируем этот критерий для уравнения \eqref{eq:4.7} с вещественными коэффициентами. Без ограничения общности будем считать, что коэффициент $a_0$ является положительным. Составим из коэффициентов $a_j ~ (j=0,1,2,\dots,n)$ квадратную матрицу размерности $n\times n$ в соответствии со следующими правилами:
\begin{itemize}
        \item Первая строка матрицы состоит из коэффициентов с нечетными индексами, начиная с $a_1$.
        \item Элементы каждой последующей строки образуются из соответствующих элементов предшествующей строки уменьшением индексов на единицу.
        \item Если при таком построении индекс $k$ какого-либо коэффициента $a_k$ превосходит значение $n$ или становится отрицательным, то он приравнивается нулю, т.е. $a_k=0$. 
\end{itemize}

В результате описанной процедуры получится $n\times n$ матрица следующего вида
\begin{equation}
        \label{eq:}
        \vec{A_R} = 
        \begin{pmatrix}
                a_1 & a_3 & a_5 & \dots & 0 & 0 \\
                a_0 & a_2 & a_4 & \dots & 0 & 0 \\
                0 & a_1 & a_3 & \dots & 0 & 0 \\
                0 & a_0 & a_2 & \dots & 0 & 0 \\
                \dots & \dots & \dots & \dots & a_{n-1} & 0 \\
                0 & 0 &0 & \dots & a_{n-2} & a_n \\
        \end{pmatrix}
\end{equation}
Заметим, что на главной диагонали матрицы $\vec{A_R}$ стоят последовательно все коэффициенты уравнения \eqref{eq:4.7}, начиная с $a_1$. Далее, выпишем все главные диагональные миноры матрицы $\vec{A_R}$ 
\begin{equation}
        \label{eq:4.8}
        \Delta_1 = a_1, ~ 
        \Delta_2 = 
        \begin{vmatrix}
                a_1 & a_3 \\
                a_0 & a_2
        \end{vmatrix}, ~ 
        \dots, ~
        \Delta_n = a_n \Delta_{n-1}.
\end{equation}

Критерий Рауса-Гурввица состоит в следующем. Для того, чтобы все корни уравнения \eqref{eq:4.7} с вещественными коэффициентами и $a_0>0$ имели отрицательные вещественные части, необходимо и достаточно, чтобы все главные диагональные миноры были положительны
\begin{equation}
        \label{eq:4.9}
        \Delta_n > 0,~\Delta_2> 0, ~ \dots,~\Delta_{n-1}>0,~\Delta_n >0.
\end{equation}

Таким образом, условия \eqref{eq:4.9} гарантируют асимптотическую устойчивость состояния равновесия линейной \eqref{eq:4.4} и нелинейной \eqref{eq:4.1} систем. Однако заметим, что в случае нелинейной системы \eqref{eq:4.1} это лишь локальная устойчивость в малой окрестности состояния равновесия.

В качестве примера применения критерия Рауса-Гурвица рассмотрим уравнение \eqref{eq:4.7} в случае $n=3$. Для удобства перепишем это уравнение в следующем эквивалентном виде
\begin{equation}
        \label{eq:4.10}
        \lambda^3 - a \lambda^2 + b \lambda +c =0
\end{equation}
где
\begin{equation}
        \label{eq:}
        a= \frac{a_1}{a_0}, ~ b= \frac{a_2}{a_0}, ~ c= \frac{a_3}{a_0}.
\end{equation}
Введем, отвечающую уравнению \eqref{eq:4.10}, матрицу
\begin{equation}
        \label{eq:}
        \vec{A_R} = 
        \begin{pmatrix}
                a & c & 0 \\
                1 & b & 0 \\
                0 & a & c \\    
        \end{pmatrix}
        .
\end{equation}
Легко видеть, что главные диагональные миноры этой матрицы имеют вид
\begin{equation}
        \label{eq:4.11}
        \Delta_1 = a, ~ \Delta_2 = ab - c, ~ \Delta_3= c(ab-c).
\end{equation}
Отсюда, согласно критерию Рауса-Гурвица, все корни уравнения \eqref{eq:4.10} имеют отрицательные вещественные части, если параметры этого уравнения удовлетворяют неравенствам
\begin{equation}
        \label{eq:4.12}
        a>0, \quad, ab-c>0, \quad c>0.
\end{equation}

\section{Второй метод Ляпунова}%
\label{sec:4.3}

Рассмотрим еще один метод, позволяющий устанавливать условия
устойчивости состояний равновесия без непосредственного нахождения
характеристических показателей. А.М. Ляпуновым была развита теория, в
основе которой лежит построение специальных функций, позволяющих, в
случае их существования, судить об устойчивости и неустойчивости состояний
равновесия. Эти функции получили название функций Ляпунова, а
базирующаяся на них теория устойчивости -- второго метода Ляпунова.
Изложим кратко основные идеи этого метода для автономных систем.

Рассмотрим скалярную функцию $V(x_1,x_2,\dots,x_n)$ или в векторном виде $V(\vec x)$, определенную в фазовом пространстве системы \eqref{eq:4.1}, непрерывную в некоторой области $D$, содержащей состояние равновесия $\vec x= \vec x^*$. Кроме того, предположим, что $V(\vec x)$ имеет в $D$ непрерывные частные производны. В основе второго метода Ляпунова лежит использование свойств так называемых 
\textbf{знакоопределенных и знакопостоянных} функций.
\begin{enumerate}
        \item Функция $V(\vec x)$ называется знакоопределенной в области $D$, если она обращается в нуль лишь в состоянии равновесия и принимает значения одного знака во всех остальных точках области $D$. Очевидно, что знакоопределенные функции бывают двух типов -- положительно и отрицательно определенные. 
        \item Функция $V(\vec x)$ называется знакопостояннной в области $D$, если она обращается в нуль не только в состоянии равновесия, но и в некоторых других точках области $D$, и имеет значения только одного знака во всех остальных точках области $D$.
\end{enumerate}


