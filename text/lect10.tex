%!TEX root = ..\lections.tex
\section{Петля сепаратрисы седло-узла}%
\label{sec:10.1}

Предположим, что система \eqref{eq:8.1} на фазовой плоскости при $\mu=0$ имеет в начале координат двухкратное равновесие (седло-узел) $O_0(0,0)$, и выполнены соответствующие (см. лекцию \ref{lect8})
бифуркационное условие и условия невырожденности
\begin{equation}
        \label{eq:10.1}
        \lambda_1(0) = 0, \quad \lambda_2(\mu) \neq 0, \quad l(\mu) \neq 0
\end{equation}

Рассмотрим нормальную форму для этой бифуркации
\begin{equation}
        \label{eq:10.2}
        \begin{cases}
                \dot u_1 = \mu + l(\mu) u_1^2 + \dots 
                \dot u_2 = \lambda_2(\mu) u_2 + \dots
        \end{cases}
\end{equation}

Пусть для определенности $\lambda_2(\mu) < 0$ и $l(\mu) > 0$. В этом случае седло-узел
имеет устойчивую узловую область, а одномерная сепаратриса $W^u(O_0)$ 
является неустойчивой. Предположим, что при $\mu=0$ сепаратриса $W^u(O_0)$ при $t \to \infty$ возвращается общим образом в точку $O_0$, образуя гомоклиническую траекторию
$\Gamma(0)$ (см. рис.\ref{fig:10.1}b ) -- петлю сепаратрисы седла-узла.
\begin{figure}[h]
        \centering
        \includegraphics[width=0.6\linewidth]{example-image-a}
        \caption{Бифуркация петли сепаратрисы седло-узла.}
        \label{fig:10.1}
\end{figure}

Заметим, что несмотря на существования траектории $\Gamma(0)$, система \eqref{eq:10.2} имеет единственный негрубый элемент -- двукратное равновесие $O_0.$

При $\mu<0$ состояние равновесия $O_0$ распадается на два грубых -- устойчивый узел и 
седло (см.лекцию \ref{lect8}). При этом гомоклиническая траектория $\Gamma_0$ трансформируется в траекторию
$\Gamma(\mu)$, образованную сепаратрисы седла, идущей в устойчивый узел (см рис.\ref{fig:10.1}а). Существование траектории $\Gamma(\mu)$ вытекает из предположения о том, что $\Gamma(0)$ возвращается в $O_0$ общим  образом, то есть сепаратрисы $W^u(O_0)$ не попадает в <<край>> 
узловой области.

При $\mu>0$ система \eqref{eq:10.2} не имеет состояний равновесия и траектории покидают окрестность начала
координат. Для исследования фазовой плоскости в этом случае построим отображение Пуанкаре. Введем
в рассмотрение два отрезка, расположенные в $\epsilon$-окрестности начала координат
\begin{align}
        S_1 &= \qty{u_1,u_2 \eval u_1 = -d, \quad \abs{u_2} \leq \sqrt{\epsilon^2-d^2} },\\
        S_2 &= \qty{u_1,u_2 \eval u_1= +d, \quad \abs{u_2} \leq \sqrt{\epsilon^2-d^2} },
\end{align}
где $0<d<\epsilon\ll 1.$ Будем строить отображение Пуанкаре $T$ в виде суперпозиции
\begin{equation}
        \label{eq:10.3}
        T = T_l \cdot  T_g 
\end{equation}
где $T_l$ -- локальное отображение, определяемое вдоль траекторий системы \eqref{eq:10.2}, начинающихся
на $S_1$  и заканчивающихся на $S_2$ (см. Рис.\ref{fig:10.2}), а $T_g$ -- глобальное отображение, определяемое 
траекториями, начинающимися на $S_2$ и заканчивающихся на $S_1$ (см. Рис.\ref{fig:10.2}).
\begin{figure}[h]
        \centering
        \includegraphics[width=0.6\linewidth]{example-image-a}
        \caption{Отображение Пуанкаре системы \eqref{eq:10.2} при $\mu>0$.}
        \label{fig:10.2}
\end{figure}

\paragraph{Отображение $T_l$.}%
Из первого уравнения системы \eqref{eq:10.2} следует, что в $\epsilon$-окрестности начала координат выполняется 
неравенство $\dot u_1 > 0$ и, следовательно, $S_1$ и $S_2$ -- секущие Пуанкаре. Рассмотрим 
траекторию системы \eqref{eq:10.2}, проходящую при $t=0$ через произвольную точку на секущей 
$S_1$, то есть траекторию, удовлетворяющую следующим начальным условиям (см. Рис.\ref{fig:10.2})
\begin{equation}
        \label{eq:10.4}
        u_1(0) = - d, \quad u_2(0) - u_2^0
\end{equation}
Поскольку между секущими $S_1$ и $S_2$ выполняется неравенство
$\dot u_1>0$ и $\lambda_2<0$, эта траектория через конечное время $\tau$ попадает на $S_2$ 
(см. Рис.\ref{fig:10.2}) находим
\begin{equation}
        \label{eq:10.6}
        \tilde u_2 = u_2^0 e^{\lambda_2 \tau} + \dots
\end{equation}
Проинтегрировав первое уравнение системы \eqref{eq:10.2}, находим время $\tau$ 
\begin{equation}
        \label{eq:10.7}
        \tau = \frac{1}{l} \int\limits_{-d}^{d} \frac{\dd{u_1}}{u_1^2 + \frac{\mu}{l}} + \dots
        = \frac{2}{\sqrt{\mu l}} \arctg \frac{d\sqrt l}{\sqrt \mu} \dots .
\end{equation}
Следовательно, отображение $T_l$ задается системой \eqref{eq:10.6}, \eqref{eq:10.7}.
Заметим, что в силу \eqref{eq:10.7} при $\mu \to 0$ время $\tau \to \infty$. Это означает, что при достаточно
малом значении $\mu>0$ изображающая точка будет очень долго двигаться в окрестности начала координат.

\paragraph{Отображения $T_g$.}%
Покажем сначала существование отображения $T_g$. Установим существование траектории
$\Gamma_{1}(\mu)$ системы \eqref{eq:10.2}, которая за конечное время переводит точку 
$(u_{1}=d, u_{2} = \tilde u_{2}) \in S_{2}$ в некоторую точку 
$\qty(u_{1}=-d,u_{2}= \bar u_{2}^{0}) \in S_{1}$. Существование траектории $\Gamma_{1}(0)$ 
при $\mu = 0$ следует из наличия по предположению гомолитической траектории $\Gamma_{1}(0)$ 
при $\mu=0$ следует из наличия по предположению гомолитической траектории $\Gamma(0)$ и теоремы о 
непрерывной зависимости решений системы дифференциальных уравнений от параметра вытекает 
существование $\Gamma_{1}(\mu)$ при $\mu>0$. Следовательно, отображение $T_g$ существует и может быть представлено
в следующем виде
\begin{equation}
        \label{eq:10.8}
        \bar{u}_{2}^{0} = g( \tilde u_{2},\mu)
\end{equation}
В силу конечности времени движения вдоль $\Gamma_{1}(\mu)$, отображение \eqref{eq:10.8} --
гомеоморфизм. Раскладывая функцию $g(\tilde{u_{2}}, \mu)$ в окрестности точки $(0,0)$ 
получим  
\begin{equation}
        \label{eq:10.9}
        g(\tilde u_{2}, \mu) = g(0,0) + \pdv{g}{\tilde u_{2}} \eval_{(0,0)}
        \cdot \tilde u_{2} + \pdv{g}{\mu} \eval_{(0,0)} \cdot \mu + \dots =
        a+ b \tilde u_{2} + c \mu + \dots,
\end{equation}
где $g>0$, а знаки параметров $a$ и $c$ не имеют значения. Пусть для определенности
$a>0$ и $c>0$.

Из \eqref{eq:10.9}, \eqref{eq:10.6} и \eqref{eq:10.7} вытекает, что отображение $T$ задается следующим образом
\begin{equation}
        \label{eq:10.10}
        \bar u_{2}^{0} = a + b e^{\lambda_{2}\tau} u_{2}^{0} + c\mu+ \dots.
\end{equation}

Нетрудно видеть, что отображение $T$ имеет нетривиальную неподвижную точку. Поскольку 
$\lambda_{2} < 0$, а $\tau\gg 1,$ мультипликатор этой неподвижной точки меньше единицы
и оно устойчива. Следовательно, при $\mu > 0 $ система \eqref{eq:10.2} имеет устойчивый 
предельный цикл (см. рис.\ref{fig:10.1}c), который появился на фазовой плоскости из траектории
$\Gamma(0)$.

Совершенно аналогично проводится исследование системы \eqref{eq:10.2} в случае 
$\lambda_{2}(\mu) >0$. В этом случае разрушение гомоклинической орбиты седло-узла
приводит к рождению неустойчивого предельного цикла.

Проиллюстрируем бифуркацию рождения предельного цикла из петли
сепаратрисы седло-узла следующим примером. Рассмотрим систему в
полярных координатах

\begin{equation}
        \label{eq:10.11}
        \begin{cases}
                \dot \rho = \rho(1-\rho),\\
                \dot \phi = \sin \phi - \mu,
        \end{cases}
\end{equation}
где параметр $\mu>0$. В системе \eqref{eq:10.11} уравнения независимы и легко
анализируются отдельно. Уравнение для $\rho$ -- уравнение на прямой, имеющее
два состояния равновесия: неустойчивые $\rho=0$ и устойчивое $\rho=1$. 
Уравнение для $\phi$ -- уравнение  на окружности (см. лекцию \ref{lect2}), которое
при $\mu>1$ не имеет состояний равновесия, а при $\mu<1$ имеет два состояния равновесия:
неустойчивое $\phi=\phi_{1}=\arcsin \mu$ и устойчивое $\phi = \phi_{2} = \pi - \arcsin \mu$.
При $\mu =1$ эти состояния равновесия сливаются, образуя полуустойчивое состояние равновесия $\phi= \frac{\pi}{2}$. Объединяя эти свойства уравнений системы \eqref{eq:10.11}, устанавливаем фазовые портреты системы
\eqref{eq:10.11} представленные на рис.\ref{fig:10.3}. При $\mu>1$ система \eqref{eq:10.11} имеет устойчивый предельный цикл,
родившийся из гомоклинической траектории $\Gamma$.

\begin{figure}[h]
        \centering
        \includegraphics[width=0.6\linewidth]{example-image-a}
        \caption{Бифуркация петли сепаратрисы седло-узла системы \eqref{eq:10.11}}
        \label{fig:10.3}
\end{figure}

\section{Заключительные замечания о бифуркациях систем на плоскости}%
\label{sec:10.2}

Таким образом, мы рассмотрели все особые траектории (состояния
равновесия, предельные циклы, сепаратрисы седел), определяющие структуру
разбиения фазовой плоскости на траектории и основные (коразмерности 1)
бифуркации, вызывающие перестройку этого разбиения. Представим сведения
об основных бифуркациях систем на плоскости в виде следующей таблицы, в
которой для краткости приведён один из возможных вариантов каждой
бифуркации. В таблице локальные бифуркации помечены цифрой I, а
нелокальные – цифрой II.

\begin{table}[h]
        \centering
        \caption{Основные бифуркации систем на плоскости.}
        \label{tab:10.1}
        \begin{tabular}{c}
            
        \end{tabular}
\end{table}

Возникает естественный вопрос – как для конкретной нелинейной
системы установить структуру разбиения фазовой плоскости на траектории и
как выяснить какие именно бифуркации в ней происходят? Ответа на этот
вопрос для произвольной нелинейной системы не существует и исследование
каждой системы требует, вообще говоря, индивидуального подхода. Другими
словами, не существует единого универсального метода исследования
динамики произвольной нелинейной системы. Однако, существуют некоторые
классы систем, для которых развиты методы и приемы исследования,
позволяющие регулярным образом получить ответ на поставленный вопрос.
Знакомство с методами исследования нелинейных систем мы начнем с
изучения так называемых релаксационных колебаний.


\section{Динамика быстро-медленной системы}%
\label{sec:10.3}

Во многих системах, самой различной природы, параметры, строго
говоря, не являются постоянными величинами, а медленно эволюционируют с
течением времени. Естественно возникает вопрос – могут или нет такие
медленные изменения параметров привести к принципиальному изменению
состояния системы? Если да, то при каких условиях это произойдет?
Медленные изменения параметра можно описать, например, с помощью
дифференциального уравнения, где в роли переменной выступает этот
параметр, а в правой части в виде сомножителя присутствует постоянный
малый параметр (см. лекцию \ref{lect8}, раздел \ref{sec:8.4}). В результате мы получим систему, у
которой часть переменных будет изменяться во времени значительно
медленнее других переменных, не содержащих в правой части малых
постоянных сомножителей. Для таких систем характерно присутствие двух
масштабов времени и двух скоростей и ассоциирующихся с ними так
называемых быстрых и медленных движений. Наличие таких видов движения
может привести к режиму, когда после кратковременного быстрого изменения
части переменных в системе наступает квазиравновесное (по отношению к
быстрым движениям) состояние, соответствующее медленным движением.
Такой процесс быстрого установления квазиравновесного состояния называется
релаксацией (термин введен Ван-дер-Полем). Если процессы релаксации
повторяются, сменяя квазиравновесный режим, то в системе могут возникнуть
колебания, которые называются релаксационными.

Рассмотрим релаксационные колебания в системах второго порядка вида
\begin{equation}
        \label{eq:10.2}
        \begin{cases}
                \dot x = P(x,y) ,\\ 
                \mu \dot y = Q(x,y),
        \end{cases}
\end{equation}
где $P(x,y)$ и $Q(x,y)$ -- однозначные непрерывные функции, имеющие
непрерывные частные производные, а $\mu$ -- малый положительный параметр. В
\eqref{eq:10.12} $x$ -- медленная, а $y$ -- быстрая переменные\footnote{В этом легко убедиться, введя
        в \eqref{eq:10.12} новое время $\mu \tau = t$. В результате система
        \eqref{eq:10.12} преобразуется к виде
        \begin{equation}
                \label{eq:}
                \dv{x}{\tau} = \mu P(x,y);\quad \dv{y}{\tau} = Q(x,y)
        \end{equation}
        отсюда при $\mu={0}$ имеем $x= \const$ и, очевидно, что при достаточно малом $\mu>0$ переменная $x$ действительно изменяется медленно по сравнению с переменной $y$. 
}

\subsection{Медленные и быстрые движения}%
\label{sub:10.3.1}

Предположим, что переменная $y$ изменяется так, что $\dot y$ является
ограниченной функцией. В этом случае при $\mu = 0$ изменение медленной
переменной $x$ будет описываться системой
\begin{equation}
        \label{eq:10.13}
        \dot x = P(x,y), \quad Q(x,y) = 0
\end{equation}
Система \eqref{eq:10.13} называется системой медленных движений. Установим условия,
при которых при достаточно малом $\mu$ движения системы \eqref{eq:10.12}
можно аппроксимировать системой \eqref{eq:10.13}, т.е. когда малый параметр
$\mu$ можно не учитывать в первом приближении. Фактически нужно найти условия
ограниченности $\dot y$. Из второго уравнения системы \eqref{eq:10.12} имеем
\begin{equation}
        \label{eq:10.14}
        \dot y = \frac{Q(x,y)}{\mu}
\end{equation}
Пусть $(\tilde x, \tilde y)$ -- точка на линии $Q(x,y) = 0,$ т.е. 
$Q(\tilde x, \tilde y)=0$, а $(x,y)$ -- точка на фазовой плоскости, лежащая вне этой линии.
Введем разности
\begin{equation}
        \label{eq:}
        \xi = x - \tilde x \text{ и } \eta = y - \tilde y
\end{equation}
Раскладывая функцию $Q(x,y)$ в ряд Тейлора по степеням $\xi$ и $\eta$, из \eqref{eq:10.14} получим
\begin{equation}
        \label{eq:10.15}
        \dot y = \frac{ Q_{x}'(\tilde x,\tilde y)\xi + Q'_y (\tilde x, \tilde y) \eta + \dots}{\mu}
\end{equation}
Из \eqref{eq:10.15} следует, что лишь внутри малой (порядка $\mu$ ) окрестности линий
$Q(x,y) = 0$ при $\mu \to 0$ величина $\dot y$ будет ограниченной функцией. 
Следовательно, только в этой малой окрестности линии $Q(x,y) = 0$ для описания 
движений системы \eqref{eq:10.12} можно использовать систему \eqref{eq:10.13}.

Вне малой окрестности линии $Q(x,y) = 0$ величина $\dot y \to \infty$, а $\dot x$ 
остается ограниченной и при $\mu\to 0$ 
\begin{equation}
        \label{eq:10.16}
        \dv{x}{y} = \frac{\mu P(x,y)}{Q(x,y)} \to 0
\end{equation}

Из \eqref{eq:10.16} следует, что в данной области фазовой плоскости фазовые траектории 
системы \eqref{eq:10.12} близки к прямым $x=x^{0}=\const$. Вдоль этих траекторий изображающая
точка двигается с большими скоростями изменение переменной $y$. Такие движения называются быстрыми.
Приближенно быстрые движения можно описать с помощью системы
\begin{equation}
        \label{eq:10.17}
        \mu \dot y = Q\qty(x^{0}, y),\quad x^0 = \const,
\end{equation}
которая называется системой быстрых движений. Заметим, что на фазовой плоскости 
$(x,y)$ состояния равновесия системы \eqref{eq:10.17} расположены в точках пересечения линии
$Q(x,y) = 0$ с прямыми $x= x^{0}=\const$.

Таким образом, исследование системы второго порядка \eqref{eq:10.12} сводится к изучению
двух систем первого порядка: системы медленных движений \eqref{eq:10.13}
в малой окрестности линии $Q(x,y) = 0$ и системы быстрых движений \eqref{eq:10.17} вне этой окрестности.

\subsection{Системы с однократной релаксацией}%
\label{sub:10.3.2}

Рассмотрим систему \eqref{eq:10.17}, которую перепишем в виде
\begin{equation}
        \label{eq:10.18}
        \pdv{y}{\tau} = Q(x^{0},y),\quad \text{ где } \tau = \frac{t}{\mu}.
\end{equation}
Предположим что при каждом значении $x^0$ уравнение $Q(x^{0},y)=0$ 
имеет единственное решение $y=y^{0}$, т.е. для любого $x^0$ уравнение \eqref{eq:10.18} имеет единственное состояние равновесия. Как известно (см. лекцию \ref{lect2}), состояние равновесия
$y=y_{0}$ будет устойчивым, если
\begin{equation}
        \label{eq:10.19}
        Q'_y\qty(x^0,y^0) <0
\end{equation}
 Пусть это условие выполняется при любом $x_{0}$. В этом случае любая траектория
системы \eqref{eq:10.18}, начинающаяся вне линии $Q(x,y)=0,$ имеет форму прямой
$x=x^{0}$, вдоль которой изображающая точка быстро приближается к состоянию
равновесия $y=y^0$, расположенному на $Q(x,y) = 0.$ (см. рис.\ref{fig:10.4}a).

Как было установлено выше, при достаточно малом $\mu>0$ траектории системы
\eqref{eq:10.12} близки к прямым $x=x^{0}$ только вне слоя медленных движений 
(см. рис.\ref{fig:10.4}b), а для описания движений внутри слоя нужно использовать систему
\eqref{eq:10.13}. При выполнении \eqref{eq:10.19} тонкий слой медленных движений притягивает все быстрые
движения, т.е. медленные движения устойчивы по отношению к быстрым. Фактически, в этом случае все установившиеся режимы в системе \eqref{eq:10.12} могут быть исследованы лишь с использованием одной системы
медленных движений \eqref{eq:10.13}, т.е. без учета малого параметра $\mu$.
\begin{figure}[h]
        \centering
        \includegraphics[width=0.6\linewidth]{example-image-a}
        \caption{Фазовые портреты в случае $Q'_{y}<0$ системы \eqref{eq:10.18} (a) и системы \eqref{eq:10.12} (b)}
        \label{fig:10.4}
\end{figure}
Если для любого состояния равновесия системы \eqref{eq:10.17} выполняется неравенство
\begin{equation}
        \label{eq:10.20}
        Q'_y \qty(x^{0}, y^{0}) >0,
\end{equation}
медленные движения будут неустойчивы по отношению к быстрым движениям. В этом случае тонкий слой
медленных движений отталкивает траектории быстрых движений и внутри слоя динамика вновь 
определяется системой \eqref{eq:10.13}. Следовательно, здесь, как и в предыдущем случае, параметр $\mu$ можно не учитывать.

Проиллюстрируем изложенную теорию примером. Рассмотрим уравнение
\begin{equation}
        \label{eq:10.21}
        \ddot \phi = \lambda \dot \phi + \sin \phi = \gamma,
\end{equation}
описывающее динамику математического маятника в вязкой  среде, находящегося под действием внешнего вращательного момента (см. лекцию \ref{lect8}). Сделав в \eqref{eq:10.21} замену времени $\tau \lambda = t$, получим уравнение
\begin{equation}
        \label{eq:10.22}
        \frac{1}{\lambda^2} \dv[2]{\phi}{\tau} + \dv{\phi}{\tau} + \sin \phi =\gamma
\end{equation}
Предположим, что среда обладает сильной вязкостью $\lambda \gg 1 $. В этом случае можно ввести малый параметр $\mu = 1 / \lambda^2 \ll 1$ и представить \eqref{eq:10.22} в виде системы
\begin{equation}
        \label{eq:10.23}
        \begin{cases}
                \dv[]{\phi}{\tau} = y\\
               \mu \dv[]{y}{\tau} = \gamma - \sin \phi - y  
        \end{cases}
\end{equation}
Система \eqref{eq:10.23} имеет цилиндрическое фазовое пространство $G = S^1 \times \R$.
Проведем исследование системы \eqref{eq:10.23} в случае достаточно сильного 
вращательного момента $\gamma>1$.

Система медленных движений имеет вид
\begin{equation}
        \label{eq:10.24}
        \dot \phi = y, \gamma - \sin \phi - y = 0
\end{equation}
На линии медленных движений $y=\gamma-\sin \phi$ определено уравнение на окружности
\begin{equation}
        \label{eq:10.25}
        \dot \phi = \gamma - \sin \phi
\end{equation}
При $\gamma>1$ выполняется неравенство $\dot \phi>0$ и любая траектория уравнения \eqref{eq:10.25} совершает вращательные движения.

Быстрые движения системы \eqref{eq:10.23} задаются уравнениями
\begin{equation}
        \label{eq:10.26}
        \mu \dot y = \gamma - \sin \phi^0 -y, \quad \phi=\phi^{0} = \const
\end{equation}
Из \eqref{eq:10.26} следует, что $Q'_y=-1<0$ и, следовательно, кривая медленных движений устойчива по 
отношению к быстрым.

Принимая во внимание установленные свойства быстрых и медленных движений, получаем фазовые портреты системы
\eqref{eq:10.23} представленные на рис.\ref{fig:10.5}. На фазовом цилиндре $G$ системы \eqref{eq:10.23} существует единственный устойчивый предельный цикл вращательного типа. Следовательно,
под действием внешнего вращательного момента маятник совершает периодическое вращение вокруг точки подвеса.

\begin{figure}[h]
        \centering
        \includegraphics[width=0.6\linewidth]{example-image-a}
        \caption{Фазовые портреты в случае $\gamma>1$ : системы \eqref{eq:10.26} (a) и системы \eqref{eq:10.23} (b).}
        \label{fig:10.5}
\end{figure}

\subsection{Релаксационные колебания}%
\label{sub:10.3.3}

Предположим, что уравнение \eqref{eq:10.17} для некоторых значений $x^0$ имеет одновременно
не одно, а несколько состояний равновесия.  При этом для части из них выполнено неравенство \eqref{eq:10.19}, а для остальных -- \eqref{eq:10.20}, т.е. одни состояния равновесия являются
устойчивыми, а другие -- неустойчивыми. В этом случае линия $Q(x,y) = 0$ распадается на некоторое
число устойчивых и неусточивых по отношению к быстрым движениям компонент. Для примера на рис.\ref{fig:10.6}а представлен случай, когда существуют две устойчивые компоненты -- $Q^{+}_{1}$ 
и $Q_{2}^{+}$ и одна неустойчивая -- $Q^{-}$. Понятно, что устойчивые и неустойчивые компоненты 
разделяются точками, в которых
\begin{equation}
        \label{eq:10.27}
        Q'_y\qty(x^{0}, y^{0}) = 0.
\end{equation}

Например, на рис.\ref{fig:10.6}а таких точек две -- $A$ и $B$. Пусть
$(x^0 = x^*, ~ y^0 = y^*)$ -- координаты одной из точек, в которой выполняется
\eqref{eq:10.27}. При $x^0 = x^*$ в уравнении \eqref{eq:10.17}, описывающем быстрые движения,
происходит бифуркация состояний равновесия. Рассмотрим наиболее типичный случай --
пусть в точке $x^0 = x^*$ происходит основная (коразмерности 1) бифуркация для
состояний равновесия -- двукратное равновесия. С геометрической точки зрения
это означает, что на фазовой плоскости к точке $(x^*, y^*)$ примыкают лишь
две компоненты кривой $Q(x,y) = 0$ (на рис.\ref{fig:10.6}а в точке $A$-- $Q_{1}^+$ и
$Q_{2}^+$, а в точке $B$-- $Q_{2}^+$ и $Q_{2}^-$).
\begin{figure}[h]
        \centering
        \includegraphics[width=0.6\linewidth]{example-image-a}
        \caption{Фазовые портреты: систем \eqref{eq:10.13}, \eqref{eq:10.17} (а) и 
        системы \eqref{eq:10.12} (b).}
        \label{fig:10.6}
\end{figure}
Разложим функцию $Q(x^0,y)$ в ряд Тейлора в окрестности точки $(x^*,y^*)$ 
\begin{equation}
        \label{eq:10.28}
\begin{align}
        Q(x^0,y) = Q(x^*,y^*) + Q'_{x}(x^*,y^*)\qty(x^0 - x^*) + Q'_y(x^*,y^*)(y-y^*)+& \\
        + \frac{1}{2} Q''_{yy}(x^*,y^*)(y-y^*)^2 &+ \dots                                                                 
\end{align}
\end{equation}
Принимая во внимание \eqref{eq:10.28} и равенство $Q(x^*,y^*)=0$, получаем, что при
$x^0 =x^*$ уравнение \eqref{eq:10.17} может быть записано в следующем виде
\begin{equation}
        \label{eq:10.29}
        \mu \dot y = a(y-y^*)^2 + \dots,
\end{equation}
где $a= \frac{1}{2} Q''_{yy}(x^*,y^*)\neq 0$, поскольку в точке $x=x^*$ имеет место 
бифуркация двукратного равновесия. Из \eqref{eq:10.27} следует, что в окрестности $y=y^*$ 
при $\mu \to 0$ $|\dot y| \to \infty$. Следовательно, точка $(x^*,y^*)$ является точкой стыковки медленных и быстрых фазовых траекторий.
В точках такого вида происходит <<срыв>> движения с одной из устойчивых компонент 
медленного движения (на рис.\ref{fig:10.6}а с $Q_{1}^+$ в точке $A$ и с $Q_{2}^+$ в точке $B$ )
и релаксация к другой устойчивой компоненте. Далее процесс может повториться и в результате, в 
системе могут возникнуть периодические релаксационные колебания. Точнее, если
в результате таких релаксационных переходов в вырожденном случае будет 
образована замкнутая фазовая кривая $L_{0}$ (см. рис.\ref{fig:10.6}а), то
существует такое число $\mu_{0}>0$, что при каждом значении параметра $\mu \in (0,\mu_{0})$ 
найдется малая окрестность траектории $L_{0}$, в которой лежит единственный цикл
$L_{\mu}$ системы \eqref{eq:10.12} (см. рис.\ref{fig:10.6}b). 
При $\mu \to 0 $ цикл $L_{\mu}$ стремится к $L_{0}$.

В качестве примера рассмотрим систему ФитцХью-Нагумо, которая
описывает электрическую активность нервной клетки – нейрона. Тело клетки
нейрона окружено биологической мембраной. Одна из важнейших функций
биологической мембраны – генерация и передача биопотенциалов.
Мембранный биопотенциал возникает из-за градиента концентрации ионов по
разную сторону мембраны и возникающего в следствии этого переноса ионов
через ионные каналы мембраны. Нейроны могут генерировать биопотенциалы
в виде как одиночных электрических импульсов возбуждения (спайков), так
серий импульсов. С физической точки зрения биомембрану можно
рассматривать как электрический конденсатор, в котором пластинами является
электролиты наружного и внутреннего растворов. Через биомембрану по
ионным каналам протекают ионные токи – калиевый, натриевый и др. Каждый
ионный ток определяется разностью мембранного потенциала и равновесного
потенциала, создаваемого диффузией ионных токов. Простейшей системой,
описывающей эти процессы является модель ФитцХью-Нагумо, имеющая вид
\begin{equation}
        \label{eq:10.30}
        \begin{cases}
                \dot u = f(u) - v,\\
                \dot v = \mu(u-I),
        \end{cases}
\end{equation}
где $u$ описывает динамику мембранного потенциала, $v$ -- совокупное действие
всех ионных токов, нелинейная функция имеет вид $f(u) = u(1-u)(u-a),~0<a<1$,
параметр $I$ контролирует уровень деполяризации мембраны, а параметр $\mu$ 
$(0<\mu \ll 1)$ определяет характерные временные масштабы импульсов возбуждения.
Следовательно, в \eqref{eq:10.28} $v$ -- медленная, а $u$ -- быстрая переменные.

Пусть параметр $I \in \qty(u_{min},u_{max})$, где $u_{min}$ и $u_{max}$ -- координаты 
минимума и максимума функции $f(u)$ соответственно. Запишем системы медленных
и быстрых движений
\begin{equation}
        \label{eq:10.31}
        \dot v = u - I, \quad v = f(u)\\
\end{equation}
\begin{equation}
        \label{eq:10.32}
        \mu \dot u = f(u) - v^0, \quad v=v^0=\const
\end{equation}

Анализируя \eqref{eq:10.31}, \eqref{eq:10.32}, устанавливаем представленные на рис.\ref{fig:10.7}
фазовые портреты системы \eqref{eq:10.30}. На фазовой плоскости существует устойчивый
предельный цикл $L_{\mu}$, соответствующий периодическому изменению мембранного потенциала.
Рассмотрим изменение переменной $u(t)$, отвечающей циклу $L_{\mu}$. Пусть в начальный момент времени
изображающая точка находилась на цикле и значение $u$ было максимальным -- точка $A$ 
(см. рис.\ref{fig:10.7}b). В начале точка двигается по циклу медленно пока не достигнет точки
$B$.
\begin{figure}[h]
        \centering
        \includegraphics[width=0.6\linewidth]{example-image-a}
        \caption{Фазовые портреты: систем \eqref{eq:10.31}, \eqref{eq:10.32} (а) и системы \eqref{eq:10.30}(b)}
        \label{fig:10.7}
\end{figure}
После этого следует очень быстрое прохождение участка $BC$ и снова медленное
движение вдоль участка $CD$, которое вновь сменяется быстрым движением на
участке $DA$. В результате в системе реализуются периодические колебания
потенциала $u$ , качественный вид которых представлен на рис. \ref{fig:10.8}.
\begin{figure}[h]
        \centering
        \includegraphics[width=0.6\linewidth]{example-image-a}
        \caption{Периодические релаксационные колебания мембранного потенциала в модели 
        Фитц-Хью-Нагумо.}
        \label{fig:10.8}
\end{figure}

