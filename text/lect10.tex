%!TEX root = ..\lections.tex
\section{Петля сепаратрисы седло-узла}%
\label{sec:10.1}

Предположим, что система \eqref{eq:8.1} на фазовой плоскости при $\mu=0$ имеет в начале координат двухкратное равновесие (седло-узел) $O_0(0,0)$, и выполнены соответствующие (см. лекцию \ref{lect8})
бифуркационное условие и условия невырожденности
\begin{equation}
        \label{eq:10.1}
        \lambda_1(0) = 0, \quad \lambda_2(\mu) \neq 0, \quad l(\mu) \neq 0
\end{equation}

Рассмотрим нормальную форму для этой бифуркации
\begin{equation}
        \label{eq:10.2}
        \begin{cases}
                \dot u_1 = \mu + l(\mu) u_1^2 + \dots 
                \dot u_2 = \lambda_2(\mu) u_2 + \dots
        \end{cases}
\end{equation}

Пусть для определенности $\lambda_2(\mu) < 0$ и $l(\mu) > 0$. В этом случае седло-узел
имеет устойчивую узловую область, а одномерная сепаратриса $W^u(O_0)$ 
является неустойчивой. Предположим, что при $\mu=0$ сепаратриса $W^u(O_0)$ при $t \to \infty$ возвращается общим образом в точку $O_0$, образуя гомоклиническую траекторию
$\Gamma(0)$ (см. рис.\ref{fig:10.1}b ) -- петлю сепаратрисы седла-узла.
\begin{figure}[h]
        \centering
        \includegraphics[width=0.6\linewidth]{example-image-a}
        \caption{Бифуркация петли сепаратрисы седло-узла.}
        \label{fig:10.1}
\end{figure}

Заметим, что несмотря на существования траектории $\Gamma(0)$, система \eqref{eq:10.2} имеет единственный негрубый элемент -- двукратное равновесие $O_0.$

При $\mu<0$ состояние равновесия $O_0$ распадается на два грубых -- устойчивый узел и 
седло (см.лекцию \ref{lect8}). При этом гомоклиническая траектория $\Gamma_0$ трансформируется в траекторию
$\Gamma(\mu)$, образованную сепаратрисой седла, идущей в устойчивый узел (см рис.\ref{fig:10.1}а). Существование траектории $\Gamma(\mu)$ вытекает из предположения о том, что $\Gamma(0)$ возвращается в $O_0$ общим  образом, то есть сепаратриса $W^u(O_0)$ не попадает в <<край>> 
узловой области.

При $\mu>0$ система \eqref{eq:10.2} не имеет состояний равновесия и траектории покидают окрестность начала
координат. Для исследования фазовой плоскости в этом случае построим отображение Пуанкаре. Введем
в рассмотрение два отрезка, расположенные в $\epsilon$-окрестности начала координат
\begin{align}
        S_1 &= \qty{u_1,u_2 \eval u_1 = -d, \quad \abs{u_2} \leq \sqrt{\epsilon^2-d^2} },\\
        S_2 &= \qty{u_1,u_2 \eval u_1= +d, \quad \abs{u_2} \leq \sqrt{\epsilon^2-d^2} },
\end{align}
где $0<d<\epsilon\ll 1.$ Будем строить отображение Пуанкаре $T$ в виде суперпозиции
\begin{equation}
        \label{eq:10.3}
        T = T_l \cdot  T_g 
\end{equation}
где $T_l$ -- локальное отображение, определяемое вдоль траекторий системы \eqref{eq:10.2}, начинающихся
на $S_1$  и заканчивающихся на $S_2$ (см. рис.\ref{fig:10.2}), а $T_g$ -- глобальное отображение, определяемое 
траекториями, начинающимися на $S_2$ и заканчивающихся на $S_1$ (см. рис.\ref{fig:10.2}).
\begin{figure}[h]
        \centering
        \includegraphics[width=0.6\linewidth]{example-image-a}
        \caption{Отображение Пуанкаре системы \eqref{eq:10.2} при $\mu>0$.}
        \label{fig:10.2}
\end{figure}

\paragraph{Отображение $T_l$.}%
Из первого уравнения системы \eqref{eq:10.2} следует, что в $\epsilon$-окрестности начала координат выполняется 
неравенство $\dot u_1 > 0$ и, следовательно, $S_1$ и $S_2$ -- секущие Пуанкаре. Рассмотрим 
траекторию системы \eqref{eq:10.2}, проходящую при $t=0$ через произвольную точку на секущей 
$S_1$, то есть траекторию, удовлетворяющую следующим начальным условиям (см. рис.\ref{fig:10.2})
\begin{equation}
        \label{eq:10.4}
        u_1(0) = - d, \quad u_2(0) - u_2^0
\end{equation}
Поскольку между секущими $S_1$ и $S_2$ выполняется неравенство
$\dot u_1>0$ и $\lambda_2<0$, эта траектория через конечное время $\tau$ попадает на $S_2$ 
(см. рис.\ref{fig:10.2}) находим
\begin{equation}
        \label{eq:10.6}
        \tilde u_2 = u_2^0 e^{\lambda_2 \tau} + \dots
\end{equation}
Проинтегрировав первое уравнение системы \eqref{eq:10.2}, находим время $\tau$ 
\begin{equation}
        \label{eq:10.7}
        \tau = \frac{1}{l} \int\limits_{-d}^{d} \frac{\dd{u_1}}{u_1^2 + \frac{\mu}{l}} + \dots
        = \frac{2}{\sqrt{\mu l}} \arctg \frac{d\sqrt l}{\sqrt \mu} \dots .
\end{equation}
Следовательно, отображение $T_l$ задается системой \eqref{eq:10.6}, \eqref{eq:10.7}.
Заметим, что в силу 


