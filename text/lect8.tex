%!TEX root = ..\lections.tex
\section{Бифуркационные условия}%
\label{sec:8.1}

Напомним, что параметры, при которых система является негрубой,
называются бифуркационными. Для того, чтобы задать те или иные
бифуркационные условия нужно нарушить условия грубости (структурной
устойчивости) динамической системы. На предыдущих лекциях мы
установили, что состояние равновесия систем с двумерным фазовым
пространством являются грубыми, если $\Re \lambda_i \neq 0,~ i=1,2$, где $\lambda_i$--
характеристические показатели состояния равновесия, а условие грубости
предельного цикла имеет вид $s\neq 1 ~ (\lambda \neq 0)$, где $s$ – мультипликатор
(характеристический показатель) предельного цикла. Поэтому бифуркации
состояний равновесия на плоскости происходят, когда, по крайней мере, один
из характеристических показателей обращается в нуль, или когда
характеристические показатели становятся мнимыми. Предельные циклы
совершают бифуркации при значениях параметров, при которых их
мультипликатор становится равным единице. Следовательно, для того чтобы
описать ту или иную бифуркацию, необходимо задать некоторое число $\vec k$
условий типа равенства на параметры (условие вырожденности). Ясно, что
степень вырожденности (степень негрубости) системы может быть различной.
Например, бифуркация состояния равновесия может происходить как в случае
обращения в нуль только одного характеристического показателя, так и в
случае, когда оба показателя одновременно становятся равными нулю.
Поэтому, чтобы разделить бифуркации по степени негрубости , нужно ввести
некоторое число условий невырожденности (условий типа неравенств). Таким
образом, в пространстве параметров динамической системы бифуркационные
условия задают некоторое многообразие коразмерности $\vec k$ , а условия
невырожденности выделют на этом многообразии области, каждой из которых
отвечает одна и та же определенная качественная структура разбиения
фазового пространства на траектории. При этом любая $\vec k$ - параметрическая
система, удовлетворяющая $\vec k$ - бифуркационным условиям и условиям
невырожденности, может быть использована для изучения данной бифуркации
и описывает ее в любой конкретной системе. Ясно, что наиболее
распространенными бифуркациями являются бифуркации коразмерности 1,
которые часто называют основными.

\section{Седло-узловая буфуркация}%
\label{sec:8.2}

Рассмотрим систему на фазовой плоскости, правые части которой зависят от параметра $\mu$
 \begin{equation}
        \label{eq:8.1}
        \begin{cases}
                \dot x_1 = f_1(x_1,x_2,\mu),\\
                \dot x_2 = f_2(x_1,x_2,\mu). 
        \end{cases}
\end{equation}
Без ограничения общности будем считать, что система \eqref{eq:8.1} при $\mu=0$ имеет
состояние равновесия $O_0$ в начале координат. Будем предполагать, что у
состояния равновесия $O_0$ один из показателей равный нулю, а второй не равен
нулю при всех $\mu \in [-\mu_0,\mu_0]$, где $0<\mu_0\ll 1$.
 При этих предположениях
нормальная форма для седло-узловой бифуркации на плоскости имеет вид
\begin{equation}
        \label{eq:8.2}
        \begin{cases}
           \dot u_1 = \mu + l(\mu) u_1^2 + \dots\\
           \dot u_2 = \lambda_2(\mu) u_2 + \dots 
        \end{cases}
\end{equation}
Для системы \eqref{eq:8.2} бифуркационное условие задается следующим образом
\begin{equation}
        \label{eq:8.3}
        \lambda_1(0)=0,
\end{equation}
а условия невырожденности --
\begin{equation}
        \label{eq:8.4}
        \lambda_2(\mu) \neq 0, \quad l(\mu) \neq 0, \quad \mu \in [-\mu_0,\mu_0].
\end{equation}
Пусть для определенности $l( \mu) > 0$ и $\lambda_2(\mu)<0$. Построим фазовые портреты системы \eqref{eq:8.2} для различный значений параметра $\mu$. Рассмотрим сначала динамику первого уравнения 
системы \eqref{eq:8.2}. На рис.\ref{fig:8.1}a представлено разбиении линии $\qty{ u_2= 0+ \dots}$ на траектории. При $m<0$ существует два состояния равновесия, имеющие координаты $u_1 = \pm \sqrt{- \frac{\mu}{l}} + \dots$. При $\mu=0$ они сливаются, образуя в начале координат двухкратное состояние
равновесия (см. лекцию \ref{lect2}), которое исчезает при $\mu>0$. Второе уравнение в системе \eqref{eq:8.2} также является уравнением первого порядка и его свойства легко устанавливаются -- любая траектория с нетривиальным начальным условием асимптотически стремится к значению $u_2=0+\dots$. Опираясь на установленные свойства каждого из уравнений в системе \eqref{eq:8.2}, построим фазовые портреты системы \eqref{eq:8.2}. Легко видеть, что при $\mu<0$ система \eqref{eq:8.2} имеет два состояния равновесия
\begin{equation}
        \label{eq:}
        O_1 \qty( u_1 = -\sqrt{-\frac{\mu}{l}} + \dots,~ u_2 = 0 + \dots ),
        O_2 = \qty( u_1 = + \sqrt{ - \frac{\mu}{l}} + \dots,~ u_2=0+\dots).
\end{equation}
\begin{figure}[h]
        \begin{minipage}{0.32\linewidth}
            \centering
            \includegraphics[width=0.6\linewidth]{example-image-a} 

            (a)
        \end{minipage}
        \begin{minipage}{0.32\linewidth}
            \centering
            \includegraphics[width=0.6\linewidth]{example-image-a} 

            (b)
        \end{minipage}
        \begin{minipage}{0.32\linewidth}
            \centering
            \includegraphics[width=0.6\linewidth]{example-image-a} 

            (c)
        \end{minipage}
        \caption{Фазовый портрет системы \eqref{eq:8.2} для различных значений параметра
        $\mu$ в случае $\lambda_2(\mu)<0,~l(\mu) >0.$}
        \label{fig:8.1}
\end{figure}
Точка $O_1$ является устойчивым узлом, а $O_2$ -- седлом. Ведущее направление узла
$O_1$ задается уравнением $u_2=0+\dots$, а неведущее -- $u_1= - \sqrt{-\frac{\mu}{l}} + \dots.$ 
Неустойчивые и устойчивые сепаратрисы седла имеют, соответственно следующий
вид
\begin{equation}
        \label{eq:}
        \qty{ u_1 = 0+\dots} \text{  и  } \qty{u_2 = \sqrt{-\frac{\mu}{l}} + \dots}.
\end{equation}
Входящие сепаратрисы сепаратрисы седла $O_2$ делят окрестность начала координат на две области (см. рис.\ref{fig:8.1}а). Из одной из этих областей все траектории системы \eqref{eq:8.2}  с  начальными условиями $u_1(0) > \sqrt{-\frac{\mu}{l}} + \dots, ~ u_2(0) \neq 0$ покидают окрестность $O_2$ 
асимптотически приближаясь к неустойчивой сепаратрисе. 
Все траектории
системы \eqref{eq:8.2}, начинающиеся во второй области, асимптотически стремятся к
устойчивому узлу  $O_1$. При $\mu=0$ состояние равновесия $O_1$ и $O_2$ сливаются,
образуя состояние равновесия $O_0(0,0)$ (рис. \ref{fig:8.1}b), которое называется седло-
узлом или двукратным равновесием. Седло-узел $O_0$ состоит из cедловой и
узловой областей. Структура узловой области качественно повторяет поведение
траекторий в окрестности устойчивого узла. Седловая область состоит из одной
одномерной выходящей сепаратрисы, уравнение которой $u_2 = 0+\dots$, к которой,
покидая окрестность $O_0$ , асимптотически приближаются все остальные
траектории. Разделение узловой и cедловой областей осуществляется двумя
траекториями вида $u_1=0+\dots$ (рис. \ref{fig:8.1}b). При $\mu>0$ система \eqref{eq:8.2} состояний
равновесия не имеет, и все траектории покидают окрестность начала координат
(рис. \ref{fig:8.1}c).

Пусть теперь $\lambda_2(\mu)>0,$ а $l(\mu)$ по-прежнему является положительной величиной. 
Очевидно, что поведение переменной $u_1$ не меняется, а $u_2$ $(u_2(0)\neq 0)$ с течением времени возрастает. Принимая во внимание эти свойства и рассуждая аналогично предыдущему, устанавливаем фазовые портреты
системы \eqref{eq:8.2} (см. рис.\ref{fig:8.2})
\begin{figure}[h]
        \begin{minipage}{0.32\linewidth}
            \centering
            \includegraphics[width=0.6\linewidth]{example-image-a} 

            (a)
        \end{minipage}
        \begin{minipage}{0.32\linewidth}
            \centering
            \includegraphics[width=0.6\linewidth]{example-image-a} 

            (b)
        \end{minipage}
        \begin{minipage}{0.32\linewidth}
            \centering
            \includegraphics[width=0.6\linewidth]{example-image-a} 

            (c)
        \end{minipage}
        \caption{Фазовый портрет системы \eqref{eq:8.2} для различных значений параметра
        $\mu$ в случае $\lambda_2(\mu)>0,~l(\mu) >0.$}
        \label{fig:8.1}
\end{figure}

В этом случае состояние равновесия $O_1$ является седлом, а $O_2$ --
неустойчивым узлом. Как и в предыдущем случае, при $\mu=0$ образуется седло-
узел $O_0$, но в этом случае узловая область состоит из неустойчивых траекторий
(рис. \ref{fig:8.2}b), а сепаратриса седловой области является устойчивой. Кроме этой
сепаратрисы и точки $O_0$ , все траектории системы \eqref{eq:8.2} покидают окрестность
$O_0$. При $\mu>0$ состояние равновесия $O_0$ исчезает и все траектории покидают
окрестность начала координат, удаляясь от линии $\qty{u_2=0+\dots}$ (рис. \ref{fig:8.2}c).
Таким образом, седло-узел (двукратное равновесие) -- негрубое состояние
равновесия, которое при сколь угодно малом изменении параметра либо
распадается на два грубых, либо исчезает.
В качестве примера возникновения седло-узловой бифуркации
рассмотрим динамику математического маятника в вязкой среде с
приложенным внешним вращающим моментом (рис. \ref{fig:8.3}). Динамика маятника
описывается системой следующего вида
\begin{equation}
        \label{eq:8.5}
        \begin{cases}
                \dot \phi =y,\\
               \dot y = \gamma - \sin \phi - \lambda y, 
        \end{cases}
\end{equation}
где $\phi$ - угол отклонения маятника от вертикали, параметр $\gamma>0$ 
характеризует действие внешнего вращательного момента, а $\lambda>0$
- вязкость среды.
 \begin{figure}[h]
        \centering
        \includegraphics[width=0.6\linewidth]{example-image-a}
        \caption{Математический маятник с приложенным внешним вращающим моментом.}
        \label{fig:8.3}
\end{figure}
Фазовый пространством системы \eqref{eq:8.5} является фазовый цилиндр $G= S^1 \times \R$.
Нетрудно видеть, что при $\gamma<1$ система \eqref{eq:8.5} имеет в $G$ два состояния равновесия:
$O_1(\phi=\phi_1, y=0)$ и $O_2(\phi=\phi_2, y=0)$, где $\phi_1= \arcsin \gamma$,
$\phi_2 = \pi - \arcsin \gamma$
Состояние равновесия $O_1$ имеет следующие характеристические показатели
\begin{equation}
        \label{eq:8.6}
        \lambda_{1,2} = - \frac{\lambda}{2} \pm \sqrt{ \frac{\lambda^2}{4} - \sqrt{1-\gamma^2} }.
\end{equation}
Из \eqref{eq:8.6} следует, что при $\lambda^2 \geq 4 \sqrt{1-\gamma^2}$ точка $O_1$ является устойчивым узлом, а при 
$\lambda^2 < 4 \sqrt{1-\gamma^2}$ - устойчивым фокусом. Точка $O_2$, всегда седло.
При $\gamma=1$ существует одно состояние равновесия $O_0(\phi= \frac{\pi}{2}, y=0)$, а пои $\gamma>1$ система \eqref{eq:8.5} состояний равновесия не имеет. Следовательно, при $\gamma=1$ 
происходит слияние точек $O_1$ и $O_2$ и образование двухкратного равновесия $O_0$.
Поскольку в окрестности значения $\gamma=1$ состояние равновесия $O_1$--устойчивый узел
, а $O_2$ седло, то состояние равновесия $O_0$ - седло-узел с устойчивой узловой областью 
и неустойчивой выходящей сепаратрисой.

\section{Бифуркация Андронова-Хопфа}%
\label{sec:8.3}

Предположим, что состояние равновесия $O_0$ системы \eqref{eq:8.1} имеет комплексно-сопряженные характеристические 
показатели $\lambda_{1,2}(\mu) = \alpha(\mu) \pm i \beta(\mu)$ и при $\mu=0$ 
выполняется бифуркационное условие
\begin{equation}
        \label{eq:8.7}
        \alpha(0) = 0.
\end{equation}
Пусть будут выполнены также следующие условия невырожденности
\begin{equation}
        \label{eq:8.8}
        \beta(\mu) \neq 0,\quad L(\mu) \neq 0,\quad \dv{\alpha(\mu)}{\mu} \eval_{\mu=0} \neq 0.
\end{equation}
Величина $L(\mu)$ называется \textbf{первой ляпуновской величиной} для $O_0$ и от её знака
при $\mu=0$ зависит структура разбиения фазовой плоскости на траектории в
окрестности состояния равновесия. Нормальная форма для бифуркации
Андронова-Хопфа имеет вид
\begin{equation}
        \label{eq:8.9}
        \begin{cases}
                \dot u_1 = \alpha(\mu) u_1 -\beta(\mu) u_2 + L(\mu)\qty(u_1^2+u_2^2) u_1+\dots\\
                \dot u_2= \beta(\mu) u_1 +\alpha(\mu) u_2 + L(\mu)\qty(u_1^2+u_2^2) u_2 +\dots
        \end{cases}
\end{equation}
Опишем кратко процедуру приведения системы \eqref{eq:8.1} к виду \eqref{eq:8.9}. Условно ее
можно разбить на несколько <<шагов>>. Первый шаг процедуры состоит в
разложении правых частей системы \eqref{eq:8.1} в окрестности точки $O_0$ в ряды
Тейлора до третьей степени. Затем с помощью линейного преобразования
координат (см. лекцию \ref{lect3}) матрица линейной части системы преобразуется к
жордановой форме. После этого, с помощью нелинейного преобразования
координат, исследуемая система приводится к виду, когда в ее правых частях
отсутствуют квадратичные слагаемые. Такое преобразование координат
существует при выполнении условий \eqref{eq:8.7}, \eqref{eq:8.8}.

Перейдем в системе \eqref{eq:8.9} к полярным координатам с помощью замены $u_1 = \rho\cos \phi,$ 
 $u_2=\rho\sin \phi.$ В результате получим эквивалентную \eqref{eq:8.9} систему вида
 \begin{equation}
         \label{eq:8.10}
         \begin{cases}
                 \dot \phi = \beta(\mu) + \dots \\
                 \dot \rho = \alpha(\mu) \rho + L(\mu) \rho^3 + \dots  
         \end{cases}
 \end{equation}
 Анализ системы \eqref{eq:8.10} удобнее провести, рассматривая сначала динамику
первого и второго уравнений отдельно. Пусть для определенности зависимость
$\alpha(\mu)$ удовлетворяет условию $\alpha(\mu) \cdot \mu >0$, если $\mu\neq 0$. Из первого уравнения в
системе \eqref{eq:8.10} имеем
\begin{equation}
        \label{eq:8.11}
        \phi(t) = \beta(\mu) t + \phi_0+\dots,
\end{equation}
где $\phi_0=\const$. Следовательно, переменная $\phi$ совершает вращательные движения с частотой 
$\beta(\mu)$. Динамика второго уравнения в \eqref{eq:8.10} зависит от знака величины $L(\mu)$.

\subsection{Первая ляпуновская величина отрицательна.}%
\label{sub:8.3.1}

Предположим, что $L(0)<0$. В этом случае, кроме тривиального состояния
равновесия, существующего при всех значениях параметра $\mu$, второе уравнение
при $\mu>0$ имеет также нетривиальное состояние равновесия
Тривиальное состояние равновесия является $\rho = \sqrt{ \frac{\alpha(\mu)}{L(\mu)}}.$ 
Тривиальное состояние равновесия является устойчивым при $\mu\leq 0$ и 
неустойчивым при $\mu>0$, а нетривиальное состояние является устойчивым (рис.
\ref{fig:8.4}b). Отсюда, принимая во внимание \eqref{eq:8.11}, устанавливаем фазовый портрет
системы \eqref{eq:8.9} для различных значений параметра $\mu$ (рис. \ref{fig:8.4}c). При изменении
параметра $\mu$ состояние равновесия $O_0$ теряет устойчивость, и из него рождается
устойчивый предельный цикл. Обратим внимание на то, что в момент
бифуркации состояние равновесия является устойчивым сложным фокусом, у
которого <<шаг>> спирали существенно меньше, чем у обычного фокуса,
поскольку при $\mu=0$ переменная $\rho$ изменяется в соответствии с уравнением
\begin{equation}
        \label{eq:}
        \dot \rho = L(0) \rho^3 + \dots
\end{equation}

\begin{figure}[h]
        \centering
        \begin{minipage}{0.25\textheight}
            \centering
            \includegraphics[width=\linewidth]{example-image-a} 

            (a)
        \end{minipage}
        \vfill
        \begin{minipage}{0.25\textheight}
            \centering
            \includegraphics[width=\linewidth]{example-image-a} 

            (b)
        \end{minipage}
        \vfill
        \begin{minipage}{0.25\textheight}
            \centering
            \includegraphics[width=\linewidth]{example-image-a} 

            (c)
        \end{minipage}
        \caption{Расположение характеристических показателей состояния равновесия
        $O_0$ на комплексной плоскости (a); динамика второго уравнения системы
\eqref{eq:8.10} (b); фазовый портрет системы \eqref{eq:8.10} для различных значений параметра $\mu$ (c).}
        \label{fig:8.4}
\end{figure}

\subsection{Первая ляпуновская величина положительна}%
\label{sub:8.3.2}

Пусть теперь $L(0)>0$. В этом случае нетривиальное состояние равновесия
уравнения для $\rho$ существует при $\mu<0$ и является неустойчивым (рис. \ref{fig:8.5}a).
Тривиальное состояние равновесия устойчиво при $\mu<0$ и неустойчиво при $m\geq 0$.
Такая динамика переменной $\rho$ при учете \eqref{eq:8.11} приводит к фазовому портрету
системы \eqref{eq:8.10}, представленному на рис. \ref{fig:8.5}b. В этом случае предельный цикл
является неустойчивым и существует при $\mu<0$. При $\mu=0$ цикл стягивается в
точку и состояние равновесия $O_0$ становится неустойчивым сложным фокусом.

\begin{figure}[h]
        \centering
        \begin{minipage}{0.25\textheight}
            \centering
            \includegraphics[width=\linewidth]{example-image-a} 

            (a)
        \end{minipage}
        \vfill
        \begin{minipage}{0.25\textheight}
            \centering
            \includegraphics[width=\linewidth]{example-image-a} 

            (b)
        \end{minipage}
        \caption{Динамика второго уравнения системы \eqref{eq:8.10} (a); фазовый портрет системы
        \eqref{eq:8.10}b для различных значений параметра $\mu$.}
        \label{fig:8.4}
\end{figure}

\subsection{<<Мягкое>> и <<жесткое>> рождение периодических колебаний}%
\label{sub:8.3.3}

Рождение цикла в случае $L(0)<0$ часто называют \textbf{мягким} (в английской
литературе supercritical bifurcation, отражая этим термином факт появления
предельного цикла после прохождения параметром бифуркационного
значения), поскольку амплитуда цикла плавно нарастает от нуля. При этом
границу области устойчивости состояния равновесия $O_0$ (значение $\mu=0$)
принято называть <<безопасной>>, так как, несмотря на потерю устойчивости при
$\mu>0$, состояние реальной системы сохраняется в малой окрестности точки $O_0$ .
Совершенно иная ситуация в случае $L(0)>0$ (это так называемая subrcritical
bifurcation; термин <<subritical>> отражает существование предельного цикла до
прохождения бифуркационного значения). Здесь при нарушении условий
устойчивости все траектории из окрестности $O_0$ переходят на другой аттрактор.
Если новый аттрактор является предельным циклом, то говорят о жёстком
рождении периодических колебаний. При этом граница области устойчивости
называется <<опасной>>, так как происходит резкое изменение состояния
реальной системы.

Величина $L(0)$ может быть вычислена через правые части системы \eqref{eq:8.1} с помощью
следующей формулы
\begin{align}
        \label{eq:8.12}
       & L(0) = \frac{1}{16} \qty[ \pdv[3]{f_1}{x_1} + \pdv[3]{f_1}{x_1^2}{x_2} +
        \pdv[3]{f_2}{x_2}]
        + \frac{1}{16 \beta(0)}
        \qty[
        \pdv{f_1}{x_1}{x_2}
        \qty( \pdv[2]{f_1}{x_1} + \pdv[2]{f_1}{x_2})] \\
       & + \frac{1}{16 \beta(0)} 
        \qty[- 
        \pdv{f_1}{x_1}{x_2} \qty( \pdv[2]{f_2}{x_1} + \pdv[2]{f_2}{x_1}) -
        \pdv{f_1}{x_1}{x_2} \cdot \pdv[2]{f_2}{x_2} +
        \pdv[2]{f_1}{x_2} +
        \pdv[2]{f_2}{x_2} 
        ],
\end{align}
где производные вычисляются в состоянии равновесия.

Отметим, что впервые бифуркация рождения предельного цикла из
состояния равновесия с чисто мнимыми характеристическими показателями в
системах на фазовой плоскости была обнаружена в 1939 году А.А. Андроновым
и Е.А. Леонтович. В 1942 году Э.Хопф распространил эту теорию на случай
многомерных систем. Выражение для первой ляпуновской величины было
получено Н.Н. Баутиным в 1949 году.


\section{Затягивание потери устойчивости при динамической бифуркации Андронова-Хопфа.}%
\label{sec:8.4}

Во многих практических задачах параметры системы не являются строго
постоянными, а медленно изменяются во времени. Рассмотрим влияние этого
эффекта на бифуркацию Андронова-Хопфа на примере следующей системы
\begin{equation}
        \label{eq:8.13}
        \begin{cases}
                \dot x = \mu x - y - x(x^2+y^2),\\
                \dot y = x + \mu y - y(x^2+y^2),
        \end{cases}
\end{equation}
где $\mu$ - контрольный параметр.

Пусть сначала $\mu = \const$. Нетрудно видеть, что в этом случае система
\eqref{eq:8.13} имеет вид \eqref{eq:8.9} с $\alpha(\mu) = \mu,~\beta(\mu)=1$ и $L(\mu)=-1$.
Следовательно, система \eqref{eq:8.13} при $\mu<0$ имеет устойчивый фокус,
притягивающий все остальные траектории.  При $\mu=0$ происходит буфуркация Андронова-Хопфа, в результате которой на фазовой плоскости мягко появляется устойчивый предельный цикл,
притягивающий все нетривиальные траектории, а состояние равновесия становится неустойчивым фокусом
(см. рис.\ref{fig:8.4}c).

Пусть теперь параметр $\mu$ растет медленно во времени, т.е. система \eqref{eq:8.13} принимает вид
\begin{equation}
        \label{eq:8.14}
        \begin{cases}
                \dot x = \mu x - y - x(x^2+y^2), \\
                \dot y = x + \mu y - y(x^2+y^2),\\ 
                \dot \mu = \epsilon,
        \end{cases}
\end{equation}
где $0<\epsilon\ll 1$. Прежде всего заметим, что система \eqref{eq:8.14} имеет трехмерное фазовое пространство. Перейдем в системе 
\eqref{eq:8.14} к полярным координатам $x=\rho\cos \phi,~ y = \rho\sin \phi$. В результате получим
\begin{equation}
        \label{eq:8.15}
        \begin{cases}
                \dot \phi =1,\\
                \dot \phi = \rho( \mu(t) - \rho^2),\\
                \dot \mu = \epsilon.
        \end{cases}
\end{equation}
Не будем пока принимать во внимание изменение переменной $\phi~(\phi=t+\phi_0)$, а исследуем динамику системы
\begin{equation}
        \label{eq:8.16}
        \begin{cases}
                \dot \rho = \rho( \mu(t) - \rho^2)\\
                \dot \mu = \epsilon.
        \end{cases}
\end{equation}
Рассмотрим поведение произвольной траектории с начальными условиями:
$\rho(0) = \rho_0$ , $\mu(0) = - \mu_0$ , где $\mu_0>0$, a $\rho_0$  
выбрано вне малой (порядка $\epsilon$ ) окрестности $U_{\epsilon}$
прямой $\rho=0$, которая, очевидно, является решением первого уравнения системы
\eqref{eq:8.16}. Поскольку $\epsilon<<l$, изменение переменной $\mu(t)$ происходит значительно
медленнее, чем переменной $\rho(t)$. Следовательно, в первом приближении можно
считать, что движение рассматриваемой траектории определяется в основном
уравнением
\begin{equation}
        \label{eq:8.17}
        \dot \rho = \rho \qty( - \mu_0 - \rho^2).
\end{equation}
При таком значении параметра $\mu= - \mu_0<0$ переменная $\rho(t)$ монотонно убывает к
значению $\rho=0$ (см. рис. \ref{fig:8.4}b). Переменная $\rho(t)$ убывает в течение некоторого
конечного времени $\tau$ , пока значение $\rho(t)$ не достигнет $U_{\epsilon}$.
В этой окрестности
уравнение \eqref{eq:8.17} становится непригодным для описания движения исследуемой
траектории, и мы должны учитывать оба уравнения системы \eqref{eq:8.16} при
начальных условиях
\begin{equation}
        \label{eq:8.18}
        \rho(\tau) = p,\quad \mu(\tau) \simeq -\mu_0,
\end{equation}
где $p$ - граничная точка окрестности $U_{\epsilon}$. Очевидно, что из второго уравнения
системы \eqref{eq:8.16} следует
\begin{equation}
        \label{eq:8.19}
        \mu = \epsilon(t-\tau) - \mu_0 \text{ при } t\geq\tau.
\end{equation}
Рассмотрим эволюцию переменной $\rho$ . Из \eqref{eq:8.16} следует, что переменная $\rho$
будет монотонно убывать, по крайней мере, пока переменная $\mu(t)$ остается
отрицательной, т.е. до значения $t = \tau + \frac{\mu_0}{\epsilon}$. 
Следовательно, на временном
интервале от $\tau$ до $t = \tau + \frac{\mu_0}{\epsilon}$
 рассматриваемая траектория расположена в
малой окрестности прямой $\rho=0$. Оценим теперь полное время нахождения
траектории в окрестности
$U_{\epsilon}$ . В этой окрестности член 
$\rho^2(t)$
 в первом
 уравнении системы \eqref{eq:8.16} пренебрежимо мал по сравнению с $\mu(t)$. Поэтому
 динамика переменной $\rho(t)$ в основном определяется уравнением
 \begin{equation}
         \label{eq:8.20}
         \dot \rho = \mu(t) \rho.
 \end{equation}
 Проинтегрировав уравнение \eqref{eq:2.20} на интервале от $\tau$ до $t$, при учете соотношений \eqref{eq:8.18}, \eqref{eq:8.19}, получим
 \begin{equation}
         \label{eq:8.21}
         \rho(t) = p e^{\epsilon\frac{t-\tau}{2}^2 - \mu_0 (t-\tau)}.
 \end{equation}
 Из \eqref{eq:8.21} вытекает, что переменная $\rho(t)$ вновь достигает значения $\rho(t)$ 
 вновь достигает значения $p$ в момент времени $t=\tau +2 \frac{\mu_0}{\epsilon}$, когда переменная
 $\mu(t)$ станет равной $\mu_0$. После выхода из $U_{\epsilon}$ 
 переменная $\rho(t)$ начинает быстро нарастать и асимптотически приближается к значению
 $\rho = \sqrt{ \mu_0}$ (см. рис.\ref{fig:8.4}b), поскольку $m_0>0$. Следовательно, динамика системы \eqref{eq:8.16} отличается от статического случая. Во-первых, при прохождении значения $\mu(t) = 0 $ принципиального изменения, как это имело место в случае $\mu=\const$,
 в динамике не происходит, а, во-вторых, существует нового пороговое значение $\mu_0>0$, при котором возникает скачкообразное увеличение переменной $\rho(t).$

 Вернемся теперь к исходной системе \eqref{eq:8.14}. Можно показать
(доказательство базируется на грубости предельного цикла, существующего в
статическом случае при $\mu>0$), что в трехмерном фазовом пространстве этой
системы при $\mu(t)>0$ существует двумерная устойчивая инвариантная
поверхность $C^S$ , близкая к поверхности, составленной из предельных циклов
системы \eqref{eq:8.13}. Как мы увидим далее, поверхность $C^S$ играет важную роль в
динамике системы \eqref{eq:8.14}. Рассмотрим движение исследуемой нами траектории
в $\R^3$. Из установленной выше динамики переменных $\phi$, $\rho$ и $\mu$ вытекает
следующее. Сначала фазовая точка скачком притянется в окрестность прямой
$x=y=0$ и будет двигаться в окрестности $U_{\epsilon}$, совершая вращательные движения
до тех пор, пока переменная $\mu$ не достигнет значения $\mu_0$ (рис. \ref{fig:8.6}). Только после
этого произойдет срыв из окрестности прямой $x=y=0$ и фазовая точка
притянется быстро к поверхности $C^S$ (рис. \ref{fig:8.6}). В окрестности $C^S$ она начнет
совершать вращательные движения, амплитуда которых нарастает $\sim \sqrt{\mu(t)}$ , а
время их возникновения $\sim \frac{1}{\epsilon}$ является достаточно большой величиной. 
\begin{figure}[h]
        \centering
        \includegraphics[width=0.6\linewidth]{example-image-a}
        \caption{Фазовое пространство системы \eqref{eq:8.14}: эффект затягивания потери устойчивости.}
        \label{fig:8.6}
\end{figure}

Описанный механизм возникновения колебаний называется \textbf{динамической бифуркацией Андронова-Хопфа},
для которой характерны следующие эффекты:
\begin{itemize}
        \item затягивания потери устойчивости (колебания возникают при $\mu=\mu_0$, а не при $\mu=0$ 
                , как в статическом случае);
        \item жесткого возникновения колебаний;
        \item памяти -- колебания возникают при значении $\mu=\mu_0$, однозначно связанным начальным значением $-\mu_0$ 
\end{itemize}

