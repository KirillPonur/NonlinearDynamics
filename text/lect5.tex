%!TEX root = ..\lections.tex
Осциллятор – простейшая динамическая система с двумерным фазовым
пространством. Несмотря на простоту, с помощью этой системы можно описать
важнейшие колебательные процессы: периодические, затухающие и
нарастающие. Круг реальных задач, приводящих к модели в виде
осциллятора,чрезвычайно широк и имеет самую разнообразную природу.
Например, к таким задачам относятся различные механические устройства, в
которых происходит взаимодействие масс и упругих сил, электрические
контуры, содержащие ёмкостные и индуктивные компоненты, некоторые виды
акустических резонаторов, простейшие популяционные задачи и др. Изучение
динамических свойств осцилляторов мы начнём с задач, в которых нелинейные
механизмы отсутствуют или пренебрежимо малы.

\section{Динамика линейного осциллятора}%
\label{sec:5.1}

Рассмотрим электрический контур, состоящий из последовательно соединённый ёмкости $C$, индуктивности $L$ и сопротивления $R$ (см. рис.\ref{fig:5.1}а).
Обозначим через $q$ заряд конденсатора $C$. Согласно закону Кирхгофа
\begin{equation}
        \label{eq:5.1}
        u_r + u_1 + u_c =0,
\end{equation}

\begin{figure}[h!]
        \centering
        \includegraphics[width=0.6\linewidth]{example-image-a}
        \caption{Линейные осцилляторы: электрический контур (а); груз массы $m$ на пружине с жёсткостью $k$, совершающий малые колебания около положения равновесия (b).}
        \label{fig:5.1}
\end{figure}
т.е. сумма падений напряжения на элементах контура равна нулю, поскольку в цепи отсутствуют внешние источники напряжения. Пусть $i$-- ток, протекающий в контуре, который, как известно, связан с зарядом $q$ следующим образом
\begin{equation}
        \label{eq:5.2}
        i = \dv{q}{t}.
\end{equation}
Тогда для напряжение на элементах контура можно записать
\begin{equation}
        \label{eq:5.3}
        u_R = Ri = R \dv{q}{t}, ~ u_L = L \dv{i}{t} = L \dv[2]{q}{t},~ u_C= \frac{q}{C}.
\end{equation}
Подставляя \eqref{eq:5.3} в \eqref{eq:5.1}, получим уравнение
\begin{equation}
        \label{eq:5.4}
        L\dv[2]{q}{t} + R \dv{q}{t} + \frac{q}{C} = 0.
\end{equation}
Перепишем уравнение \eqref{eq:5.4}, для удобства дальнейшего изложения, в следующем эквивалентном виде
\begin{equation}
        \label{eq:5.5}
        \ddot x + 2 \delta x + \omega_0^2 x =0,
\end{equation}
где 
\begin{equation}
        \label{eq:}
        2 \delta = \frac{R}{L}, ~ \omega_0^2=\frac{1}{LC}.
\end{equation}
Реальные системы, динамика которых описывается уравнением \eqref{eq:5.5}, принято называть
\textbf{линейными осцилляторами}. Уравнение \eqref{eq:5.5} содержит два параметра, имеющих ясный смысл: $\omega_0$--частота собственных колебаний, а параметр $\delta$ характеризует потери в системе.

Другим примером линейного осциллятора может служить груз на пружине (см. рис.\ref{fig:5.1}b), совершающий малые колебания вблизи положения равновесия при наличии силы трения пропорциональной скорости $\dot x$. Динамика такой системы также описывается уравнением \eqref{eq:5.5}, в котором $x$ -- смещение груза из положения равновесия.

\subsection{Гармонический осциллятор}%
\label{sub:5.1.1}


