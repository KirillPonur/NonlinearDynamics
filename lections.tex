\documentclass[a4paper,14pt]{extarticle}

% Шрифты, кодировки, символьные таблицы, переносы
\usepackage{cmap}
\usepackage[T2A]{fontenc}
\usepackage[utf8x]{inputenc}
\usepackage[english, russian]{babel}

% Пакеты американского математического сообщества
\usepackage{amssymb,amsfonts,amsmath,amsthm}  
% Сокращения
\usepackage{cancel}

\theoremstyle{definition}
\newtheorem{definition}{Определение}

% Красная строка
\usepackage{indentfirst}

% Ссылки в pdf 
\usepackage[unicode, colorlinks, urlcolor=magenta, linkcolor=black]{hyperref}

% Таблицы
\usepackage{makecell,multirow} 

% Графика
\usepackage{graphicx}
\usepackage[usenames,dvipsnames]{color} 
\usepackage{float}

% Геометрия страницы
\usepackage{geometry}
\geometry{left=2cm,right=2cm,top=2.5cm,bottom=2.5cm,bindingoffset=0cm,headheight=18pt}

% Колонтитулы
\usepackage{fancyhdr} 
% применим колонтитул к стилю страницы
\pagestyle{fancy} 
%очистим "шапку" страницы
\fancyhead{} 
%слева сверху на чётных и справа на нечётных
\fancyhead[R]{Лекции В.И. Некоркина 2018-2019} 
% \fancyhead[R]{Сарафанов Ф.Г., Понур К.А. и др.} 
%справа сверху на четных и слева на нечетных
\fancyhead[L]{Теория колебаний} 
%очистим "подвал" страницы
\fancyfoot{} 
% номер страницы в нижнем колонтитуле в центре
\fancyfoot[C]{\thepage} 

% Междустрочный отступ
\usepackage{setspace}
\linespread{1.15} % капельку увеличенный
\frenchspacing % <<французские>> пробелы

% Нумерация
\renewcommand{\labelenumii}{\theenumii)}
% В заголовках появляется точка, но при ссылке на них ее нет
\usepackage{misccorr}

% Содержание
\usepackage{tocloft}
\usepackage{secdot}
\sectiondot{subsection}

% Физика
\usepackage{physics}

% Новые команды
\newcommand{\Mean}[1]{\langle#1\rangle}
\newcommand{\Defi}{\underset{def}{=}}
\newcommand{\Inte}{\int\limits_{-\infty}^{\infty}} 
\newcommand{\const}{\textrm{const}}

\addto\captionsrussian{%
	\renewcommand{\contentsname}{Оглавление}
	\renewcommand{\partname}{Раздел}%
}
\def\thepart{\arabic{part}}
\usepackage{tocloft}
\renewcommand{\cftpartleader}{\cftdotfill{\cftdotsep}} % for parts
\renewcommand{\cftsecleader}{\cftdotfill{\cftdotsep}} % for chapters
\renewcommand\thepart{\arabic{part}.}

\renewcommand{\cftsecaftersnum}{.}

\renewcommand{\cftsecdotsep}{\cftdotsep}
\renewcommand{\kappa}{\varkappa}
\renewcommand{\phi}{\varphi}
\renewcommand{\epsilon}{\varepsilon}
% #1: math symbol
% #2: legend
\def\alegend#1#2{\overset{\underset{\scriptstyle\downarrow}{\scriptstyle\text{#2}}}{#1}}
\def\blegend#1#2{\underset{\underset{\scriptstyle\text{#2}}{\scriptstyle\uparrow}}{#1}}
\def\hp{\hat{p}}
\def\hx{\hat{x}}
\def\hH{\hat{H}}

\usepackage[explicit]{titlesec}
\usepackage{epigraph}


\newcommand\praktika[1]{
\stepcounter{section}
\vspace{1.5em}
\noindent\textbf{\Large{Занятие \arabic{section}.\hspace{.2em} #1}}
% \newline 
\vspace{-0.5em}
\addcontentsline{toc}{section}{Занятие \arabic{section}.\hspace{.5em} #1}
}

\usepackage{mathtools}
% \mathtoolsset{showonlyrefs=true}


% https://tex.stackexchange.com/questions/8720/overbrace-underbrace-but-with-an-arrow-instead

\usepackage{xparse}% http://ctan.org/pkg/xparse

\NewDocumentCommand{\overarrow}{O{=} O{\uparrow} m}{%
  \overset{\makebox[0pt]{\begin{tabular}{@{}c@{}}$#3$\\[0pt]\ensuremath{#2}\end{tabular}}}{#1}
}
\NewDocumentCommand{\underarrow}{O{=} O{\downarrow} m}{%
  \underset{\makebox[0pt]{\begin{tabular}{@{}c@{}}\ensuremath{#2}\\[0pt]$#3$\end{tabular}}}{#1}
}

\newcommand\undernoteqty[2]{
	%
	\underarrow[
		\qty(\underbrace{#1})
	][\uparrow]{\substack{#2}}
	%
}

\newcommand{\pvec}[1]{\vec{#1}\mkern2mu\vphantom{#1}}
% Нормальный вектор для штрихов
\newcommand{\phat}[1]{\hat{#1}\mkern2mu\vphantom{#1}}

\newcommand\undernote[2]{
	%
	\underarrow[
		#1
	][\uparrow]{\substack{#2}}
	%
}



% ##############################################################################
\newcommand*\dotvec[1][1,1]{\crossproducttemp#1\relax}
\def\crossproducttemp#1,#2\relax{{\qty[\vec{#1}\times\vec{#2}\,]}}

\newcommand*\prodvec[1][1,1]{\crossproducttempa#1\relax}
\def\crossproducttempa#1,#2\relax{{\qty[{#1}\times{#2}\,]}}
% ##############################################################################
\usepackage{kbordermatrix}%
\renewcommand{\kbldelim}{(} % change default array delimiters to parentheses
\renewcommand{\kbrdelim}{)}
\DeclareMathOperator{\diag}{diag}
\DeclareMathOperator{\Rot}{rot}
\newcommand{\markerr}{\textcolor{red}{\textbf{??}}} 
\usepackage{wrapfig,dsfont}
\usepackage{chngcntr}

\counterwithin*{equation}{section}
\begin{document}
%!TEX root = lections.tex
\begin{titlepage}
\thispagestyle{empty}

\begin{center}
	{\small\textsc{Нижегородский государственный университет имени Н.\,И. Лобачевского}}
	\vskip 3pt \hrule \vskip 5pt
	{\small\textsc{Радиофизический факультет}}

	\vfill

	\begin{spacing}{2}
	% {\huge \bf  Курин\, В.В.}\\[1.5em]
	{\Huge \bf  Лекции по основам \\ теории колебаний}\\%\vspace{1em}
	\end{spacing}
	{ Набор и вёрстка:}\\[.5em]
	{ 	\href{https://github.com/AnnaKarusevich}{\color{black}{Карусевич А.А.}}
		\href{https://github.com/KirillPonur}{\color{black}{Понур К.А.}},
		\href{https://github.com/fedorsarafanov}{\color{black}{Сарафанов Ф.Г.}}, 
		\href{https://github.com/BigBigGamer}{\color{black}{Шиков А.П.}},
	\\ Платонова М.В.}\\[2em]
	% {\large }\\
	\vspace{1em}
\end{center}

\textbf{Disclaimer.} В данном документе нами набраны лекции по теории колебаний (нелинейной динамике), прочитанные на 3 курсе радиофизического факультета ННГУ \textbf{Владимиром Исааковичем Некоркиным}, но не вошедшие в существующие методические пособия. Документ призван облегчить подготовку к зачётам и экзаменам и восполнить пробелы в знаниях читателя по теории колебаний. Разрешено копирование и распространение данного документа с обязательным указанием первоисточника. 

% При обнаружении ошибок, опечаток и прочих вещей, требующих исправления, можно либо создать issues в \href{https://github.com/FedorSarafanov/NonlinearDynamics}{репозитории на github.com}, либо написать на электронную почту  \href{mailto:sfg180@yandex.ru}{\color{black}{sfg180@yandex.ru}}.

\begin{center}
	\vfill
	4 апреля -- \today\\Нижний Новгород
\end{center}

\end{titlepage}


\newpage
\tableofcontents 

\newpage
\section{Колебания и волны в цепочке взаимосвязанных тождественных осцилляторов}
\label{sec:osci_and_wave_in_ordered_struct}
	% 08.04.19
	%!TEX root = ../lections.tex

Рассмотрим в качестве осциллятора обычный маятник, совершающий малые колебания около нижней точки равновесия. Пусть у нас есть система из связанных осцилляторов (см. рис. \ref{fig:1}): маятники с точками подвеса на расстоянии $a$ друг от друга, связанные пружинками жесткости $\gamma$. 

\begin{figure}[h!]
	\centering
	\includegraphics[scale=1.5]{img/osci_and_wave_in_ordered_struct/struct_of_pend}
	\label{fig:fig1}
\end{figure}

Обозначим $n$ -- номер маятника, $\omega_0^2=\frac{g}{l}$ -- собственная частота. Угол отклонения $n$-го маятника $\phi_n$, причем мы рассматриваем случай малых углов.


Состояние маятника зависит не только от времени $t$, но и от номера маятника $n$, то есть $n$ в некотором смысле играет роль пространственной координаты. Запишем уравнение динамики такого маятника:

\begin{equation}
	\ddot{\phi}_n+\omega^2_o \phi_n=\frac{\gamma}{m}\qty[\vphantom{\bigg|}(\phi_{n-1}-\phi_n)+(\phi_{n+1}-\phi_n)].
	\label{eq:1}
\end{equation}

Каждый маятник действует на соседние, причём сила взаимодействия зависит от разности значений углов. Упростим уравнение \eqref{eq:1}:

\begin{equation*}
	\ddot{\phi}_n+\omega^2_o \phi_n=\frac{\gamma}{m}\qty[\vphantom{\bigg|}\phi_{n-1}-2\phi_n+\phi_{n+1}].
\end{equation*}

Часто такую связь называют диффузионной, хотя, конечно, никакого отношения к процессу диффузии она не имеет. В системе нет диссипации, она линейна (нелинейность порождала бы новые частоты). Поэтому решение будем искать в следующем виде:
\begin{equation}
	\phi_n=A e^{i(\omega t-nka)}.
	\label{eq:2}
\end{equation}

Такая форма записи учитывает, что возмущение от маятника к маятнику проходит за некоторое конечное время.
\begin{equation}
	-\omega^2+\omega_0^2=\frac{\gamma}{m}\qty(e^{-ika}-2+e^{ika})
	\quad \Rightarrow \quad
	\omega^2=\omega_0^2-\frac{\gamma}{m}\qty(e^{-ika}-2+e^{ika})
	\label{eq:3}
\end{equation}
Рассмотрим случай  действительного $k$. Тогда
\begin{equation}
	\omega^2=\omega_0^2-\frac{\gamma}{m}(-2+2\cos{ka})=\omega_0^2+\frac{4\gamma}{m}\sin^2{\frac{ka}{2}}.
	\label{eq:4}
\end{equation}
Итак, мы установили, что $\omega$ и $k$ связаны соотношением \eqref{eq:4}. Построим график $\omega(k)$:
\begin{figure}[H]
	\centering
	\includegraphics[scale=1.5]{img/osci_and_wave_in_ordered_struct/disp_of_struct}
\end{figure}
Нетрудно получить из уравнения \eqref{eq:4}, что 
\begin{equation*}
	\omega_{\max}=\sqrt{\omega_0^2+\frac{4\gamma}{m}}.
\end{equation*}
Если $\omega_0 < \omega < \omega_{max}$, то каждому $\omega$ соответствуют действительные $k_0$ и $-k_0$. Это означает, что каждой частоте соответствует две гармонические бегущие волны
\begin{equation*}
	\phi_n=A e^{i(\omega t+nk_0a)} 
	\qquad\text{и}\qquad
	\phi_n=A e^{i(\omega t-nk_0a)},
\end{equation*}
где $k_0$ -- волновое число. Поскольку система линейна, любая линейная комбинация решений тоже будет решением. 

Диапазон $\omega_0 < \omega < \omega_{\max}$ называют \textbf{полосой прозрачности} или полосой пропускания. Вне этой полосы решению не отвечают действительные $k$. В этом случае число \textit{$k$ - чисто мнимое} (чисто -- так как нет диссипации в системе) и $k=i\kappa$. Подставив такую связь в \eqref{eq:4}, получим
\begin{equation}
	\omega^2=\omega_0^2-\frac{4\gamma}{m}\sh^2{\frac{\varkappa a}{2}}.
	\label{eq:5}
\end{equation}
Подставляя \eqref{eq:5} в \eqref{eq:2}, получим вид волны в этом случае $\phi_n=A e^{-n\varkappa a} e^{i\omega t}$. Заметим, что при $n\rightarrow \infty$ функция  $\phi_n \rightarrow 0$, то есть в этих областях волна не проходит. 

Если мы находимся в полосе прозрачности, то $v_\text{фаз}=v_\text{фаз}(k), v_\text{фаз}=v_\text{фаз}(\omega)$. Если фазовая скорость зависит от частоты или волнового числа, то среда диспергирующая, а \eqref{eq:4} - дисперсионное соотношение. 

Дисперсия возникает из-за наличия собственных пространственных и временных масштабов : $a$ и $\omega_0$. У каждой компоненты волнового пакета будет своя фазовая скорость, возникнет его деформация. Именно наличием собственных масштабов объясняется то, что в одних случаях система пропускает волну, а в других нет.





\subsection{Предельный переход от цепочной структуры к среде}
Введём пространственную координату $x$ вдоль балки, на которой расположены точки подвеса. Сделаем замену, считая, что $\phi_n$ зависит от двух переменных:
\begin{equation*}
	\phi_n(t) \rightarrow \phi(x,t).
\end{equation*}
Считая $a$ малым, разложим $\phi_{n\pm 1}$ в ряд Тейлора по степеням $a$:
\begin{gather}
	\phi_{n+1}(t) \rightarrow \phi(x+a,t)=\phi(x,t)+\pdv{\phi}{x}a +\frac12 \pdv[2]{\phi}{x}a^2+\ldots,\\\nonumber
	\phi_{n-1}(t) \rightarrow \phi(x-a,t)=\phi(x,t)-\pdv{\phi}{x}a +\frac12 \pdv[2]{\phi}{x}a^2+\ldots
	\label{eq:6}
\end{gather}
Полученные разложеня до второго порядка подставим в \eqref{eq:1}:
\begin{gather*}
	\pdv[2]{\phi}{t} + \omega_0^2 \phi=\frac{\gamma}{m}a^2 \pdv[2]{\phi}{x}.
\end{gather*}
Обозначим $\frac{\gamma}{m}a^2 = v^2$, тогда уравнение  примет вид, известный как \textbf{уравнение Клейна-Гордона}:
\begin{equation}
	\pdv[2]{\phi}{t}-v^2\pdv[2]{\phi}{x}+\omega_0^2\phi=0.
	\label{eq:7}
\end{equation}
Уравнение \eqref{eq:7} не что иное, как уравнение в частных производных. Когда мы можем использовать \eqref{eq:7} вместо \eqref{eq:1}? Вспомним, что мы предполагали при решении задачи:
\begin{enumerate}
	\item Угол $\phi_n$, определенный в точке, определен и между дискретными точками подвеса;
	\item Отброшены величины третьего порядка;
	\item Расстояние между точками подвеса $a$ -- мало
\end{enumerate}

Построим дисперсионную характеристику для \eqref{eq:7}. Для этого подставим решение в виде бегущей волны в \eqref{eq:7}:
\begin{gather*}
	\phi(x,t)=Ae^{i(\omega t + kx)} \quad \Rightarrow \quad -\omega^2-k^2v^2+\omega_0^2=0,
\end{gather*}
\begin{equation}
	\omega^2 = \omega_0^2 + k^2v^2.
	\label{eq:8}
\end{equation}
\begin{figure}[H]
	\centering
	\includegraphics[scale=1.5]{img/osci_and_wave_in_ordered_struct/disp_of_cont}
\end{figure}
Нетрудно получить, что график имеет две асимптоты $\omega=\pm vk$. 

\paragraph{Условие совпадения дисперсионных соотношений. } При $\lambda\gg a$ или $ka\ll$ -- это условие длинноволновой зоны -- можно от соотношения \eqref{eq:4} перейти к \eqref{eq:8}. В этом случае пространственный масштаб не сказывается и мы им пренебрегаем. 

Если  $\omega_0 \rightarrow 0$, то из \eqref{eq:8} следует, что 
\begin{equation}
	\omega^2=k^2v^2.
	\label{eq:9}
\end{equation}
В этом случае система не обладает дисперсией:

\begin{figure}[H]
	\centering
	\includegraphics[scale=1.5]{img/osci_and_wave_in_ordered_struct/no_disp} 
\end{figure}

Дисперсионная характеристика проявляется в областях прозрачности или непрозрачности и зависимости $v_\text{фаз}$ от $k$ или $\omega$.

% Получили, шарики плотно друг к другу и покоятся.

Переход от цепочки к среде называется длинноволновым переходом: при нем теряется дискретность системы.




\subsection{Дисперсионное уравнение произвольной линейной системы}

Рассмотрим произвольную многомерную систему
\begin{equation}
	A\pdv{\vec{u}}{t}+B\pdv{\vec{u}}{x}+C\vec{u}=0
	\label{eq:10}
\end{equation}
Здесь $A, B, C$ --  матрицы размера $n\times n$, а $\vec{u}(x,t)$ описывает состояние системы. Будем искать решение в виде
\begin{equation}
	\vec{u}=\vec{\Psi} e^{i(\omega t - kx)}, \quad
	\vec\Psi=
	\begin{pmatrix}
		\Psi_1 \\
		\vdots \\
		\Psi_n \\
	\end{pmatrix}
	\label{eq:11}
\end{equation}
Подставляя \eqref{eq:11} в \eqref{eq:10}, получим
\begin{equation}
	Ai\omega\vec\Psi-iBk\vec\Psi+C\vec\Psi=0
	\quad \Rightarrow \quad
	(A\omega-Bk-iC)\vec\Psi i=0.
	\label{eq:12}
\end{equation}

Уравнение \eqref{eq:12} представляет собой систему линейных однородных уравнений относительно компонент вектора $\vec\Psi$. Она имеет решение, если ее детерминант равен нулю:
\begin{equation}
	\det\qty[A\omega-Bk-iC]=0.
	\label{eq:13}
\end{equation}

Для краткости это уравнение часто записывают в виде $D(\omega,k)=0$. Оно связывает $\omega$ и $k$, то есть задает дисперсионную характеристику. Следовательно, для $\forall k$ дисперсионное соотношение определяют $n$ значений $\omega$: $\omega_1(k), \dots, \omega_n(k)$. 

Каждой паре $k$, $\omega_s(k)$ отвечает некоторый вектор, определяемый \eqref{eq:12}. При этом решением будут не только $k, \omega, \vec\Psi$, но и комплексно сопряжённые $k^*, \omega^*, \vec\Psi^*$. Тогда можно построить действительное решение:
\begin{equation}
	\vec{u}(x,t)=\vec\Psi e^{i(\omega t-kx)}+\vec\Psi^* e^{-i(\omega^* t-k^*x)}.
	\label{eq:14}
\end{equation}
Уравнение \eqref{eq:14} задает гармоническую волну, если $k, \omega$ действительные. Если же $k, \omega$ комплексные, то \eqref{eq:13} задает нарастающее или затухающее колебание. При этом общее решение может быть записано в виде 
\begin{equation*}
	\vec{u}(x,t)=\sum_{s=1}^{n}\qty[ \Psi^{(s)}e^{i(\omega_s(k)t-kx)} + \text{к.с}.]
\end{equation*}

Как только мы учтём граничные условия для распределенной системы, получится аналог характеристических уравнений.




\newpage
\subsection{Влияние граничных условий}

% Рассмотрим несколько примеров постановки граничных условий в разных задачах.
% 
\paragraph{Кольцо из маятников.} Пусть маятники были свернуты в кольцо длины $L$ (см. рис. \ref{fig:figg}а). В силу того, что последний маятник является первым, $\phi_N = \phi_0$.

Подставляя граничное условие в решение, которые мы ищем $\phi_n = e^{i(\omega t - nka)}$, получим выражение для $k$
\begin{equation*}
	k=\frac{2\pi m}{L}, \quad m=1,2,3,\ldots
\end{equation*}
Следовательно, на длине кольца должно уложиться целое число волн.
% \begin{gather*}
% 	n=1,\dots,N;~~\phi_{n-1}-2\phi_n+\phi_{n+1}; \\
% 	k=\frac{2\pi n}{l};~~\phi_{N+1}=\phi_1;~~k=k_n.
% \end{gather*}	
\begin{figure}[H]
	\centering
	\includegraphics[scale=1.5]{img/osci_and_wave_in_ordered_struct/ring_of_pend}
	\caption{Различные динамические системы}
	\label{fig:figg}
\end{figure}

\paragraph{Закреплённая струна. } Примером распределенной системы может служить струна длины $l$, концы которой закреплены, а колебания описываются функцией $u(x,t)$ (см. рис. \ref{fig:figg}б). Граничное условие закрепления
\begin{equation*}
	u(0,t)=u(l,t)=0
\end{equation*}
Исходя из вида решения $u(x,t)=\Psi_1 e^{i(\omega t-kx)}+\Psi_2 e^{i(\omega t+kx)}$, получим условие укладывания целого числа полуволн:
\begin{equation*}
	\begin{cases}
		\Psi_1 e^{i\omega t}+\Psi_2 e^{i\omega t}=0 \\
		\Psi_1 e^{i\omega t}+\Psi_2 e^{i\omega t+ikl}=0
	\end{cases} \quad\Rightarrow\quad 
	\Psi_1=-\Psi_2,~~ \sin{kl}=0,~~ k=\frac{\pi m}{l}=k_m
\end{equation*}
Мы на нескольких примерах наблюдали, как наложение граничных условий приводит к дискретности спектра. 

Для характеристики произвольных линейных распределённых систем записывают систему
\begin{equation}
	\begin{cases}
		D(\omega,k)=0 \\
		k=k_n
	\end{cases}
	\label{eq:15}
\end{equation}

Отсюда при подстановке конкретных формул найдётся $\omega=\omega_n$. Если среда без дисперсии, то спектр волновых чисел и спектр частот будут эквидистантны. 





\subsection{Устойчивость состояний равновесия нелинейных распределенных систем}
Простейшим типом решений распределенных систем являются такие состояния, которые не меняются ни во времени, ни в пространстве: 
\begin{equation*}
	u=u_0 = \const
\end{equation*}
Будем их называть состояниями равновесия.

Уравнение, описывающее нелинейную распределенную систему, имеет вид
\begin{equation}
	\pdv{u}{t}=f\qty(u)+D\pdv[2]{u}{x},
	\label{eq:15}
\end{equation}
где $f(u)$ - нелинейная функция. Это уравнение реакции диффузии. Если убрать коэффициент диффузии $D$, то уравнение будет описывать динамику точки в среде. Мы рассмотрим  кубическую $f(u)$:
\begin{figure}[H]
	\centering
	\includegraphics[scale=1.5]{img/osci_and_wave_in_ordered_struct/cub}
\end{figure}
В этом случае будет три точки, где $f(u)=0$. В системе могут быть стационарные решения $u=u_0(x)$ -- это другой тип решения. 

Будем изучать устойчивость решений для конкретного возмущения. Зададим возмущение в классе гармонических функций: 
\begin{equation*}
	\xi(x,t) = A e^{pt+ikx}
\end{equation*}
Линеаризуем \eqref{eq:16} на состояниях равновесия $u^*$, где $u=u^*+\xi(x,t)$:
\begin{equation}
	\pdv{\xi}{t}=D\pdv[2]{\xi}{x}+f'(u^*)\xi(x,t)
	\approx D\pdv[2]{\xi}{x}+f(u^*)+f'(u^*)\cdot\xi(x,t)
	\label{eq:16}
\end{equation}
Подставляя конкретный вид $\xi(x,t)$ и учитывая, что в состоянии равновесия $f(u^*)=0$, получим
\begin{gather*}
	p =-Dk^2+f'(u^*)
\end{gather*}


Проанализируем поведение системы в случае кубической $f(u)$, заданной ранее. В точках  $u=0$ и $u=u_2$ будет $f'(u^*)\hspace{-4pt}<\hspace{-4pt}0$, затухание возмущений. Значит, эти точки устойчивы. 



% Для каждого А.
\begin{figure}[H]
	\centering
	\includegraphics[scale=1.5]{img/osci_and_wave_in_ordered_struct/stability_or_instability} 
\end{figure}

В точке же $u_1$  производная $f'(u_1)>0$, значит, решение неустойчивое в классе возмущений $\xi(x,t)=A e^{pt+ikx}$.

Уравнение \eqref{eq:16}, в частности, описывает распространение пламени по бикфордовому (огнепроводному) шнуру. Волна превращает вещество из несгоревшего материала в сгоревшее.

\newpage
\section{Диффузионная неустойчивость. Структура Тьюринга}
\label{sec:diffusion_instability}
	% 15.04.19
	%!TEX root = ../lections.tex

Многие биологические объекты обладают периодической структурой. Например, зебра, дождевой червь, сороконожка. Однако известно, что изначально, при появлении, они были однородны. Мы будем рассмотривать случай одномерных структур.

Тьюринг предположил, что существуют два вещества. Одно из них -- $U(x,t)$ -- стимулирует рост клеток. Его назвали активатором. Другое -- $V(x,t)$ -- замедляет рост. Это ингибитор. Следующим предположением было наличие некоторых химических реакций и присутствие диффузии. Фактически, Тьюринг записал уравнение реакции диффузии
\begin{equation}
\left\{\begin{aligned}
	\pdv{U}{t}=f(U,V)+D_1 \pdv[2]{U}{x} \\
	\pdv{V}{t}=g(U,V)+D_2 \pdv[2]{V}{x},
\end{aligned}\right.
	\label{eq:s2:1}
\end{equation}
где $f,g$  - это некоторые нелинейные функции.

Эта система двухкомпонентная, произвольная координата одна. Предположим, \eqref{eq:s2:1} имеет состояние равновесия, т.е. существует решение системы
\begin{equation}
\left\{\begin{aligned}
		f(U,V)=0 \\
		g(U,V)=0
	\end{aligned}\right.	\quad \Rightarrow \quad U=U_0, V=V_0
	\label{eq:s2:2}
\end{equation}
Исследуем уравнение \eqref{eq:s2:1} на устойчивость. Будем подставлять решение, возмущённое около равновесия
\begin{gather*}
	U=U_0+\xi(x,t) \\
	V=V_0+\eta(x,t).
\end{gather*}
Линеаризуем \eqref{eq:s2:1}, подставив туда возмущенное решение, разложив $f, g$ в ряд Тейлора и оставив у них только линейную часть:
\begin{equation}
	\left\{\begin{aligned}
		\pdv{\xi}{t}=f'_u(U_0, V_0)\xi(x,t)+f'_v(U_0, V_0)\eta(x,t)+D_1 \pdv[2]{\xi(x,t)}{x} \\
		\pdv{\eta}{t}=g'_u(U_0, V_0)\xi(x,t)+g'_v(U_0, V_0)\eta(x,t)+D_2 \pdv[2]{\eta(x,t)}{x}.
	\end{aligned}\right.
	\label{eq:s2:3}
\end{equation}

Получили уравнения в частных производных. Будем искать решение в виде
\begin{equation}
	\xi(x,t)=A e^{pt+ikx}, \qquad \eta(x,t)=B e^{pt+ikx}.
	\label{eq:s2:4}
\end{equation}
Сначала перепишем  \eqref{eq:s2:3}, учитывая, что  производные $f'_u(U_0,V_0)$, $f'_v(U_0,V_0)$, а также и $g'_u(U_0,V_0)$, $g'_v(U_0,V_0)$ -- производные в точках (константы) и обозначив их $a$, $b$, $c$, $d$: 
\begin{equation}
	\left\{\begin{aligned}
		&\pdv{\xi}{t}=a\xi+b\eta+D_1 \pdv[2]{\xi}{x} \\
		&\pdv{\eta}{t}=c\xi+d\eta+D_2 \pdv[2]{\eta}{x},
	\end{aligned}\right.
	\label{eq:s2:5}
\end{equation}
Подставляя решение в виде \eqref{eq:s2:4} в \eqref{eq:s2:5}, получим
% где
% \begin{gather*}
% 	a=f'_u(U_0, V_0),~~b=f'_v(U_0, V_0),~~c=g'_u(U_0, V_0),~~d=g'_v(U_0, V_0).
% \end{gather*}
\begin{equation}
	\left\{\begin{aligned}
		&Ap=aA+bB-k^2 D_1A\\
		&Bp=cA+dB-k^2 D_2A,
	\end{aligned}\right.
	\label{eq:s2:6}
\end{equation}

Система \eqref{eq:s2:6} представляет собой систему линейных однородных уравнений относительно констант $A$ и  $B$. Она имеет нетривиальное решение, если ее определитель не равен нулю. Раскрывая определитель, получим характеристическое уравнение для $p$:
\begin{equation}
	p^2+(D_2k^2+D_1k^2-a-d)p+(D_1k^2-a)(D_2k^2-d)-bc=0.
	\label{eq:s2:7}
\end{equation}
Пусть в начальный момент диффузии нет, $D_1=D_2=0$:
\begin{equation}
	p^2-(a+d)p+ad-bc=0.
	\label{eq:s2:8}
\end{equation}
\begin{figure}[H]
	\centering
	\includegraphics[scale=1.5]{img/diffusion_instability/static_u} 
	\caption{Устойчивая в отсутствии диффузии система}
\end{figure}
Предположим, что без диффузии система устойчива ($\Re p < 0$). Условие для этого
\begin{gather}
	\Delta_0=ad-bc>0, \qquad \sigma_0=a+d<0,
	\label{eq:s2:9}
\end{gather}
В этом (устойчивом) случае по оси $x$ активатор $U$ и ингибитор $V$ постоянны. 

Пусть теперь $D_1, D_2 \neq 0$. В этом случае \eqref{eq:s2:7} после преобразований примет вид:
\begin{gather*}
	p^2-\sigma p+\Delta=0, \quad \text{где}\\
	\Delta=a+d-(D_1+D_2)k^2=\sigma_0-(D_1+D_2)k^2, \\ \sigma=ad-bc-(aD_2+dD_1)k^2+D_1D_2k^4=\Delta_0-(aD_2+dD_1)k^2+D_1D_2k^4.
\end{gather*}
Несложный анализ показывает следующую зависимость $\Re p$  от $k^2$:
\begin{figure}[h]
\begin{minipage}[h]{0.49\linewidth}
\center{\includegraphics[scale=1.5]{img/diffusion_instability/re_p}}
\end{minipage}
\hfill
\begin{minipage}[h]{0.49\linewidth}
\center{\includegraphics[scale=1.5]{img/diffusion_instability/u_x}}
\end{minipage}
	\caption{Зависимость $\Re p$ от $k^2$ при $D_{1,2}\ne0$ и структура Тьюринга (справа)}
\end{figure}

Есть диапазон, где произошла потеря устойчивости состояния равновесия за счет действия диффузии. Такую потерю устойчивости называют \textbf{диффузионной неустойчивостью}.При этом оказывается, что возникает аналог бифуркации Андронова-Хопфа. 

Возникает периодическое по пространству (в нашем случае -- по $x$) решение, и это решение пространственно-неоднородное и переодическое (структура, паттерн). Оказыватся, этот паттерн структурно устойчив.

Динамическая структура, обладающая свойством структурной устойчивости, иногда называется диссипативной структурой. Она возникает за счет баланса активатора и ингибитора. 

После открытия Тьюрингом возможности существования таких паттернов за счет активатора и ингибитора, были реально обнаружены в природе такие вещества.



\subsection{Простые волны. Образование разрывов.}
Рассмотрим однородную линейную среду без дисперсии, свойства которой описывает скалярная функция $U(x,t)$. В таких средах возможны волновые движения, при этом все переменные описываются одинаковыми уравнениями
\begin{equation}
	\pdv{U_j}{t}+V\pdv{U_j}{x}=0,
	\label{eq:s2:10}
\end{equation}
где $V$ -- константа. В среде нет собственных масштабов и нелинейностей. В этой среде возможно существование так называемых римановых волн
\begin{equation}
	U_j=\phi(x-Vt),
	\label{eq:s2:11}
\end{equation}
где $\phi$ - произвольная, обязательно дифференцируемая, функция.

Проверим прямой подстановкой. Введем бегущую координату $\xi=x-Vt$ и подставим \eqref{eq:s2:11} в \eqref{eq:s2:10}:
\begin{gather*}
	\dv{\phi}{\xi} \pdv{\xi}{t}+V\dv{\phi}{\xi} \pdv{\xi}{x}=0
	\quad\Rightarrow\quad
	-V\dv{\phi}{\xi}+V \dv{\phi}{\xi} \equiv 0.
\end{gather*}
Значит, \eqref{eq:s2:11} действительно является решением такой системы. 

Теперь рассмотрим нелинейную среду без дисперсии. Оказывается, в таких средах могут существовать волны, которые сохраняют одно свойство римановых:  переменные связаны алгебраически (если есть $U_1$ и $U_2$, то $U_3$ всегда можно пересчитать). 



\newpage
\paragraph{Волны на мелкой воде. } Типичный пример таких волн -- волны на мелкой воде, $\lambda \gg h_0$ (см. рис. ниже). Это гравитационные волны, бегущие вдоль оси $x$. 
Состояние жидкости описывается скоростью волны $V$  и профилем $h(z)$. Поскольку волны длинные, считаем, что $V$ не зависит от $z$.

\begin{figure}[H]
	\centering
	\includegraphics[scale=1.5]{img/diffusion_instability/long_waves}
\end{figure}

Запишем уравнение Эйлера. Жидкость несжимаемая ($\rho=\const$): 
\begin{gather*}
	\pdv{V}{t}+V\pdv{V}{x}+\frac1{\rho}\pdv{p}{x}=0
	% , \\ \rho=const, p-\text{среднее по высоте}, \\ \.
\end{gather*}
Давление больше там, где выше жидкость: $\frac1{\rho}\pdv{p}{x}=g\pdv{h}{x}$, тогда
\begin{equation}
	\pdv{V}{t}+V\pdv{V}{x}+g\pdv{h}{x}=0.
	\label{eq:s2:12}
\end{equation}
Второе уравнение -- уравнение непрерывности (в терминах потоков через близкие сечения скорость изменения высоты слоя связана с разностью потока через $x$ и $\dd x$):
\begin{equation}
	\pdv{h}{t}+\pdv{(Vh)}{x}=0. 
	\quad \Rightarrow \quad
	\pdv{h}{t}+V\pdv{h}{x}+h\pdv{V}{x}=0.
	\label{eq:s2:13}
\end{equation}

Предположим, что $h$ и $V$ не являются независимыми переменными и $h=h(V)$, тогда $\pdv{h}{x}=\dv{h}{V}\pdv{V}{x}$ и из уравнения \eqref{eq:s2:12} будет следовать, что 
\begin{gather*}
	\pdv{V}{t}+V\pdv{V}{x}+g\pdv{h}{x} = 0.
	 % ~(*)
	 \tag{*}
\end{gather*}
Аналогично подставим в уравнение \eqref{eq:s2:13}, тогда
\begin{gather*}
	\dv{h}{V}\pdv{V}{t}+V\dv{h}{V}\pdv{V}{x}+h(V)\pdv{V}{x} = 0
	\quad \Rightarrow \quad
	\pdv{V}{t}+V\pdv{V}{x}+\frac{h(V)}{\dv{h}{V}}\pdv{h}{x} = 0.
	\tag{**}
\end{gather*}
Уравнения (*) и (**) должны совпадать, так как описывают одну и ту же величину. Значит, нужно прировнять коэффициенты при производных, и тогда получим уравнение для нахождения $h$:
\begin{equation*}
	g\dv{h}{V}=\frac{h}{\dv{h}{V}}
	\quad \Rightarrow \quad
	\dv{h}{V}=\pm \sqrt{\frac{h(V)}{g}}
\end{equation*}
Подставив полученное выражение в (**), получим уравнение простой волны
\begin{equation}
	\pdv{V}{t}+V\pdv{V}{x}\pm \sqrt{h(V)g\,} \pdv{V}{x} = 0
	\quad \Rightarrow \quad
	\pdv{V}{t}+\qty(V\pm \sqrt{h(V)g}\,)\pdv{V}{x} = 0.
	\label{eq:s2:14}
\end{equation}

\paragraph{Уравнение простой волны. } В общем виде это уравнение справедливо не только на мелкой воде, и выглядит следующим образом:
\begin{equation}
	\pdv{U}{t}+V(U)\pdv{U}{x} = 0
	\label{eq:s2:15}
\end{equation}
Здесь $V$ -- это функция среды. 

Давайте исследуем уравнение \eqref{eq:s2:15}. Будем искать решение в виде
\begin{equation}
	U = \phi(\xi), \quad \text{где}\quad
	\xi=x-V(U)t.
	\label{eq:s2:16}
\end{equation}
Убедимся, что \eqref{eq:s2:16} - решение. Найдём производные в силу \eqref{eq:s2:16}:
\begin{gather}
	\nonumber\pdv{U}{t}=\dv{\phi}{\xi}\pdv{\xi}{t}=\dv{\phi}{\xi}\qty(-V(U)-t\dv{V}{U}\pdv{U}{t})
	\,\Rightarrow \\ \Rightarrow 
	\pdv{U}{t}\qty(1+\dv{\phi}{\xi}\dv{V}{U}t)=-V(U)\dv{\phi}{\xi}
	\vspace{-0.5em}\quad \Rightarrow \vspace{-0.5em}\quad
	\pdv{U}{t}=-V(U)\frac{\dv{\phi}{\xi}}{\qty(1+\dv{\phi}{\xi}\dv{V}{U}t)}
	\label{eq:s2:6}
\end{gather}
Аналогичным образом получается
\begin{gather*}
	\pdv{U}{x}=\dv{\phi}{\xi} \frac{1}{\qty(1+\dv{\phi}{\xi}\dv{V}{U}t)}.
\end{gather*}
Действительно, подставляя полученные выражения в \eqref{eq:s2:15}, получим тождество.

Исследуем решение \eqref{eq:s2:16}. Для определенности зададим вид нелинейности $V(U)$ и профиль волны в начальный момент времени $\phi(x,t=0)$\footnote{Есть два способа исследования: задать начальное распределение по $x$ и смотреть, что с ним будет происходить во времени, или наоборот -- создать на границе постоянные колебания и смотреть, что с ними произойдет при распространении вдоль $x$} (см. рис. \ref{fig:uvux}).

\begin{figure}[H]
\begin{minipage}[h]{0.49\linewidth}
\center{\includegraphics[scale=1.4]{img/diffusion_instability/vu}}
\end{minipage}
\hfill
\begin{minipage}[h]{0.49\linewidth}
\center{\includegraphics[scale=1.4]{img/diffusion_instability/ux}}
\end{minipage}
	\caption{}
	\label{fig:uvux}
\end{figure}

% У максимального значения U скорость наибольшая. Задается профиль $\phi$ при $t=0$:
% \begin{figure}[H]
% 	\centering
% 	\includegraphics[width=0.4\linewidth]{fig/fig13.pdf}   
% \end{figure}
% или при $x=0$:
% \begin{figure}[H]
% 	\centering
% 	\includegraphics[width=0.4\linewidth]{fig/fig14.pdf}   
% \end{figure}
Во время распространения гребень волны будет обгонять подошву волны, будет просходить укручение фронта: волна деформируется.
\begin{figure}[H]
	\centering
	\includegraphics[scale=1.4]{img/diffusion_instability/ux2.pdf}
\end{figure}
% \begin{figure}[H]
% 	\centering
% 	\includegraphics[width=0.4\linewidth]{fig/fig15.pdf}   
% \end{figure}
% \begin{figure}[H]
% 	\centering
% 	\includegraphics[width=0.4\linewidth]{fig/fig16.pdf}   
% \end{figure}
% \begin{figure}[H]
% 	\centering
% 	\includegraphics[width=0.4\linewidth]{fig/fig17.pdf}   
% \end{figure}
В момент времени
\begin{equation}
	1+\dv{\phi}{\xi}\dv{V}{U} t = 0
\end{equation}
появится точка, где производные $\pdv{U}{t}, \pdv{U}{x} \rightarrow \infty $. Образуется бесконечный градиент и разрыв, который характеризуется состоянием $U^*, t^*, x^*$. Дальнейшее описание в рамках простой волны невозможно. Такой момент в решении называют \textbf{градиентной катастрофой}. %Вторые производные тоже обращаются в бесконечность, и точки эти можно найти.



\newpage
\section{Стационарные волны в консервативных и автоколебательных распределенных системах}
\label{sec:volnu}
	\subsection{Солитоны в уравнении Кортевега -- де Вриза}
	\label{ssec:soliton}
		%!TEX root = lections.tex
Рассмотрим узкий бассейн с жидкостью (должен быть достаточно длинным):
\begin{figure}[H]
	\centering
	\includegraphics[width=0.4\linewidth]{fig/fig18.pdf}   
\end{figure}

Подбирают соотношение масс воды в емкостях. Резко поднимают задвижку. В зависимости от соотношения масс воды, могут побежать разные волны, разное количество холмов. 

Канал, который мы рассматриваем, мелкий, средней глубины $l$; $l+\eta(x,t)$. Поверим на слово: 
\begin{equation*}
	\pdv{\eta}{t}=\frac32 \sqrt{\frac{g}{l}}\pdv{}{x}\qty(\frac32 \alpha \eta+\frac12 \eta^2+\frac13 \sigma \pdv[2]{\eta}{x}),
\end{equation*}
где $\sigma=\frac{l^3}{3}-\frac{Tl}{\rho g}$, $\alpha$-произвольная константа, $T$ - коэффициент поверхностного натяжения, $\rho$ - плотность жидкости.

Преобразовав, получим
\begin{equation}
	\pdv{U}{t}+U\pdv{U}{x}+\beta \pdv[3]{U}{x}=0.
	\label{eq:33}
\end{equation}
$\beta$ - некоторая константа, а последнее слагаемое характеризует дисперсию. Видно, что уравнение простой волны дополняется дисперсией. (К экзамену построить дисперсионную характеристику). 

Пусть есть квадратичная среда, в нее запускают волну $e^{i(\omega_o t+kx)}$. Среда нелинейная и без дисперсии. При таких условиях квадратичная среда порождает новые частоты. $\omega_o$ порождает $2\omega_o$. Если нет дисперсии, то нет пространственных масштабов. $2\omega_o$ порождает $3\omega_o$ и так далее, лавинно. Спектр стремится в бесконечность. 
\begin{figure}[H]
	\centering
	\includegraphics[width=0.4\linewidth]{fig/fig4.pdf}   
\end{figure}

Теперь включим дисперсию (в данном случае - высокочастотную). Она ограничивает частотный рост и стабилизирует волну. 

Будем искать решение уравнения \eqref{eq:33} в виде $U=U(x-Vt)$:
\begin{equation}
	-\pdv{U}{\xi}+U\pdv{U}{\xi}+\beta \pdv[3]{U}{\xi}=0.
	\label{eq:34}
\end{equation}

Интегрируя:
\begin{equation}
	\beta \pdv[2]{U}{\xi}+\frac{U^2}{2}-VU=0.
	\label{eq:35}
\end{equation}

Это есть уравнение нелинейного осциллятора. Для простоты константу интегрирования приравняли  к нулю, что дает уровень, откуда изменяется $U(x,t)$.
\begin{equation}
	\begin{cases}
		\dot{U}=y \\
		\beta \dot{y} =VU-\frac{U^2}{2}.		
	\end{cases}
	\label{eq:36}
\end{equation}

Получили систему для нелинейного осциллятора.
\begin{gather*}
	\beta \frac{y^2}{2}+E_{\text{п}} = const, \\ E_{\text{п}} = \frac{U^3}{6}-V\frac{U^2}{2}
\end{gather*}
\begin{figure}[H]
	\centering
	\includegraphics[width=0.4\linewidth]{fig/fig19.pdf}   
\end{figure}

Фазовый портрет:
\begin{figure}[H]
	\centering
	\includegraphics[width=0.5\linewidth]{fig/fig20.pdf}   
\end{figure}
Солитоном и является эта, покрашенная в восхитительный purple, гомоклиническая орбита.

22.04

Напомню, что дифференцирование в системе \eqref{eq:36} производится по бегущей координате. 

Вернемся к фазовому портрету. Неограниченные траектории лишены физического смысла, поэтому нас интересуют только ограниченные, то есть те, что находятся внутри петли сепаратрис, включая ее саму.

Пусть 1 - траектория вблизи состояния равновесия центр. Она замкнутая, колебание близко к гармоническому. Колебание происходит на фоне колебаний 2V. Таких колебаний континуум. 
\begin{figure}[H]
	\centering
	\includegraphics[width=0.4\linewidth]{fig/fig21.pdf}   
\end{figure}

Выберем 2 - вблизи седла, но все еще внутри гомоклинической орбиты. Выберем точку и положим при $\xi=0$ амплитуда максимальна. При движении по траектории к седлу (зеленым), U убывает. 
\begin{figure}[H]
	\centering
	\includegraphics[width=0.4\linewidth]{fig/fig22.pdf}   
\end{figure}
Около самого состояния равновесия скорость мала, движение в ее окрестности будет проходить медленно. В конце концов, при выходе из окрестности состояния равновесия, скорость опять увеличится. Мы получим профиль, так называемой, кноидальной волны, которая далека от гармонической. Волны такие, потому что система нелинейная. 
Если брать траектории все ближе к седлу, полка будет увеличиваться. В конце концов, когда попадем на петлю:

Для того, чтобы нарисовать профиль в область $\xi<0$, нужно пройти по траектории в обратную сторону (синим). Рисунок качественный, нарисован из понимания поведения функции на концах и зная максимальную амплитуду. 
\begin{figure}[H]
	\centering
	\includegraphics[width=0.4\linewidth]{fig/fig23.pdf}   
\end{figure}
Получившийся одиночный холм - это математический образ солитона. Точка 0 - решение, там нет пересечения траекторий, они стремятся туда асимптотически. 

Запишем наше уравнение в виде:

\begin{equation}
	\beta\dv[2]{U}{\xi}=VU-\frac{U^2}{2}
	\label{eq:37}
\end{equation}

Решение будем искать в виде:
\begin{equation}
	U=\frac{U_{max}}{\ch^2{(\xi / \Delta)}}
	\label{eq:38}
\end{equation}

Введенные параметры: $U_{max}, \Delta, V$ (V скрыто в $\xi=x-Vt$).
\begin{equation}
	\dv[2]{U}{\xi}=-2\frac{U_m}{\Delta^2}\frac{(3-2\ch^2{\xi / \Delta)}}{\ch^4{(\xi / \Delta)}}
	\label{eq:39}
\end{equation}
\begin{gather*}
	\frac{2\beta U_m}{\Delta^2}\frac{(3-2\ch^2{(\xi / \Delta))}}{\ch^4{(\xi / \Delta)}}=\frac{U_m V}{\ch^2{(\xi / \Delta)}}-\frac{U_m^2}{2\ch^4{(\xi / \Delta)}}.
\end{gather*}

Знаменатель не обращается в ноль.
\begin{gather*}
	-2\beta[3-2\ch^2{(\xi / \Delta)}]=\Delta^2 \ch^2{(\xi / \Delta)}V-\frac{U_m \Delta^2}{2}, \\
	-6\beta=\frac{U_m \Delta^2}{2}, \\
	4\beta=\Delta^2V.
\end{gather*}
\begin{equation}
	U_m \Delta^2= 12\beta,
	\label{eq:40}
\end{equation}
\begin{equation}
	\Delta^2V=4\beta.
	\label{eq:41}
\end{equation}
Параметры связаны, но их 3, а условия 2. Задав один, два других найдутся из \eqref{eq:40} и \eqref{eq:41}. $\beta$ характеризует дисперсию, не относится к самому солитону. V - скорость солитона, $U_m$ - его высота, $\Delta$ оказывается шириной. Ширина солитона вычисляется на уровне $U=\frac{4U_m}{2+e}$. 

Из уравнения \eqref{eq:41} следует, что, чем шире солитон, тем меньше его скорость: $V=\frac{4\beta}{\Delta^2}$. Из \eqref{eq:40}, что тогда и меньше амплитуда: $U_m=\frac{12\beta}{\Delta^2}$.

Если в диссипативной системе есть петля, то при изменении параметров она разрушается.Есть всего одна траектория, формирующая солитон. Солитоны устойчивы относительно большого числа начальных распределений. 

\subsubsection{Устойчивость солитона}
Замечание: рассмотрим уравнение Шредингера для определения статистического состояния. 
\begin{equation}
	\dv[2]{\psi}{x}+[U(x)+\epsilon]\psi=0.
	\label{eq:42}
\end{equation}

$U(x)>0, ~U\rightarrow 0$ при $x\rightarrow \pm \infty$

Есть решение, когда спектр дискретный: $\epsilon=\epsilon_n, \psi, \psi' \rightarrow 0$ при $x\rightarrow \pm \infty$.\

Существует связь между \eqref{eq:41} и устойчивостью солитонов уравнения Кортевега - де Фриза. Надо подставить нормированное решение:
\begin{equation*}
	\dv[2]{\psi}{x}+[\frac{1}{6\beta}U(x,t)+\epsilon]\psi=0.
\end{equation*}

Покажем, что, если речь о солитоне, то $\epsilon$ не будет зависеть от t.
\begin{equation*}
	U(x,t)=-6\beta(\frac{\psi''}{\psi}+\epsilon),
\end{equation*}
здесь $'$ - дифференцирование по x. Подставим в уравнение Кортевега - де Фриза:
\begin{equation*}
	\pdv{U}{t}+U\pdv{U}{x}+\beta \pdv[3]{U}{x}=0.
\end{equation*}
\begin{equation}
	\psi^2\dv{\epsilon}{t}=(\psi'A-A\psi').
	\label{eq:43}
\end{equation}
\begin{equation*}
	A(x,t)=6\beta(\frac1{\beta}\pdv{\psi}{t}-3\frac{\psi'\psi''}{\psi}+\psi'''-\frac{\epsilon}{6}\psi').
\end{equation*}

Проинтегрируем \eqref{eq:42} по переменной x в бесконечных пределах. Вспомним, что $\psi, \psi' \rightarrow 0$ при $x\rightarrow \pm \infty$. Получим:
\begin{equation*}
	\dv{\epsilon}{t}\int^{+\infty}_{-\infty}\psi^2dx=0.
\end{equation*}

В силу нормировки интеграл не равен нулю. Следовательно, $\dv{\epsilon}{t}=0$ и $\epsilon\neq \epsilon(t)$.
\begin{equation*}
	U(x,t)=U_{max}c\ch^{-2}(\frac{x-Vt}{\Delta}).
\end{equation*}

Здесь t можно выбрать любое. Положим, $t=0$.
\begin{equation}
	\psi''+(U_0 \ch^{-2}\alpha x+\epsilon)\psi=0.
	\label{eq:44}
\end{equation}
\begin{equation*}
	U_0=\frac{U_m}{6\beta},~\alpha=\frac1{\Delta}.
\end{equation*}

Пользуясь случаем, передаем приветы и спасибо третьему тому Ландау-Лившица, параграф 23, задача 4, где это уравнение решено. Спектр:
\begin{gather*}
	\epsilon_n=-\alpha(s-n), n=0,1,2,\dots;~n<s \\ s=\frac12(-1+\sqrt{1+\frac{4U_0}{\alpha^2}})=\frac12(-1+\sqrt{\frac{4U_m\Delta^2}{6\beta}})=\frac12(-1+3)=1 \\ \epsilon_n=-\alpha(1-n) \\ n=0:~ \epsilon_0=-\alpha^2=-\frac{4U_m}{12\beta}.
\end{gather*}

Если в уравнение Шредингера подставить уравнение солитона, такому потенциалу соответствует одно собственное значение. Пусть в начальный момент есть распределение (положительно определенное, но не совпадающее с солитоном). Подставляя его в уравнения Шредингера, решив, получим столько $\epsilon_j$, сколько солитонов может существовать при таких начальных условиях.
\begin{figure}[H]
	\centering
	\includegraphics[width=0.4\linewidth]{fig/fig25.pdf}   
\end{figure}

Пусть $j=3$:
\begin{figure}[H]
	\centering
	\includegraphics[width=0.4\linewidth]{fig/fig26.pdf}   
\end{figure}

Это обратная задача рассеяния. Качественные рассуждения: введем величину (из \eqref{eq:40}) $\sigma=\frac{\Delta^2 U_m}{12 \beta}=1$. Здесь $U_m$ характеризует нелинейность, а $\beta$ - дисперсию. Предположим, что при $t=0$ задали такое распределение, что $\sigma \ll 1$. Это означает малость $U_m$, мы находимся около стационарного состояния (около нуля). Если мы близки к линейной задаче, главную роль играют дисперсионные механизмы. Есть области прозрачности и непрозрачности. В области прозрачности фазовая скорость у каждой компоненты своя, пакет расплывается.  Приходим к единственному солитону.

Если  $\sigma \gg 1$, преобладает нелинейность. Она порождает новые гармоники. Переход к образованию состава из солитонов. Дисперсия ограничивает частоты, фронт стабилизируется.


	\subsection{Стационарные ударные волны}
	\label{ssec:shock_waves}
		%!TEX root = lections.tex
Рассмотрим
\begin{equation}
	\pdv{U}{t}+U\pdv{U}{x}+\beta \pdv[3]{U}{x}-\nu\pdv[2]{U}{x}=0.
	\label{eq:45}
\end{equation}

$\nu>0$ - диссипация. К чему приведет ее учет?

Если $\beta=0$, то уравнение называется уравнением Бюргерса. 

$\xi=x-Vt$ - называют стационарным решением, т.к. профиль не меняется. 
\begin{equation*}
	-V\dv{U}{\xi}+U\dv{U}{\xi}+\beta\dv[3]{U}{\xi}-\nu\dv[2]{U}{\xi}=0.
\end{equation*}

Проинтегрируем:
\begin{equation*}
	\beta\dv[2]{U}{\xi}-\nu \dv{U}{\xi}+\frac{U^2}{2}-VU=0.
\end{equation*}
($\cdot$ - дифференцирование по $\xi$)
\begin{equation}
	\begin{cases}
		\dot{U}=y \\
		\beta \dot{y} =\nu y-\frac{U^2}{2}+VU.		
	\end{cases}
	\label{eq:46}
\end{equation}

Система определена на фазовой плоскости. Состояния равновесия: $O_1(0,0)$, $O_2(2V,0)$. Линеаризуем в этих точках. $O_1$ - седло, а для $O_2$:
\begin{gather*}
	p^2-\frac{\nu}{\beta}p+\frac{V}{\beta}=0, \\ p_{1,2}=\frac{\nu}{2\beta}\pm \sqrt{\frac{\nu^2}{4\beta^2}-\frac{V}{\beta}}.
\end{gather*}

Если $\frac{\nu^2}{4\beta^2}-\frac{V}{\beta}<0$, то имеем неустойчивый фокус.
\begin{equation*}
	\frac{\nu^2}{4\beta}=V \rightarrow \beta=\frac{\nu^2}{4V}.
\end{equation*}
\begin{figure}[H]
	\centering
	\includegraphics[width=0.4\linewidth]{fig/fig27.pdf}   
\end{figure}

Возьмем полную энергию при $\nu=0$:
\begin{equation*}
	V(U,y)=\beta\frac{y^2}{2}+\frac{U^3}{6}-V\frac{U^2}{2},
\end{equation*}

\begin{figure}[H]
	\centering
	\includegraphics[width=0.4\linewidth]{fig/fig30.pdf}   
\end{figure}

\begin{equation*}
	\dot{V}=\beta y \dot{y}+\frac{U^2}{2}\dot{U}-VU\dot{U}=\nu y^2-\frac{U^2}{2}y+VUy-VUy+\frac{U^2}{2}y=\nu y^2\geqslant 0,
\end{equation*}
т.е. траектории системы пересекают линии уровня в сторону возрастания. Из существования этой функции следует, что система не имеет предельных циклов. 

Как ведут себя устойчивые сепаратрисы седла? Ограниченное решение существует, оно единственно и соответствует красной траектории. Зафиксируем $\beta$:
\begin{figure}[H]
	\centering
	\includegraphics[width=0.4\linewidth]{fig/fig28.pdf}   
\end{figure}

Нарисуем профиль этой волны (для траектории, котора успеет сделать много витков, до того, как придет в состояние равновесия) ($\nu\ll1$):
\begin{figure}[H]
	\centering
	\includegraphics[width=0.6\linewidth]{fig/fig29.pdf}   
\end{figure}

Чем дальше от седла, тем ближе точки, выше и меньше полки. Это ударная волна.  Среда была в покое, прошел фронт и перебросил среду в новое состояние $2V$.

Для красной траектории:
\begin{figure}[H]
	\centering
	\includegraphics[width=0.6\linewidth]{fig/fig31.pdf}   
\end{figure}

Для траектории, идущей из неустойчивого состояния равновесия (фокуса или узла) в седло:
\begin{figure}[H]
	\centering
	\includegraphics[width=0.6\linewidth]{fig/fig32.pdf}   
\end{figure}

Здесь не будет осцилляций. Во всех случаях есть передний фронт.


\newpage
\section{Параметрические колебания}
\label{sec:parametric_oscillations}
% 29.04.19 
	%!TEX root = ../lections.tex
Ранее мы рассматривали системы, в которых под действием внешних силовых воздействий появлялись новые режимы. Они являлись неавтономными. Кроме них существует другой тип неавтономных систем и внешнего воздействия. Внешнее воздействие находится внутри системы и может изменять его параметры.
Такие системы -- параметрические, а их колебания - параметрические колебания. Пример системы -- маятник, длина которого периодически меняется, т.е. параметр длины меняется со временем. 

Параметрические системы делят на 2 важных класса: \textbf{резонансные} и \textbf{нерезонансные}. В резонансных системах период изменения параметров находится в целочисленном соотношении с периодом собственных колебаний. В таких системах в такт с изменением энергии, соответствующей собственным колебаниям, вносится энергия, вызванная работой внешнего воздействия. При определенных условиях это может приводить к эффекту раскачки колебаний за счет накапливающейся в системе энергии. Этот эффект лежит в основе работы параметрических усилителей и генераторов. 

К нерезонансным неколебательным параметрическим системам относятся системы, в которых параметры изменяются очень быстро или очень медленно по сравнению с характерным временными масштабами изменения переменных системы. Динамика линейных параметрических систем описывается системой линейных дифференциальных уравнений с периодическими коэффициентами. 

Далее будем рассматривать только линейные параметрические системы. В основе теории таких систем лежит так называемая \textbf{теория Флоке}. 

Рассмотрим двумерную систему
\begin{equation}
	\left\{\begin{aligned}
		&\dot{x_1}=p_{11}(t)\cdot x_1+p_{12}(t)\cdot x_2 \\
		&\dot{x_2}=p_{21}(t)\cdot x_1+p_{22}(t)\cdot x_2,	
	\end{aligned}\right.
	\label{eq:47}
\end{equation}
где $p_{jk}(t+T)=p_{jk}(t)$, то есть  коэффициенты периодически изменяются во времени. Запишем общее решение в матричном виде:
\begin{equation*}
	\vec{x} = X \vec{C}, 
\end{equation*}
где
\begin{gather*}
	\vec{x}= 
	\begin{pmatrix}
		x_1 \\
		x_2
	\end{pmatrix}
	,\quad
	\vec{C}= 
	\begin{pmatrix}
		c_1 \\
		c_2
	\end{pmatrix}
	,\quad
	\vec{X}(t)= 
	\begin{pmatrix}
		\phi_1 ~~  \psi_1 \\
		\phi_2  ~~ \psi_2
	\end{pmatrix}.
\end{gather*}
Здесь векторы
\begin{gather*}
	\vec{\phi}= 
	\begin{pmatrix}
		\phi_1 \\
		\phi_2
	\end{pmatrix}
	,\quad
	\vec{\psi}= 
	\begin{pmatrix}
		\psi_1 \\
		\psi_2
	\end{pmatrix}
\end{gather*}
являются линейно независимыми, следовательно, образуют фундаментальную систему решений. 

Покажем, что в качестве $\phi,\psi$ можно выбрать такие функции, которые удовлетворяют начальным условиям:
\begin{gather}
	\phi_1(0)=1,\quad \psi_1(0)=0, \notag \\ 
	\phi_2(0)=0,\quad \psi_2(0)=1.		
	\label{eq:48}
\end{gather}

Согласно общей теории линейных дифференциальных уравнений вектора $\phi$, $\psi$  будут линейно независимы, если определитель Вронского не обращается в ноль:
\begin{gather*}
	W(t)= 
	\begin{vmatrix}
		\phi_1(t) ~~\psi_1(t) \\ 
		\phi_2(t) ~~\psi_2(t)
	\end{vmatrix}
	,
\end{gather*}
Для ограниченных $W$ верна теорема Лиувилля:
\begin{equation}
	W(t)=W(0)\exp\qty[\int_0^t(p_{11}(t)+p_{22}(t))\dd{t}].
	\label{eq:49}	
\end{equation}

Для начальных условий \eqref{eq:48} получается $W(0)=1$, отсюда следует, что вронскиан отличен от нуля, значит $\phi$, $\psi$ линейно независимы и образуют ФСР.

Поскольку $p_{jk}(t)$ являются периодическими, то вектора $\vec{\phi}(t+T)$, $\vec{\psi}(t+T)$ также линейно не зависимы и являются решением системы \eqref{eq:47}. Как всякое решение, эти вектора можно выразить через ФСР:
\begin{gather}
	\left\{\begin{aligned}
	\vec{\phi}(t+T)=a\vec{\phi}(t)+b\vec{\psi}(t), \\ 
	\vec{\psi}(t+T)=c\vec{\phi}(t)+d\vec{\psi}(t),			
	\end{aligned}\right.
	\label{eq:50}
\end{gather}
где $a,b,c,d$ -- некоторые константы. Их можно найти, подставив $t=0$ и учтя начальные условия \eqref{eq:48}:
\begin{equation}
	a=\phi_1(T),~ b=\phi_2(T),~ c=\psi_1(T),~ d=\psi_2(T).
	\label{eq:51}	
\end{equation}

Соотношения \eqref{eq:51} говорят о том, что константы могут быть найдены, если известно общее решение $\vec\phi, \vec\psi$.

Покажем, что существуют такие константы $a,b,c,d$, что для системы  \eqref{eq:47} справедливо:
\begin{equation}
	\vec{x}(t+T)=s\vec{x}(t),
	\label{eq:52}	
\end{equation}
т.е. через период решение повторяется с точностью до множителя.

Поскольку любое решение системы \eqref{eq:47} можно получить из общего решения надлежащим выбором констант, то
\begin{equation}
	\vec{x}(t)=A\vec{\phi}(t)+B\vec{\psi}(t),
	\label{eq:53}	
\end{equation}
% 
% Запишем состояние этого вектора в момент времени $t+T$:
\begin{equation}
	\vec{x}(t+T)=A\vec{\phi}(t+T)+B\vec{\psi}(t+T)
	\label{eq:54}	
\end{equation}
С другой стороны, подставим \eqref{eq:53} и \eqref{eq:54} в \eqref{eq:52}:
\begin{equation}
	A\vec{\phi}(t+T)+B\vec{\psi}(t+T)=s\qty[A\vec{\phi}(t)+B\vec{\psi}(t)].
	\label{eq:55}	
\end{equation}
Теперь используем \eqref{eq:50}:
\begin{equation}
	\qty[A(a-s)+Bc]\vec{\phi}(t)+\qty[Ab+B(d-s)]\vec{\psi}(t) \equiv 0.
	\label{eq:56}	
\end{equation}
Для выполнения равенства при любом $t$ нужно, чтобы скобки с коэффициентами обратились в ноль:
\begin{equation}
	\left\{\begin{aligned}
		&A(a-s)+Bc=0 \\
		&Ab+B(d-s)=0.		
	\end{aligned}\right.
	\label{eq:57}
\end{equation}
Это СЛАУ относительно коэффициентов $A$, $B$. Раскрываем определитель:
\begin{equation}
	s^2-(a+d)s+ad-bc=0.
	\label{eq:58}	
\end{equation}
Оказывается, что $ad-bc$ всегда можно найти, не находя общего решения.

Вернемся к вронскиану. Подсчитаем его в момент времени $T$:
\begin{gather*}
	W(T)= 
	\begin{vmatrix}
		\phi_1(T) & \psi_1(T) \\ 
		\phi_2(T) & \psi_2(T)
	\end{vmatrix}
	=
	\begin{vmatrix}
		a & c \\ 
		b & d
	\end{vmatrix}
	=ad-bc.
\end{gather*}
с другой стороны,
\begin{eqnarray*}
	W(T)=\underbrace{W(0)}_{\text{=1}}\exp\qty[\int_0^T(p_{11}(t)+p_{22}(t))\dd{t}] \\
\end{eqnarray*}
\begin{equation}
	ad-bc=\exp\qty[\int_0^T\qty(p_{11}(t)+p_{22}(t))\dd{t}].
	\label{eq:59}
\end{equation}
Так как $p_{11}, p_{22}$ мы знаем из постановки конкретной задачи, то $ad-bc$ всегда найдём.

Предположим, что \eqref{eq:58} не имеет кратных корней, следовательно, существует два значения мультипликатора и два решения:
\begin{gather}
	\vec{x}_1(t+T)=s_1\vec{x}_1(t) \notag \\ 
	\vec{x}_2(t+T)=s_2\vec{x}_2(t).		
	\label{eq:60}
\end{gather}
Таким образом, решение воспроизводит себя через период с точностью до множителя. 
\begin{gather}
	\vec{x}_1(t+nT)=(s_1)^n\vec{x}_1(t) \notag \\ 
	\vec{x}_2(t+nT)=(s_2)^n\vec{x}_2(t).		
	\label{eq:61}
\end{gather}

Покажем, что решения $\vec{x}_1,\vec{x}_2$ можно представить в следующем виде:
\begin{gather}
	\vec{x}_1(t)=e^{\lambda_1t}\vec{\Phi}_1(t), \qquad
	\vec{x}_2(t)=e^{\lambda_2t}\vec{\Phi}_2(t),		
	\label{eq:62}
\end{gather}
где $\lambda$ -- некоторые числа, которые называются характеристическими, а вектора $\Phi$ имеют вид
\begin{gather*}
	\vec{\Phi}_1(t)= 
	\begin{pmatrix}
		\Phi_{11} \\
		\Phi_{21}
	\end{pmatrix}
	,\quad
	\vec{\Phi}_2(t)= 
	\begin{pmatrix}
		\Phi_{12} \\
		\Phi_{22}
	\end{pmatrix}
	,
\end{gather*}
и являются периодическими с периодом $T$:
\begin{equation}
	\Phi_{jk}(t+T)=\Phi_{jk}(t),
	\label{eq:63}
\end{equation}
а характеристические числа представляются в следующем виде:
\begin{gather}
	\lambda_j=\frac{1}{T}\,\mathrm{Ln}\,s_j=\frac1{T}\qty[\vphantom{\bigg|}\ln|s_j|\pm i\qty(\vphantom{\big|}\arg s_j +2\pi k)], \quad
	\begin{aligned}
		&j=1,2, \\
		&k=0,\pm 1, \pm 2, \ldots	
	\end{aligned}
	\label{eq:64}
\end{gather}
$s$ -- мультипликаторы, которые могут быть комплексными, поэтому присутствует аргумент комплексного числа.

Покажем справедливость \eqref{eq:63}. Из \eqref{eq:62}:
\begin{equation*}
	\vec{\Phi}_j(t)=e^{-\lambda_j t}\vec{x}_j(t),
\end{equation*}
но $\vec{x}_j$ можно выразить из \eqref{eq:60}:
\begin{gather*}
	\vec{x}_j(t+T)=s\vec{x}_j(t)=e^{\lambda_0 T}\vec{x}_j(t), \\
	\vec{\Phi}_j(t+T)=e^{-\lambda_0(t+T)}\vec{x}_j(t+T)=e^{\lambda_0 T}\vec{x}_j(t)e^{-\lambda_j(t+T)}=\vec{x}_j(t)e^{-\lambda_j t}=\vec{\Phi}_j(t).
\end{gather*}

С другой стороны, $\vec{x}_j$ образуют ФСР, поэтому общее решение можно записать в виде
\begin{equation}
	\vec{x}_j(t)=\sum_{j=1}^2 c_j e^{\lambda_j t}\vec{\Phi}_j(t).
	\label{eq:65}
\end{equation}

Это основное соотношение ЛДУ с переменными параметрами. Здесь $\Phi$ -- функции Флоке. 

\paragraph{Замечание. } Теория Флоке может быть применена к предельным циклам, так как  они являются переодическим решением. 




\subsection{Отображение через период}

С точки зрения теории динамических систем, \eqref{eq:47} представляет собой неавтономную систему, следовательно, порождает точечное отображение 
% \begin{figure}[H]
% 	\centering
% 	\includegraphics[scale=1.5]{img/parametric_oscillations/x_to_v}
% \end{figure}

\begin{wrapfigure}{l}{6cm}
	\centering
    \vspace{-1ex}%сместить картинку немного вверх  относительно текста
    \includegraphics[scale=1.5]{img/parametric_oscillations/x_to_v}
    % \caption{Предельная нагрузка настила по условию прогиба}
    \label{img02}
\end{wrapfigure}
$$g^t:\mathds{R}^2\rightarrow \mathds{R}^2.$$
Под действием системы \eqref{eq:47} вектор $\vec{x}(0)$ переходит в другой вектор $\vec{v}(t)$:
\begin{equation*}
	g^t\,\vec{x}(0)=\vec{v}(t).
\end{equation*}
Если $t=0: \,\vec{x}(0)=\vec{v}(0)$.
В силу теории Флоке наибольший интерес представляет отображение через период, т.е. $g^T$. Заметим, что решение $x_1=0, x_2=0$ является неподвижной точкой этого отображения. Более того, с точки зрения неавтономной системы, это периодическое решение любого периода. 

Воспользуемся соотношением Флоке:
\begin{gather}
	x_1(t)=C_1 e^{\lambda_1 t}\Phi_{11}(t)+C_2 e^{\lambda_2 t}\Phi_{12}(t) \notag \\ 
	x_2(t)=C_1 e^{\lambda_1 t}\Phi_{21}(t)+C_2 e^{\lambda_2 t}\Phi_{22}(t).	
	\label{eq:66}
\end{gather}
Пусть $x_1(0)=x_1$, $x_2(0)=x_2$. Подставим $t=0$ в \eqref{eq:66}, находим $C_1^0, C_2^0$ через получившуюся СЛАУ. 
% Задаем точку, находим ее образ через период:
% \begin{gather}
% 	x_1(t)=C_1^0 e^{\lambda_1 t}\Phi_{11}(t)+C_2^0 e^{\lambda_2 t}\Phi_{12}(t) \notag \\ 
% 	x_2(t)=C_1^0 e^{\lambda_1 t}\Phi_{21}(t)+C_2^0 e^{\lambda_2 t}\Phi_{22}(t),		
% 	\label{eq:67}
% \end{gather}

Используя \eqref{eq:53}, а также связь A и B с a, b, c, d:
\begin{gather*}
	\Phi_{11}(0)=A_1,~\Phi_{12}(0)=A_2,~\Phi_{21}(0)=B_1,~\Phi_{22}(0)=A_2.	
\end{gather*}
\begin{equation}
	g^T\rightarrow
	\left\{\begin{aligned}
		x_1(T)=a x_1(0)+c x_2(0) \\
		x_2(T)=b x_1(0)+d x_2(0)		
	\end{aligned}\right.
	\label{eq:68}
\end{equation}
\begin{gather*}
	\vec{x}(T)=G \vec{X}(0),~ G=
	\begin{pmatrix}
		a ~b \\
		c ~d \\
	\end{pmatrix}
	,\\
	a=\phi_1(T), ~ b=\phi_2(T), \\
	c=\psi_1(T), ~ d=\psi_2(T).
\end{gather*}

Траектории системы порождают порождают точечное отображение. Роль дискрета играет период T. Отображение линейное. Нужно исследовать устойчивость неподвижных точек. 
\subsection{Устойчивость нулевого решения}
Вспомним точечное отображение. Поведение $g^T$ зависит от мультипликаторов.

Пусть $s_1, s_2$ - действительные. Тогда \eqref{eq:68} с помощью преобразования можно привести к виду:
\begin{equation*}
	\left\{\begin{aligned}
		U_1(T)=s_1 U_1(0) \\
		U_2(T)=s_2 U_2(0)		
	\end{aligned}\right.
\end{equation*}
\begin{enumerate} 
	\item Поскольку мультипликаторы $s_1, s_2$ соответствуют решениям неавтономной \eqref{eq:47} системы, они удовлетворяют условию $s_1\cdot s_2>0$;
	\item Если $|s_j|<1$, неподвижная точка устойчива, если $|s_j|>1$ - неустойчива. Если $|s_i|<1, |s_j|>1, s_i \neq s_j$, то точка - седло. 
\end{enumerate} 

Пусть $0<s_j<1$:
\begin{figure}[H]
	\centering
	\includegraphics[width=0.35\linewidth]{fig/fig34.pdf}   
\end{figure}

Вектор уменьшит длину и повернется. Любое начальное условие стремится к нулю, следовательно, решение устойчивое.

$-1<s_j<0$:
\begin{figure}[H]
	\centering
	\includegraphics[width=0.35\linewidth]{fig/fig35.pdf}   
\end{figure}

Вектор уменьшит длину и переместится в отрицательную область. Будут скачки. Решение устойчиво.

$0<|s_1|<1, |s_2|>1$:
\begin{figure}[H]
	\centering
	\includegraphics[width=0.35\linewidth]{fig/fig36.pdf}   
\end{figure}

По одной координате растяжение, а по другой сжатие. Вектор асимптотически прижимается к оси ординат.

$-1<|s_1|<0, |s_2|<-1$:
\begin{figure}[H]
	\centering
	\includegraphics[width=0.35\linewidth]{fig/fig37.pdf}   
\end{figure}

Вектор прижимается к оси координат и растягивается.

29.04

Рассмотрим комплексно-сопряженные корни. Линейным преобразованием систему можно привести к нормальной форме.
\begin{equation}
	U_j(T)=s_j U_j(0), ~j=1,2.
	\label{eq:69}
\end{equation}

$|s|<1$ - длина вектора меняется.
\begin{equation}
	s_j=\alpha \pm i\beta.
	\label{eq:70}
\end{equation}
Поскольку мультипликаторы являются комплексно сопряженными, переменные $U_j(T), U_j(0)$ являются комплексными функциями. 
\begin{gather}
	U_j(0)=U(0)\pm iV(0) \notag \\ 
	U_j(T)=U(T)\pm iV(T).		
	\label{eq:71}
\end{gather}

Подставляя \eqref{eq:70}, \eqref{eq:71} в \eqref{eq:69} и разделяя реальную и мнимую части, получим:
\begin{gather}
	U(T)=\alpha U(0)- \beta V(0) \notag \\ 
	V(T)=\beta U(0)+ \alpha V(0).		
	\label{eq:72}
\end{gather}
\begin{gather*}
	U=\rho \cos{\phi}, V=\rho \sin{\phi}, \\ 
	s=\alpha \pm i\beta=|s|e^{\pm i \omega}, \\
	|s|=\sqrt{\alpha^2+\beta^2},		
\end{gather*}
\begin{gather}
	\alpha=|s|\cos{\omega} \notag \\ 
	\beta=|s|\sin{\omega}.		
	\label{eq:73}
\end{gather}
\begin{equation}
	\left\{\begin{aligned}
		\rho(T)\cos{\phi(T)}=\alpha \rho(0)\cos{\phi(0)}-\beta \rho(0)\sin{\phi(0)} \\
		\rho(T)\sin{\phi(T)}=\beta \rho(0)\cos{\phi(0)}+\alpha \rho(0)\sin{\phi(0)}.
	\end{aligned}\right.
	\label{eq:74}
\end{equation}
Из этой системы находят $\rho(T)$ и $\phi(T)$:
\begin{equation}
	\left\{\begin{aligned}
		\phi(T)=\phi(0)+\omega \\
		\rho(T)=|s|\rho(0).
	\end{aligned}\right.
	\label{eq:75}
\end{equation}

Эта система задает отображение $g^T$. Полярный угол $\phi$ меняется за период на $\omega$, а начальная величина вектора $\rho(0)$ изменяется в зависимости от s. 

Если $|s|<1$:
\begin{figure}[H]
	\centering
	\includegraphics[width=0.35\linewidth]{fig/fig38.pdf}   
\end{figure}

Вектор поворачивается на угол $\omega$ и сокращается в длину. Состояние равновесия и соответствующая ему неподвижная точка $x_1=x_2=0$ являются асимптотически устойчивыми.

$|s|>1$:
\begin{figure}[H]
	\centering
	\includegraphics[width=0.35\linewidth]{fig/fig39.pdf}   
\end{figure}

Состояние равновесия неустойчивое, вектор поворачивается и растягивается.

$|s|=1$:
\begin{figure}[H]
	\centering
	\includegraphics[width=0.35\linewidth]{fig/fig40.pdf}   
\end{figure}

Вектор поворачивается, длина его не меняется. 

Вектор, пройдя период, может либо совпасть с начальным, либо нет. В зависимости от этого решение будет периодическим или квазипериодическим.

Таким образом, состояние равновесия или периодическая траектория любого периода в параметрической системе может быть устойчивой либо не устойчивой.

\subsection{Основные режимы линейной параметрической системы}

Поскольку система линейная, из условия устойчивости состояния равновесия вытекают все остальные свойства линейных параметрических систем, а именно:
\begin{enumerate} 
	\item Параметрическая система, находящаяся в начальный момент в состоянии равновесия, останется в этом состоянии при $\forall t>0$, поскольку состояние равновесия $x_1=x_2=0$ существует всегда в линейной системе. Параметрическую систему, находящуюся в состоянии равновесия, нельзя вывести из этого состояния, изменяя ее параметры. Пример - маятник на нитке: дергая за нитку, маятник нельзя раскачать;
	\item Состояние равновесия параметрической системы может быть как устойчивым, так и не устойчивым;
	\item Если параметры системы таковы, что она неустойчива и система выведена из состояния равновесия, то в ней возникают колебания, амплитуда которых экспоненциально возрастает. Этот процесс возрастания размахов колебаний при периодическом нарастании колебаний, называется параметрическим резонансом.
\end{enumerate} 

\subsection{Параметрические колебания. Резонанс.}
Установим условие существования параметрического резонанса в одном частном, но важном случае. Рассмотрим систему \eqref{eq:47} в случае $p_{11}(t) \equiv0, p_{12}(t)\equiv 1, p_{22}(t) \equiv 0$, тогда система примет вид:
\begin{equation}
	\left\{\begin{aligned}
		\dot{x_1}=x_2 \\
		\dot{x_2}=p_{21}(t)x_1.
	\end{aligned}\right.
	\label{eq:76}
\end{equation}

Система \eqref{eq:76} охватывает уравнения Матье и Хилла. 

Поведение системы определяют мультипликаторы, а они, в свою очередь, определяются характеристическим уравнением:
\begin{gather*}
	s^2-(a+d)s+(ad-bc)=0, \\
	ad-bc=exp\qty[\int_0^T(p_{11}(t)+p_{22}(t))\dd{t}]=1.		
\end{gather*}

Фактически, варьируются только коэффициенты a и d. Введем: $2P=a+d$ - контрольный параметр.
\begin{gather*}
	s^2-2Ps+1=0.		
\end{gather*}

Проанализируем поведение s.
Пусть $|P|<1$
\begin{gather*}
	s_{1,2}=P\pm i\sqrt{1-P^2}, \\
	|s|=1.		
\end{gather*}

Неподвижная точка отображения $g^T$ будет эллиптической. 

Достаточно рассмотреть только $x_1(t)$:
\begin{equation}
	x_1(t)=c_1e^{\lambda_1 t}\Phi_{11}(t)+c_2e^{\lambda_2 t}\Phi_{12}(t).
	\label{eq:77}
\end{equation}

Поскольку $|s|=1$, характеристические показатели $\lambda$ связаны с s: $\lambda_1=\frac{q}{T}i,~\lambda_2=-\frac{q}{T}i$, где $q=|arg(s_j)+2\pi k|$ (из формулы \eqref{eq:64}). $\lambda$ чисто мнимые, x действительные, следовательно, $c_1, c_2, \Phi_{11}, \Phi_{12}$ должны быть комплексными:
\begin{equation}
	\left\{\begin{aligned}
		c_1=\frac{A}{2}e^{ic} \\
		c_2=\frac{A}{2}e^{-ic} \\
		\Phi_{11}=h(t)e^{i\varkappa(t)} \\
		\Phi_{12}=h(t)e^{-i\varkappa(t)}.
	\end{aligned}\right.
	\label{eq:78}
\end{equation}

Подставляя \eqref{eq:78} в \eqref{eq:77}, получим:
\begin{equation}
	x_1(t)=Ah(t)\cos (\frac{q}{T}t+c+\varkappa(t)).
	\label{eq:79}
\end{equation}

Вспомним косинус суммы, чтобы выделить соответственные временные масштабы. $h(t)$ и $\varkappa(t)$ периодичны с периодом T (из теории Флоке). T - период изменения параметров.
\begin{equation}
	x_1(t)=H(t)\cos{(\frac{q}{T}t+c)}+F(t)\sin{(\frac{q}{T}t+c)}.
	\label{eq:80}
\end{equation}
\begin{gather*}
	H(t)=Ah(t)\cos \varkappa(t), \\
	F(t)=-Ah(t)\sin \varkappa(t).		
\end{gather*}

В \eqref{eq:80} $H(t)$ и $F(t)$ периодичны с периодом T. Есть два временных масштаба: $T_1=T$ и $T_2=\frac{2\pi}{q}T$, которым отвечают частоты $\omega_1$ и $\omega_2$ соответственно.

$\frac{\omega_1}{\omega_2}=\frac{2\pi}{q}$, откуда вытекает, что, если $\frac{2\pi}{q}$ - рациональное число, то $x_1(t)$ - периодическая функция, если иррациональное, то $x_1(t)$ - квазипериодическая функция. Таким образом, при выполнении $P>1$, в системе \eqref{eq:76} реализуются ограниченные колебания, которые называются параметрическими.

Пусть $|P|>1$.

Из характеристического уравнения $s_1 \cdot s_2=1$, они действительны. Для определенности $|s_1|>1,~ |s_2|<1$.

Пусть $P>1$. Действует отображение $g^T$:
\begin{figure}[H]
	\centering
	\includegraphics[width=0.35\linewidth]{fig/fig41.pdf}   
\end{figure}

По переменной $U_1$ увеличение вектора, по $U_2$ - уменьшение. Процесс повторяется. Вектор вытягивается и прижимается к оси абсцисс. Если $n\rightarrow \infty$ (число итераций): $U_2(nT)\rightarrow 0,~U_1(nT)\rightarrow \infty$. В системе реализуется параметрический параметрический резонанс. 

Получим приближенное соотношение для $x_1(t)$:
\begin{gather*}
	\lambda_1=\frac{1}{T}ln(s_1)>0, \\
	\lambda_2=\frac{1}{T}ln(s_2)=\frac{1}{T}ln(\frac{1}{s_1})=-\frac{1}{T}ln(s_1)<0.
\end{gather*}

Функция Флоке ограниченная
\begin{equation*}
	x_1(t)\approx c_1 e^{\frac{1}{T}ln(s_1)t}\Phi_{11}(t).
\end{equation*}
\begin{figure}[H]
	\centering
	\includegraphics[width=0.5\linewidth]{fig/fig42.pdf}   
\end{figure}

В начальный момент система должна быть выведена из равновесия. Это геометрическая прогрессия, экстремумы функции лежат на экспоненте. 
\begin{itemize}
	\item В линейном осцилляторе экстремумы на прямых;
	\item Если линейный осциллятор в покое, мы можем его раскачать;
\end{itemize}

Пусть $s_1,s_2<0,~|s_1|>1,~|s_2|<1$:
\begin{figure}[H]
	\centering
	\includegraphics[width=0.5\linewidth]{fig/fig43.pdf}   
\end{figure}

Вектор вернется через 2 итерации, период 2T, экстремумы отстоят на период.
\begin{figure}[H]
	\centering
	\includegraphics[width=0.4\linewidth]{fig/fig44.pdf}   
\end{figure}

Пусть $|P|=1$. В этом случае, если $P=1$, то $s_1=s_2=1$; если $P=-1$, то $s_1=s_2=-1$. Корни кратные, мы такое не рассматривали. Решение в этом случае задается в виде:
\begin{gather*}
	x_1(t)=c_1 e^{\lambda t}\Phi(t)+c_2te^{\lambda t}\Phi(t), \\
	\lambda=\frac{1}{T}ln|s|.
\end{gather*}

На границе неустойчивость и состояние равновесия $x_1=x_2=0$ является неустойчивым.

\subsection{Параметрические колебания маятника}
\begin{equation}
	\ddot{\varphi}+2\delta \dot{\varphi}+\frac{g}{l(t)}\varphi=0.
	\label{eq:81}
\end{equation}

Рассмотрим маятник, у которого длина периодически меняется, $\delta$ характеризует диссипацию. Кусочно-линейная аппроксимация. $l(t+T)=l(t)$.

Будем считать, длина меняется мгновенно:
\begin{equation*}
	l(t)=
	\left\{\begin{aligned}
		l_0-\frac{a}{2}, 0 \leq t \leq \frac{T}{2} \\
		l_0+\frac{a}{2}, \frac{T}{2} \leq t \leq T		
	\end{aligned}\right.
\end{equation*}

далее периодично. Получается меандр.
\begin{equation*}
	l_0=\frac{a}{2}>0,~\omega_1^2=\frac{g}{l_0-\frac{a}{2}},~\omega_2^2=\frac{g}{l_0+\frac{a}{2}}.
\end{equation*}

Рассмотрим консервативный случай ($\delta=0$). Перепишем \eqref{eq:81} в виде:
\begin{equation}
	\left\{\begin{aligned}
		\dot{\varphi}=y \\
		\dot{y} =-\omega^2(t)\varphi.	
	\end{aligned}\right.
	\label{eq:82}
\end{equation}
\begin{equation}
	\omega^2(t)=
	\left\{\begin{aligned}
		\omega^2_1, 0 \leq t \leq \frac{T}{2} \\
		\omega^2_2, \frac{T}{2} \leq t \leq T		
	\end{aligned}\right.
	\label{eq:83}	
\end{equation}

\eqref{eq:82} является частным случаем рассмотренного выше, где $p_{21}=\omega^2(t)$. Для построения областей надо построить границы $P=\pm 1$, найдя a,d. Они являются элементами матрицы G. которая задает $g^T$. Поскольку система является кусочно-линейной, можно представить: $G=G_1\cdot G_2$, где $G_1$ действует на интервале от 0 до $\frac{T}{2}$, $G_2$ от $\frac{T}{2}$ до $T$. 

\underline{Матрица $G_1$:}

$\omega^2(t)\equiv\omega_1^2$

Решая \eqref{eq:82}:
\begin{gather*}
	\varphi_1(t)\equiv A_1 \cos(\omega_1 t) + B_1\sin(\omega_1 t), \\
	\varphi_2(t)\equiv -A_1 \sin(\omega_1 t) + \omega_1 B_1\cos(\omega_1 t).
\end{gather*}

Положим: $\varphi_1(0)=1,~\varphi_2(0)=0$. $A_1=1,~B_1=0$. Искомое решение примет вид: 
\begin{equation}
	\left\{\begin{aligned}
		\varphi_1(t) = \cos(\omega_1 t) \\
		\varphi_2(t) = -\sin(\omega_1 t).		
	\end{aligned}\right.
	\label{eq:84}	
\end{equation}

Аналогично находится второе решение:
\begin{gather}
	\psi_1(0)=0, \psi_2(0)=1, \notag \\
	\left\{\begin{aligned}
		\psi_1(t) = \frac{\sin(\omega_1 t)}{\omega_1} \\
		\psi_2(t) = \cos(\omega_1 t).		
	\end{aligned}\right.
	\label{eq:85}	
\end{gather}

Подставляя $t=\frac{T}{2}$, найдем элементы $G_1$:
\begin{equation}
	\left\{\begin{aligned}
		a_1 = \cos(\omega_1 \frac{T}{2}) \\
		b_1 = -\omega_1 \sin(\omega_1 \frac{T}{2}) \\
		c_1 = \frac{\sin(\omega_1 \frac{T}{2})}{\omega_1} \\
		d_1 = \cos(\omega_1 \frac{T}{2}),	
	\end{aligned}\right.
	\label{eq:86}	
\end{equation}
\begin{gather*}
	G_1= 
	\begin{pmatrix}
		\cos \alpha~~~~~\frac{\sin\alpha}{\omega_1} \\
		-\omega_1 \sin \alpha~~\cos \alpha
	\end{pmatrix}
	, 
	\alpha=\omega_1 \frac{T}{2}.	
\end{gather*}

\underline{Матрица $G_2$:}
Вид устанавливается аналогично:
\begin{gather*}
	G_1= 
	\begin{pmatrix}
		\cos \beta~~~~~\frac{\sin\beta}{\omega_2} \\
		-\omega_2 \sin \beta~~\cos \beta
	\end{pmatrix}
	, 
	\beta=\omega_2 \frac{T}{2}.	
\end{gather*}

\underline{ $G_1\cdot G_2$:}

Получили матрицу, элементы которой имеют вид:
\begin{gather*}
	a=\cos \alpha \cdot \cos \beta -\frac{\omega_1}{\omega_2} \sin \alpha \cdot \sin \beta, \\
	b=-\omega_2 \cos \alpha \cdot \sin \beta - \omega_1 \sin \alpha \cdot \sin \beta, \\
	c=\frac{\sin \alpha \cdot \cos \beta}{\omega_1}+\frac{\cos \alpha \cdot \sin \beta}{\omega_2}, \\
	d=\cos \alpha \cdot \cos \beta-\frac{\omega_2}{\omega_1}\sin \alpha \cdot \sin \beta,
\end{gather*}
\begin{equation}
	2P=a+d=2\cos \alpha\cdot \cos \beta - \frac{\omega_1^2+\omega_2^2}{\omega_2 \omega_1} \sin \alpha \cdot \sin \beta.
	\label{eq:87}	
\end{equation}

Введем более физичные параметры: собственная частота, если бы длина не менялась: $\omega_0^2=g/l_0$; период, если бы длина не менялась: $T_0=2\pi /\omega_0$; глубина параметрической модуляции: $\varepsilon=a/2l_0$; отношение собственного периода к периоду параметрической накачки: $\gamma=T/T_0$. Тогда:
\begin{gather*}
	\alpha=\frac{\pi \gamma}{\sqrt{1-\epsilon}},~\beta=\frac{\pi \gamma}{\sqrt{1+\epsilon}},~\frac{\omega_1^2+\omega_2^2}{\omega_2 \omega_1}=\frac{2}{\sqrt{1-\epsilon^2}},~\varepsilon<1.
\end{gather*}

Если $\varepsilon=0$, то параметрической модуляции нет.
\begin{equation}
	P=\cos(\frac{\pi \gamma}{\sqrt{1-\epsilon}}) \cos (\frac{\pi \gamma}{\sqrt{1+\epsilon}}) - \frac{1}{\sqrt{1-\epsilon^2}} \sin(\frac{\pi \gamma}{\sqrt{1-\epsilon}}) \sin (\frac{\pi \gamma}{\sqrt{1+\epsilon}}).
	\label{eq:88}	
\end{equation}

20.05

Таким образом, у нас осталось два параметра: $\gamma$ и $\varepsilon$. Строим границы, где реализуется параметрический резонанс, используя формулу \eqref{eq:88}. Подставим в \eqref{eq:88} $\varepsilon=0$. P примет вид: $P=\cos(2\pi \gamma)$.

Границы области: $P=\pm1, P=0$.
\begin{gather*}
	P=0 \rightarrow	\gamma=\frac14+\frac{k}2,~k=0,1,2,\dots, \\
	P=-1 \rightarrow	\gamma=k+\frac12,~k=0,1,2,\dots, \\
	P=1 \rightarrow	\gamma=k,~k=1,2,\dots.
\end{gather*}

Нарисуем плоскость параметров (для малых $\varepsilon$, построен численно; при увеличении $\varepsilon$ границы зоны сближаются, образуют петлю и снова расходятся):
\begin{figure}[H]
	\centering
	\includegraphics[width=0.5\linewidth]{fig/fig45.pdf}   
\end{figure}

Получились "клювы", выходящие из последовательности точек на оси абсцисс. Таких точек бесконечное число. 

Зоны становятся уже с увеличением номера. Одно семейство образует $P^-$ зоны, лежащие в области $P<0$. Семейство в $P^+$ лежат в других областях. Во всех зонах реализуется параметрическая неустойчивость. В зонах, принадлежащих $P^-$ мультипликаторы нулевого состояния равновесия являются отрицательными, в $P^+$ - положительными.  

Вне этих зон будут либо периодические, либо квазипериодические колебания, в зависимости от параметра $\gamma$. Диссипация, если таковая имеется, приведет к тому, что глубины модуляции изменятся. Зоны будут начинаться не с 0, а с $\varepsilon>0$. 
	
	\subsection{Динамика маятника с вибрирующей точкой подвеса. Маятник Капицы}
	\label{ssec:kap_pendulum}
	% 29.04.19 
		%!TEX root = lections.tex
Рассмотрим шарик на упругом стержне, массой стержня можно пренебречь. Точка подвеса колеблется с амплитудой a и периодом $2\tau$. Изначально маятник расположен в верхнем положении равновесия. 

\begin{wrapfigure}[12]{l}{0.5\linewidth} 
\vspace{0.25em}
\centering
\includegraphics[scale=1.5]{fig/fig46.pdf}
\vspace{-0.25em}
% \caption{Some caption}
% \label{fig:somelabel}
\end{wrapfigure}

% \begin{figure}[H]
% 	\centering
% 	\includegraphics[width=0.5\linewidth]{fig/fig46.pdf}   
% \end{figure}

$\tau \ll 1$ (точка подвеса быстро осциллирует) и $l\gg a$.

Пусть точка подвеса совершает равнопеременные движения с ускорением $a \pm c$, тогда с каждым полупериодом $c=\frac{8a}{\tau^2}$, частота $\omega_p^2=\frac{c}{l}=\frac{8a}{l\tau^2}$. Считаем, что диссипации нет. 

Систему можно описать уравнениями:
\begin{equation}
	\begin{cases}
		\dot{\varphi}=y \\
		\dot{y}=(\omega_0^2 \pm \omega_p^2)	\varphi,
	\end{cases}
	\label{eq:89}	
\end{equation}
где $\omega_0$ - собственная частота маятника, если бы не было колебаний подвеса. Знак $\pm$ изменяется через время $\tau,~ \omega_p^2>\omega_0^2$. Эта система параметрическая, к ней можно применить недавно рассмотренную теорию. 

Система  \eqref{eq:89} является кусочно-линейной. Построим матрицу G.

\underline{Матрица $G_1$:}
Пусть в начальный момент точка подвеса в крайнем верхнем положении. Тогда в течение первого полупериода:
\begin{gather*}
	\dot{\varphi}=y,~\dot{y}=p^2\varphi,~p^2=\omega_0^2 + \omega_p^2.
\end{gather*}
p действительные и разного знака. ФСР можно записать в виде:
\begin{gather}
	\phi_1(t)=\ch(pt), \psi_1(t)=\frac1{p}\sh(pt), \notag \\ 
	\phi_2(t)=p\sh(pt), \psi_2(t)=\ch(pt).		
	\label{eq:90}
\end{gather}

Получим элементы:
\begin{gather*}
	a_1=\ch(p\tau), c_1=\frac1{p}\sh(p\tau), \notag \\ 
	b_1=p\sh(p\tau), d_1=\ch(p\tau).		
\end{gather*}

При нахождении \underline{матрицы $G_2$:} будем считать, что маятник в нижнем положении и начинает движение вверх:
\begin{gather}
	\dot{\varphi}=y,~\dot{y}=p^2\varphi,~p^2=\omega_0^2 - \omega_p^2.
	\label{eq:91}
\end{gather}

Корни чисто мнимые. ФСР:
\begin{gather*}
	\phi_1(t)=\cos(\omega t), \psi_1(t)=\frac1{\omega}\sin(\omega t), \notag \\ 
	\phi_2(t)=-\omega \sin(\omega t), \psi_2(t)=\cos(\omega t).		
\end{gather*}

Подставляя $\tau$, найдем:
\begin{gather*}
	a_1=\cos(\omega \tau), c_1=\frac1{\omega}\sin(\omega \tau), \notag \\ 
	b_1=-\omega \sin(\omega \tau), d_1=\cos(\omega \tau).		
\end{gather*}

Перемножаем $G_1 \cdot G_2$, получаем:
\begin{gather*}
	P=a+d=2\ch(p \tau)\cos(\omega \tau)+(\frac{p}{\omega}-\frac{\omega}{p})\sh(p \tau)\sin(\omega \tau).		
\end{gather*}

Для устойчивости $|P|<1$. Получаем неравенство на параметры. Оно строилось численно, заданы: $a=0.1$см, $l=100$ см.

\begin{wrapfigure}[9]{l}{0.5\linewidth} 
	\vspace{0.1em}
	\centering
	\includegraphics[scale=1.1]{fig/fig47.pdf}
	\vspace{-0.25em}
\end{wrapfigure}

Существует интервал, где неравенство выполняется. В данном случае $\tau_s=0.006738$. Число колебаний $N=\frac{1}{2\tau}$. Если $N>678$ колебаний в секунду, то верхняя точка становится устойчивой. Внешнее воздействие превратило неустойчивое состояние равновесия в высокочастотное колебание вблизи состояния равновесия.

\subsubsection{Колебания линейного осциллятора с медленно меняющейся частотой}
Рассмотрим линейный осциллятор:
\begin{gather}
	\ddot{x}+\omega_0^2(\mu t)x=0, \\
	\label{eq:92}
	\omega_0(\mu t)>0,~0< \mu \ll t, \notag \\
	\tau=\mu t, \notag \\
	\begin{cases}
		\mu \dv{x}{\tau}=y \\
		\mu \dv{y}{\tau}=-\omega_0^2(\tau)x.
	\end{cases}	
	\label{eq:93}
\end{gather}

Для исследования используем метод Вентцеля-Крамерса-Бриллюэна(ВКБ). В соответствии с ним, будем искать решение в виде:
\begin{gather}
	\begin{cases}
		x=e^{\frac{s(\tau)}{\mu}}\sum\limits_{j=0}^{\infty}\mu^j U_j(\tau) \\
		y=e^{\frac{s(\tau)}{\mu}}\sum\limits_{j=0}^{\infty}\mu^j V_j(\tau),
	\end{cases}	
	\label{eq:94}
\end{gather}
где $s(\tau),~U_j(\tau),~v_j(\tau)$ - неизвестные функции, которые надо найти. 

Будем искать решение нулевого порядка по $\mu$:
\begin{gather}
	\begin{cases}
		x=e^{\frac{s(\tau)}{\mu}}\qty[U_0(\tau)+\mu U_1(\tau)] \\
		y=e^{\frac{s(\tau)}{\mu}}\qty[V_0(\tau)+\mu V_1(\tau)].
	\end{cases}	
	\label{eq:95}
\end{gather}

Подставляем \eqref{eq:95} в \eqref{eq:93} и группируем слагаемые по порядку $\mu$:
\begin{equation}
	\mu^0:
	\begin{cases}
		\dv{s}{\tau} U_0(\tau)-V_0(\tau)=0 \\
		\omega^2_0(\tau)U_0(\tau)+\dv{s}{\tau} V_0(\tau)=0, 		
	\end{cases}
	\label{eq:96}	
\end{equation}
\begin{equation}
	\mu^1:
	\begin{cases}
		\dv{s}{\tau} U_1(\tau)+\dv{U_0}{\tau}=V_1(\tau) \\
		\dv{s}{\tau} V_1(\tau)+\dv{V_0}{\tau}=-\omega^2_0(\tau)U_1(\tau). 		
	\end{cases}
	\label{eq:97}	
\end{equation}

Рассмотрим систему относительно $U_0,~V_0$:
\begin{equation}
	(\dv{s}{\tau})^2+\omega^2_0(\tau)=0,
	\label{eq:98}	
\end{equation}
\begin{equation}
	\dv{s}{\tau}=\pm i\omega_0(\tau),
	\label{eq:99}	
\end{equation}

Проинтегрируем:
\begin{gather}
	s(\tau)=\pm i \int_0^{\tau}\omega_0(\tau)d \tau, ~U_0=\psi(\tau), ~V_0=\pm i \omega_0(\tau)\psi(\tau), 
	\label{eq:100}
\end{gather}
где $\psi(\tau)$- произвольная функция.

Система \eqref{eq:100} определяет пару комплексно сопряженных решений. Рассмотрим с $"+"$:
\begin{equation}
	V_1(\tau)=\dv{s}{\tau}U_1(\tau)+\dv{\psi}{\tau}.
	\label{eq:101}	
\end{equation}

Подставляя \eqref{eq:100} и \eqref{eq:101} в \eqref{eq:97}, получим:
\begin{equation}
	2\omega_0(\tau)\dv{\psi}{\tau}=-\psi\dv{\omega_0(\tau)}{\tau}.
	\label{eq:102}	
\end{equation}

В \eqref{eq:102} переменные разделяются. Проинтегрируем:
\begin{equation*}
	\psi(\tau)=\frac{A}{\sqrt{\omega_0(\tau)}}.
\end{equation*}

Если взять <<-->>, получим:
\begin{equation*}
	x(\tau) \approx \frac{A}{\sqrt{\omega_0(\tau)}} \exp \qty[\frac{i}{\mu}\int_0^{\tau}\omega_0(\tau)d \tau]+\frac{A}{\sqrt{\omega_0(\tau)}} \exp \qty[-\frac{i}{\mu}\int_0^{\tau}\omega_0(\tau)d \tau],
\end{equation*}
или, возвращаясь к исходному времени
\begin{gather}
	x(t) \approx \frac{2A}{\sqrt{\omega_0(\mu t)}}\cos\vartheta, \\
	\label{eq:103}
	\vartheta=\int_0^{\mu \tau}\omega_0(t)dt. \notag	
\end{gather}

Энергия сохраняется, найдем полную энергию осциллятора:
\begin{equation}
	E=\frac{(\dot x)^2}{2}+\omega_0^2(\mu t)\frac{x^2}{2}, 
	\label{eq:104}
\end{equation}
\begin{equation}
	\dot{x}=-\frac{A\cos\vartheta \cdot \dot{\omega}(\mu t)}{\omega_0(\mu t)\sqrt{\omega_0(\mu t)}}-\frac{2A\sin\vartheta \cdot \omega_0(\mu t)}{\sqrt{\omega_0(\mu t)}}.
	\label{eq:105}	
\end{equation}

В первом приближении $\dot \omega \approx 0$
\begin{equation}
	\dot{x}=-2A\sqrt{\omega_0(\mu t)}\sin\vartheta.
	\label{eq:106}	
\end{equation}

Подставим \eqref{eq:103}, \eqref{eq:106} в \eqref{eq:104}:
\begin{equation*}
	E \approx A^2 \omega_0(\mu t).
\end{equation*}

Энергия меняется пропорционально частоте изменения осциллятора:
\begin{equation*}
	\frac{E}{\omega_0(\mu t)}=const.
\end{equation*}

Увеличивая частоту, увеличиваем и энергию. Но отношение их есть адиабатический (т.к. медленно меняем частоту) инвариант. 

Рассмотрели линейные параметрические системы: резонансные(условия параметрического резонанса), быстро осциллирующие и медленно осциллирующие.


\newpage
\section{Введение в теорию многомерных динамических систем}
\label{sec:multidimensional_dynamic_systems}
% 29.04.19 
	%!TEX root = ../lections.tex
Под многомерными динамическими системами понимаются системы, размерность фазового пространства которых равна трем и выше. 
\begin{wrapfigure}[9]{l}{0.75\linewidth} 
	\vspace{0.1em}
	\centering
	\includegraphics[scale=1]{fig/fig48.pdf}
	\vspace{-0.25em}
\end{wrapfigure}
Для систем на плоскости задача была разбить плоскость параметров на области с разными характеристическими бифуркационными линиями.

На фазовой плоскости существовало конечное число предельных циклов и состояний равновесия.

Многомерные системы делятся на 2 класса:
\begin{enumerate}
	\item Системы с конечным числом состояний равновесия, периодических движений и числом бифуркаций. Такие системы носят название систем Морса-Смейла. Они есть подобие двумерных систем.
	\item Системы с бесконечным числом периодических движений и бифуркаций. 
\end{enumerate}

Интересны бифуркации, при которых один класс переходит в другой.

\subsection{Основные бифуркации состояний равновесия трехмерных систем}
Запишем систему:
\begin{equation}
	\begin{cases}
		\dot x = P(x, y, z, \mu) \\
		\dot y = Q(x, y, z, \mu) \\
		\dot z = R(x, y, z, \mu),
	\end{cases}
	\label{eq:107}	
\end{equation}

Без ограничения общности, будем считать, что состояние равновесия в начале координат $O_0(x=y=z=\mu=0),~\mu \in [-\mu_0,\mu_0]$.

\textit{Двукратное равновесие:}

$\lambda_1(0)=0,~Re[\lambda_{2,3}(\mu)] \neq 0$

Перепишем, приведя систему к нормальному виду:
\begin{gather*}
	\dot U_1 = \mu+lU_1^2+\dots, \\
	\dot V=\beta(\mu)V+\dots, \\
	V=
	\begin{pmatrix}
		U_2 \\
		U_3
	\end{pmatrix}
	.
\end{gather*}

\underline{Случай а:}
\begin{gather*}
	\beta(\mu)=
	\begin{pmatrix}
		\lambda_2~~0 \\
		0~~\lambda_3
	\end{pmatrix}
	.
\end{gather*}

Случай аналогичен седлоузлу в двумерной системе. Условие невырожденности: $l\neq 0$,
\begin{gather*}
	\beta(\mu)=
	\begin{cases}
		\dot U_1 = \mu+lU_1^2+\dots \\
		\dot U_2 = \lambda_2U_2+\dots \\
		\dot U_3 = \lambda_3U_3+\dots.
	\end{cases}
\end{gather*}

Пусть $\lambda_2,~\lambda_3<0,~l>0,~U_1^2=-\frac{\mu}{l}$:
\begin{figure}[H]
	\centering
	\includegraphics[width=1\linewidth]{fig/fig49.pdf}   
\end{figure}

При $\mu<0$ - устойчивый узел и седло с одномерной сепаратрисой. При $\mu=0$ седлоузел - негрубое состояние равновесия, которое при изменении параметра либо распадается на два, либо исчезает. 

\underline{Случай b:}
\begin{gather*}
	\lambda_{2,3}=\alpha(\mu)\pm i \beta(\mu), \\
	\beta(\mu)=
	\begin{pmatrix}
		\alpha~~-\beta \\
		\beta~~~~\alpha
	\end{pmatrix}
	.
\end{gather*}

Предполагаем, что $Re[\alpha(\mu)]<0$.
\begin{gather*}
	\beta(\mu)=
	\begin{cases}
		\dot U_1 = \mu+lU_1^2+\dots \\
		\dot U_2 = \alpha U_2-\beta U_3 \\
		\dot U_3 = \beta U_2+\alpha U_3.
	\end{cases}
\end{gather*}
\begin{figure}[H]
	\centering
	\includegraphics[width=1\linewidth]{fig/fig50.pdf}   
\end{figure}

При $\mu<0$ - устойчивый фокус и седлофокус. При $\mu=0$ седлофокус-фокус.

Пусть корни все действительные. $\lambda_3<0,~\lambda_2>0,~\mu<0$. Для удобства расположим оси координат следующим образом:
\begin{wrapfigure}[6]{l}{0.5\linewidth} 
	\vspace{0.1em}
	\centering
	\includegraphics[scale=1]{fig/fig51.pdf}
	\vspace{-0.25em}
\end{wrapfigure}

Одно седло имеет одномерную сепаратрису и двумерное устойчивое многообразие, другое - одно неустойчивое двумерное многообразие и устойчивое одномерное.

\begin{figure}[H]
	\centering
	\includegraphics[width=1\linewidth]{fig/fig52.pdf}   
\end{figure}

Многообразия пересекаются, сближаются. Образуется точка, которая при $\mu>0$ исчезает.

27.05
\subsection{Бифуркация коразмерности 1 Андронова-Хопфа}
Предположим, есть состояние равновесия. Его корни комплексно-сопряженные: $\lambda_{1,2}=\alpha(\mu)\pm i \beta(\mu),~\beta(\mu)>0,~\mu=0$ - бифуркационный параметр, который состоит в том, что $\alpha(0)=0$. Условие невырожденности - первая ляпуновская величина $L(\mu)\neq 0$.

Рассмотрим конкретный вид зависимости:
\begin{wrapfigure}[7]{l}{0.42\linewidth} 
	\vspace{0.1em}
	\centering
	\includegraphics[scale=1]{fig/fig53.pdf}
	\vspace{-0.25em}
\end{wrapfigure}

$\lambda_3(\mu)\neq 0,~L(\mu)<0$,
\begin{equation}
	\begin{cases}
		\dot U_1 = \alpha(\mu)U_1-\beta(\mu)U_2+L(\mu)(U_1^2+U_2^2)U_1+\dots \\
		\dot U_2 = \beta(\mu)U_1+\alpha(\mu)U_2+L(\mu)(U_1^2+U_2^2)U_2+\dots \\
		\dot U_3 = \lambda_3(\mu)U_3+\dots
	\end{cases}
	\label{eq:108}	
\end{equation}

 Для определенности $\lambda_3(\mu)<0$ (рассмотреть $\lambda_3(\mu)>0$).

 Фактически, для рассмотрения такой системы нам достаточно знаний о двумерных системах. Из третьего уравнения системы находим, что $U_3=0$ - инвариантная плоскость. Без третьего уравнения остается двумерная система. Легко восстановить фазовые портреты:
 \begin{figure}[H]
	\centering
	\includegraphics[width=1\linewidth]{fig/fig54.pdf}   
\end{figure}

При $\mu=0$ состояние равновесия не центр, а сложный устойчивый фокус. В этом случае динамику системы определяют кубические слагаемые. При $\mu>0$ появляется предельный устойчивый цикл. 

$\lambda_3<0,~U_3 \rightarrow 0$. $U_3$ - устойчивая плоскость (если рассмотреть одномерную систему на прямой).

\begin{center}
    \begin{minipage}{0.32\linewidth}
        \includegraphics[width=\linewidth]{fig/55_1.jpg} 
        \vspace{-50pt}
        \label{fig:1}
    \end{minipage}
\hfill     
    \begin{minipage}{0.3\linewidth}
        \includegraphics[width=\linewidth]{fig/55_2.jpg} 
        \vspace{-50pt}
        \label{fig:1}
    \end{minipage}
\hfill     
    \begin{minipage}{0.3\linewidth}
        \includegraphics[width=\linewidth]{fig/55_3.jpg} 
        \vspace{-50pt}
        \label{fig:1}
    \end{minipage}    
\end{center}

Справа - неустойчивый седлофокус.

$L>0:$
\begin{figure}[H]
	\centering
	\includegraphics[width=1\linewidth]{fig/fig58.pdf}   
\end{figure}

\begin{center}
    \begin{minipage}{0.32\linewidth}
        \includegraphics[width=\linewidth]{fig/fig61.jpg} 
        \vspace{-50pt}
        \label{fig:1}
    \end{minipage}
\hfill     
    \begin{minipage}{0.3\linewidth}
        \includegraphics[width=\linewidth]{fig/fig60.jpg} 
        \vspace{-50pt}
        \label{fig:1}
    \end{minipage}
\hfill     
    \begin{minipage}{0.3\linewidth}
        \includegraphics[width=\linewidth]{fig/fig59.jpg} 
        \vspace{-50pt}
        \label{fig:1}
    \end{minipage}    
\end{center}

Слева седловой предельный цикл, по центру - точка стала неустойчивым сложным фокусом.

\subsection{Бифуркации периодических движений в трехмерном пространстве(бифуркация +1)}
Предположим, есть система третьего порядка, в которой существует предельный цикл $L_{\mu}$, зависящий от $\mu$. Для введения мультипликатора выберем произвольную точку, через нее проведем секущую Пуанкаре. Мультипликаторов три, один из которых тривиальный: $s_1(\mu),~s_2(\mu)$.
\begin{wrapfigure}[7]{l}{0.6\linewidth} 
	\vspace{0.1em}
	\centering
	\includegraphics[scale=1]{fig/fig62.pdf}
	\vspace{-0.25em}
\end{wrapfigure}

Первое - седло-узловая бифуркация предельных циклов (двукратный предельный цикл). Предположим, $s_j(\mu)>0$, и при $\mu<0$ оба мультипликатора расположены внутри единичной окружности. При этом $0<s_2(\mu)<1$.
\begin{figure}[H]
	\centering
	\includegraphics[width=1\linewidth]{fig/fig63.pdf}   
\end{figure}

При $\mu=0: s_1(0)= 1$. Запишем нормальную форму для этой бифуркации. На секущей Пуанкаре вводятся координаты. Тогда поведение вблизи $L_{\mu}$ может быть описано двумерным отображением:
\begin{equation}
	\begin{cases}
		\stackrel{\_}{\xi}= \mu+\xi+\xi^2+\dots \equiv g\\
		\stackrel{\_}{\eta}= s_2(\mu)\eta+\dots
	\end{cases}
	\label{eq:109}	
\end{equation}

Переменные $\xi, \eta$ вводятся в малой $\varepsilon$-окрестности предельного цикла. $\mu$ близко к 0.

g может быть исследовано независимо. Неподвижные точки $\stackrel{\_}{\xi}=\xi$:
$$\xi^2+\mu=0.$$

При $\mu<0: \xi^2=-\mu~\rightarrow \xi=\pm \sqrt{-\mu}$.

Устойчивость:
\begin{gather*}
	g'_{\xi}=1+2\xi, \\
	g''_{\xi \xi}=2,
\end{gather*}
$\xi$ мало, следовательно, $g'_{\xi}>0$. Функция монотонно растет и выпукла вниз.
\begin{figure}[H]
	\centering
	\includegraphics[width=1\linewidth]{fig/fig64.pdf}   
\end{figure}

При $\mu=0 g'=1, s_1=1$ - точка полуустойчивая. 

Из второго уравнения системы получаем, что $\eta=0$ - инвариантная прямая. Любая точка, взятая на этой прямой, перепрыгнет на нее же. Она устойчива:  
\begin{figure}[H]
	\centering
	\includegraphics[width=1\linewidth]{fig/fig65.pdf}   
\end{figure}

Слева направо: на вертикальных прямых действует второе отображение. На них траектории стремятся к $\eta=0$. Затем остается одна неподвижная точка - седлоузел (это не равновесие, а отображение). После чего исчезает.

Это все отображения, т.е набор точек (нарисовано сплошными).
\begin{figure}[H]
	\centering
	\includegraphics[width=1\linewidth]{fig/fig66.pdf}   
\end{figure}

Слева устойчивый (внешний) и седловой (внутренний) предельные циклы. Им соответствуют неподвижные точки. По центру негрубая неподвижная точка и двукратный предельный цикл. Справа - нет предельного цикла.

\subsection{Бифуркация удвоения периода (бифуркация -1)}
Пусть мультипликаторы отрицательные:
\begin{figure}[H]
	\centering
	\includegraphics[width=1\linewidth]{fig/fig67.pdf}   
\end{figure}

Динамику поведения в окрестности предельного цикла можно описать нормальной формой:
\begin{equation}
	\begin{cases}
		\stackrel{\_}{\xi}= -(1+\mu)\xi+\xi^3+\dots \equiv g(\xi,\mu)\\
		\stackrel{\_}{\eta}= s_2(\mu)\eta+\dots
	\end{cases}
	\label{eq:110}	
\end{equation}

Первое уравнение можно исследовать отдельно. Мультипликаторы отрицательны, следовательно, будут перескоки. 

Удобнее начать с отображения $\stackrel{=}{\xi}$, т.е. посмотреть, как ведет себя система через две итерации.
\begin{gather*}
	\stackrel{=}{\xi}=-(1+\mu)\stackrel{\_}{\xi}+\stackrel{\_}{\xi}^3+\dots= \\
	=-(1+\mu)(-(1+\mu)\xi+\xi^3)+(-(1+\mu)\xi+\xi^3)^3+\dots= \\
	=(1+\mu)^2\xi-(1+\mu)\xi^3-(1+\mu)^3\xi^3+\dots=(1+\mu)^2\xi-(1+\mu)((1+\mu)^2+1)\xi^3+\dots
\end{gather*}


Найдем неподвижные точки. Есть тривиальная $\xi=0$ -- она устойчива при $\mu<0$,т.к. мультипликатор меньше единицы. При $\mu=0$ мультипликатор равен единице, точка полуустойчава, при $\mu>0$ неустойчива.

Нетривиальные точки:
\begin{gather*}
	1=(1+\mu)^2-(1+\mu)(2+2\mu+\mu^2)\xi^2, \\
	\cancel{1}=\cancel{1} +2\mu+\mu^2-(1+\mu)(2+2\mu+\mu^2)\xi^2, \\
	\xi^2=\frac{\mu(2+\xcancel{\mu)}}{(1+\cancel{\mu})(2+\xcancel{2\mu}+\xcancel{\mu^2})},
\end{gather*}

Вспоминая, что $\mu$ мало и близко к 0:
\begin{gather*}
	\xi^2 \approx \frac{2\mu}{2}=\mu.
\end{gather*}

При $\mu>0$ существует пара неподвижных точек $\xi=\pm \sqrt{\mu}$:
\begin{figure}[H]
	\centering
	\includegraphics[width=1\linewidth]{fig/fig68.pdf}   
\end{figure}
Слева -- устойчивая точка, по центру -- касание (мультипликатор 1), справа появляются нетривиальные точки.

Если в отображении $g^2$ есть неподвижная точка, то для $g^1$ есть периодическая траектория периода 2.
\begin{figure}[H]
	\centering
	\includegraphics[width=1\linewidth]{fig/fig69.pdf}   
\end{figure}
Синяя - функция последования. Справа -- устойчивая точка с перескоком. По центру -- сложная устойчивая неподвижная точка, траектории близко друг к другу. Справа -- есть периодическая траектория.

В исходной системе (секущая Пуанкаре будет вертикально):
\begin{figure}[H]
	\centering
	\includegraphics[width=1\linewidth]{fig/fig70.pdf}   
\end{figure}
Слева -- устойчивая неподвижная точка, которой соответствует устойчивый предельный цикл. Неподвижная точка стала седловой, устойчивый предельный цикл стал неустойчивым. Появился новый предельный цикл с удвоенным исходным периодом. 

Был исходный предельный цикл, мы слегка поменяли параметр. Траектория начала промахиваться и замкнулась только на втором обороте. 

\subsection{Бифуркация рождения инвариантного тора}
Пусть есть предельный цикл и
\begin{figure}[H]
	\centering
	\includegraphics[width=1\linewidth]{fig/fig71.pdf}   
\end{figure}
В момент бифуркации при $\mu=0$ можно записать: $s_{1,2}=e^{\pm i\omega}$, где $\omega \neq 0,\omega \neq \pi$ (такое уже рассмотрели) и $\omega \neq \frac{2\pi}{3},\omega \neq \frac{\pi}{2}$ (резонансны случаи, описывают бифуркации коразмерности 2 и выше).

Отображение Пуанкаре в полярных координатах:
\begin{equation}
	\begin{cases}
		\stackrel{\_}{\rho}= (1+\mu)\rho+G(\mu)\rho^3+\dots \\
		\stackrel{\_}{\theta}= \theta+\omega+B(\mu)\rho^2+\dots
	\end{cases}
	\label{eq:s9:4}	
\end{equation}

Здесь $G(\mu)$ играет роль первой ляпуновской величины. Предполагаем, что $G(\mu)\neq 0$. Если $G(\mu)< 0$, то неподвижная точка устойчива, если $G(\mu)> 0$ -- неустойчива. 

Рассмотрим $G(\mu)< 0$. 

$B(\mu)$ - некоторая константа, от нее ничего не зависит. Удобно переписать:
\begin{equation}
	\begin{cases}
		\stackrel{\_}{\rho}= (1+\mu)\rho-\rho^3+\dots \\
		\stackrel{\_}{\theta}= \theta+\omega+B(\mu)\rho^2+\dots
	\end{cases}
	\label{eq:s9:5}	
\end{equation}
Исследуем для $\rho$: $\rho=0$, если $\mu<0$ устойчива, если $\mu>0$ -- неустойчива.
\begin{figure}[H]
	\centering
	\includegraphics[width=1\linewidth]{fig/fig73.pdf}   
\end{figure}
\begin{figure}[H]
	\centering
	\includegraphics[width=1\linewidth]{fig/fig72.pdf}   
\end{figure}
Справа $\rho \downarrow: ~\stackrel{\_}{\theta}= \theta+\omega$ -- неподвижная точка, преобразование поворота на угол $\omega$. По середине -- то же самое, характер подхода другой. Справа -- есть замкнутая инвариантная кривая радиуса $\sqrt{\mu}$.
Правильно рисовать не сплошными линиями, а точками, поскольку это не отображение.

\subsection{Бифуркация Неймара-Сакера}
\begin{center}
    \begin{minipage}{0.32\linewidth}
        \includegraphics[width=\linewidth]{fig/fig74.jpg} 
        \vspace{-50pt}
        \label{fig:1}
    \end{minipage}
\hfill     
    \begin{minipage}{0.3\linewidth}
        \includegraphics[width=\linewidth]{fig/fig75.jpg} 
        \vspace{-50pt}
        \label{fig:1}
    \end{minipage}
\hfill     
    \begin{minipage}{0.3\linewidth}
        \includegraphics[width=\linewidth]{fig/fig76.jpg} 
        \vspace{-50pt}
        \label{fig:1}
    \end{minipage}    
\end{center}
Слева -- устойчивый предельный цикл. По центру -- то же самое, но с более плотной накруткой. Справа - двумерный инвариантный тор, внутри которого неустойчивый предельный цикл. 

Поведение на торе: либо периодические движения, когда $\frac{\omega_1}{\omega_0}$(одна близка к предельному циклу, другая связана с существованием тора) есть рациональное число. Когда дробь иррациональна движения квазипериодические.


\end{document}
